%!TEX root=finmath2.tex

\chapter{Модель Хестона: цены европейских опционов}
\label{ch:heston-formula}
\chaptertoc

Модель Хестона (S.~Heston, 1993) "--- это одна из популярных моделей стохастической волатильности.
Ее главное достоинство состоит в том, что в ней имеется формула для цен европейских опционов колл и пут, которая сводится к вычислению некоторого интеграла, что позволяет быстро находить цены опционов численно.

\section{Описание модели}
\subsection{Уравнения, задающие модель}

Далее всегда будем считать, что исходная вероятностная мера $\P$ уже является мартингальной.
Кроме того, для простоты изложений и краткости формул мы будем предполагать, что безрисковая процентная ставка нулевая, а рисковый акив не платит дивиденды (о том, как к такому случаю свести общую модель, см.~раздел \ref{gen:s:forward} лекции \ref{ch:general}).

% \begin{remark}
% Как было сказано в лекции \ref{ch:general}, часто удобнее задавать модель в терминах форвардной, а не спотовой цены рискового актива "--- тогда в формулах не будут в явном виде присутствовать величины $r_t$ и $q_t$.
% При этом процесс  волатильности спотовой цены совпадает с процессом волатильности $T$-форвардной цены для любого $T$ (см.~следствие \ref{gen:c:forward-vol} в лекции \ref{ch:general}).

% Имея это ввиду, мы зададим модель Хестона уравнением для форвардной цены.
% Мы будем опускать верхний индекс $T$ в обозначении процесса $F_t^T$, так как для разных $T$ уравнения будут одинаковы; различаются только начальные условия $F_0^T$.
% \end{remark}

Цена рискового актива в модели Хестона описывается уравнениями
\begin{align}
\label{hes:S}
&dS_t = \sqrt{V_t}S_t d W_t^{(1)},\\
\label{hes:V}
&d V_t = \kappa(\theta-V_t)dt + \xi\sqrt{V_t} d W_t^{(2)},
\end{align}
где процесс $V_t$ называется \emph{стохастической дисперсией} ($\sqrt{V_t}$ "--- \emph{стохастическая волатильность}), $W_t^{(1)},W_t^{(2)}$ "--- стандартные броуновские движения с коэффициентом корреляции $\rho\in(-1,1)$, \te\ $\E(W_t^{(1)}W_t^{(2)}) = \rho t$.
Величины $\kappa,\theta,\xi>0$, $\rho\in(-1,1)$, а также начальное значение $V_0\ge 0$ являются параметрами модели, которые нужно оценить из рыночных данных.
Отметим, что процесс стохастической дисперсии $V_t$ "--- это процесс CIR, рассмотренный в лекции \ref{ch:sde}.

Начальные условия $S_0$ и $V_0$ далее всегда будет считаться строго положительными.

\begin{proposition}
Система уравнений \eqref{hes:S}--\eqref{hes:V} имеет единственное сильное решение $(S,V)$, причем процесс $S_t$ строго положителен.
Двумерный процесс $(S_t,V_t)$ является строго марковским.
Если $2\kappa\theta\ge \xi^2$, то процесс $V_t$ строго положителен. 
\end{proposition}

\begin{proof}
Существование единственного сильного решения уравнения \eqref{hes:V}, а также то, что условие $2\kappa\theta\ge\xi^2$ (условие Феллера) приводит к положительности решения, обсуждалось в лекции \ref{ch:sde}.
Сильное решение уравнения \eqref{hes:S} можно построить в явном виде:
\[
S_t = S_0 e^{\int_0^t \sqrt{V_s} dW_s^{(1)}  - \frac12 \int_0^t V_s ds}.
\]
Проверка того, что это действительно решение проводится по формуле Ито.
Единственность (впрочем, как и существование) вытекает из того, что $S_t$ является \emph{стохастической экспонентой} процесса $d X_t = \sqrt{V_t} dW_t^{(1)}$ и общего результата о существовании и единственности стохастической экспоненты (см.~подробнее приложение \ref{ch:stoch-exp}).

Строго марковское свойство процесса $(S,V)$ следует из существования и единственности решения (см.~теорему \ref{sde:t:strong-markov} в лекции \ref{ch:sde}).
\end{proof}

Можно дать следующую интерпретацию параметрам процесса стохастической дисперсии.
Во-первых, заметим, что
\begin{equation}
\label{hes:EV}  
\E V_t = \theta + (V_0 - \theta)e^{-\kappa t}.
\end{equation}
Эта формула получается, если взять математическое ожидание от обеих частей интегральной формы уравнения \eqref{hes:V}.
Тогда, обозначая, $f(t) = \E V_t$, получаем уравнение $f(t) = V_0 + \kappa\int_0^t (\theta - f(s))ds$, решением которого является \eqref{hes:EV}.

Таким образом, процесс $V_t$ обладает свойством возврата к уровню $\theta$, называемому \emph{долгосрочным средним}.
Параметр $\kappa$ задает \emph{скорость возврата} к $\theta$.
Параметр $\xi$ называют \emph{волатильностью волатильности} (\emph{vol-of-vol}), он контролирует насколько сильны колебания волатильности.


\subsection{Мартингальность процесса цены \difficult}

Напомним (см.~лекцию \ref{ch:general}), что желательным свойством любой модели является настоящая (не локальная) мартингальность цены рискового актива $S_t$.

\begin{proposition}
\label{hes:p:price-martingale}
В модели Хестона процесс $S_t$ является мартингалом.
\end{proposition}

\begin{lemma}[\emph{теорема сравнения} для стохастических дифференциальных уравнений, см.~\cite{KaratzasShreve91}, гл.~5, предложение 2.18]
Пусть $X^{(1)}$ и $X^{(2)}$ "--- сильные решения уравнений
\[
d X_t^{(i)} = a^{(i)}(t, X_t^{(i)}) dt + b(t,X_t^{(i)}) d W_t, \qquad X_0^{(i)} = x_0^{(i)},
\]
заданные на одном вероятностном пространстве.
Предположим, что выполнены следующие условия:
\begin{enumerate}
\item функции $a^{(i)}(t,x)$, $b(t,x)$ непрерывны;
\item $|a^{(i)}(t,x) - a^{(i)}(t,y)| \le K|x-y|$ с некоторой константой $K>0$ для всех $t,x$ и хотя бы одного $i=1,2$;
\item найдется возрастающая функция $h(x)$ такая, что $h(0) = 0$, $\int_0^\epsilon h(x) dx = \infty$ для любого $\epsilon>0$, а также $|b(t,x) - b(t,y)| \le h(|x-y|)$ для всех $x,y$;
\item $x_0^{(1)} \le x_0^{(2)}$;
\item $b^{(1)}(t,x) \le b^{(2)}(t,x)$ для всех $t,x$;
\end{enumerate}
Тогда $X_t^{(1)} \le X_t^{(2)}$ \as\ для всех $t\ge 0$.
\end{lemma}

\begin{proof}[Доказательство предложения~\ref{hes:p:price-martingale}]
Как было отмечено выше, процесс $S_t$ является стохастической экспонентой процесса $d X_t = \sqrt{V_t}dW_t^{(1)}$.
Следовательно, для того, чтобы $S_t$ был мартингалом, достаточно выполнения условия Новикова (см.~приложение~\ref{ch:stoch-exp})
\[
\E \exp\biggl(\frac12 \int_0^T V_t dt\biggr) < \infty.
\]
В свою очередь, для этого достаточно (см.~следствие \ref{stochexp:novikov-corollary} в приложении \ref{ch:stoch-exp}), чтобы для некоторого $\epsilon>0$ и для всех $t\in[0, T-\epsilon]$ было выполнено неравенство
\[
\E \exp\biggl(\frac12 \int_t^{t+\epsilon} V_s ds\biggr) < \infty.
\]
Заметим, что по неравенству Йенсена $\exp(\int_t^{t+\epsilon} \frac 12 V_s ds) \le \frac1\epsilon\int_t^{t+\epsilon} \exp(\frac\epsilon2 V_s) ds$, а поэтому
\begin{equation}
\label{hes:mart-bound}
\E \exp\biggl(\frac12 \int_t^{t+\epsilon} V_s ds\biggr) 
\le \frac1\epsilon \int_t^{t+\epsilon} \E e^{\frac\epsilon2 V_s} ds.
\end{equation}
Оценим $\E e^{\frac\epsilon2 V_s}$.
Положим $n = 4\lceil \kappa\theta/\xi^2 \rceil$, $ \alpha = \kappa/2$ и рассмотрим процесс
\[
d V_t' = (n\xi^2/4 - 2\alpha V'_t) dt + \xi\sqrt{V_t'} d W_t^2, \qquad V_0' = V_0.
\]
Параметры $n$ и $\alpha$ выбраны так, что для всех $v\ge0$ выполнено неравенство $n\xi^2/4-2\alpha v \ge \kappa (\theta-v)$.
Тогда, используя теорему сравнения, находим, что 
\[
V_t' \ge V_t\ \text{\as\ для всех $t\ge 0$}.
\]

Таким образом, достаточно будет оценить сверху подходящим образом ожидание $\E e^{\frac\epsilon2 V_t'}$.
Для этого покажем, что верно равенство по распределению $V_t' \stackrel{d}{=} U_t$ для процесса
\[
U_t = \sum_{i=1}^n (Y_t^{(i)})^2,
\]
где $Y^{(i)}$ "--- независимые процессы Орштейна"--~Уленбека
\[
d Y_t^{(i)} = -\alpha Y_t^{(i)} dt +  \frac\xi2 d B_t^{(i)}
\]
с начальными условиями $Y_0^{(i)}$ такими, что $\sum_i (Y_0^{(i)})^2 = V_0'$.

Действительно, применяя формулу Ито, получаем
\[
d U_t = \left(\frac{n\xi^2}{4}-2\alpha U_t\right) dt + \sum_{i=1}^n Y_t^{(i)} d B_t^{(i)} = 
\left(\frac{n\xi^2}{4}-2\alpha U_t\right) dt + \xi \sqrt{U_t} d Z_t
\]
с процессом 
\[
d Z_t = \frac{\sum_i Y_t^{(i)}}{\xi \sqrt{U_t}} d B_t^i.
\]
Процесс $Z_t$ является локальный мартингалом.
Кроме того, его квадратическая характеристика равна $t$ (\te\ $Z_t^t -t $ является локальным мартингалом), в чем нетрудно убедиться по формуле Ито, показав, что дифференциал $d(Z_t^2 - t)$ не содержит члена с $dt$.
Отсюда по теореме Леви%
\footnote{Непрерывный локальный мартингал с нулевым начальным значением является стандартным броуновским движением тогда и только тогда, когда его квадратическая характеристика равна $t$.}
следует, что $Z_t$ "--- стандартное броуновское движение и, значит, $U_t$ удовлетворяет тому же уравнению, что и $V_t'$, \te\ является процессом CIR с теми же параметрами.
Таким образом, $U$ и  $V'$  совпадают по распределению.

Так как процессы Орнштейна"--~Уленбека $Y_t^i$ имеют гауссовские распределения, то для любого $c>1$ и всех $t\in[0,T]$ найдется малое $\epsilon>0$ такое, что $\E e^{\frac\epsilon2 (Y_t^i)^2} < c$, и тогда
\[
\E e^{\frac\epsilon2 V_t'} \le c^n.
\]
Отсюда следует, что правая часть \eqref{hes:mart-bound} конечна, что нам и требовалось.
\end{proof}


\subsection{Условие для безарбитражности модели \difficult}
Выше мы предполагали, что исходная вероятностная мера $\P$ уже является мартингальной.
Если же относительно меры $\P$ процесс цены $S_t$ имеет произвольный коэффициент сноса, то вопрос о существовании эквивалентной мартингальной меры и выполнении свойства безарбитражности (NFLVR) становится деликатным и зависит от выполнения условия Феллера.
Для простоты изложения, ограничимся случаем постоянной безрисковой ставки $r$ и постоянного коэффициента сноса $\mu$.

\begin{proposition}[см.~\cite{DesmettreLeobacherRogers21}]
Пусть относительно меры $\P$ цена рискового актива задается уравнениями
\begin{align*}
&d S_t = \mu S_t dt + \sqrt{V_t} d W_t^{(1)},\\
&d V_t = \kappa(\theta-V_t)dt + \sigma \sqrt{V_t} d W_t^{(2)},
\end{align*}
где $\mu\neq r$.
Тогда ЭЛММ в такой модели существует в том и только том случае, когда выполнено условие Феллера $2\kappa\theta\ge \xi^2$.
\end{proposition}


\section{Формула для цены европейских опционов}

В этом разделе будет получена формула для цен ванильных опционов колл и пут в модели Хестона, которая сводится к некоторому интегралу.
Этот интеграл можно найти методами численного интегрирования.
Идея будет состоять в том, чтобы сначала получить аналитическую формулу для \emph{характеристической функции} распределения логарифма цены, обратив которую найти вероятность исполнения опциона, а из нее уже выразить его цену.
Вычисление интеграла будет возникать при обращении характеристической функции.

Далее будем рассматривать только опционы колл.
Цены пут можно найти аналогично или воспользоваться паритетом цен колл-пут.


\subsection{Формулировка теоремы}

Зафиксируем время исполнения $T$. Определим функции
\begin{align*}
&\phi(t,x,v;u) = \exp(C(T-t,u) + D(T-t,u)v + iu x),\\
&\tilde \phi(t,x,v;u) = \frac{\phi(t,x,v;u-i)}{\phi(t,x,v;-i)},
\end{align*}
где
\begin{align*}
&C(\tau,u) = \frac{\kappa\theta}{\sigma^2}
  \biggl(
    (\kappa- \rho\sigma iu - d(u))\tau 
    - 2\ln\biggl(\frac{1-g(u)e^{-d(u)\tau}}{1-g(u)}\biggr)
  \biggr),\\
&D(\tau,u) = \frac{\kappa - \rho\sigma i u - d(u)}{\sigma^2}
  \biggl(\frac{1-e^{-d(u)\tau}}{1-g(u)e^{-d(u)\tau}}\biggr),\\
&d(u)=\sqrt{(\rho\sigma iu - \kappa)^2 + \sigma^2(iu + u^2)},\quad
  g(u) = \frac{\rho\sigma iu- \kappa +d(u)}{\rho\sigma iu - \kappa -d(u)}
\end{align*}
(здесь везде $i=\sqrt{-1}$).

Как будет видно далее, $\phi$ является условной характеристической функцинй $\E (e^{iu X_T}\mid X_t=x, V_t=v)$, где $X_t=\ln F_t$, а $\tilde \phi$ "--- это та же характеристическая функция, но вычисленная по другой мере.

\begin{theorem}[С.~Хестон]
Пусть безрисковая процентная ставка равна 0, а базовый актив не выплачивает дивиденды.
Тогда цена опциона колл со страйком $K$ и временем исполнения $T$ в момент времени $t$ равна
\begin{equation}
\label{hes:call-repr}
C_t = f(t,\ln S_t,V_t),
\end{equation}
где функция $f$ имеет вид
\begin{equation}
\label{hes:call-undiscounted}
f(t,x,v) = e^x \tilde \Pi(t,x,v) - K \Pi(t,x,v)
\end{equation}
с функциями $\tilde\Pi$, $\Pi$ определенными по формулам
\begin{align*}
&\Pi(t,x,v) = \frac12 + \frac1\pi \int_0^\infty
  \mathrm{Re}\biggl(\frac{e^{-iu \ln K}\phi(t,x,v;u)}{iu}\biggr) du,\\[0.5em]
&\tilde\Pi(t,x,v) = \frac12 + \frac1\pi \int_0^\infty
  \mathrm{Re}\biggl(\frac{e^{-iu \ln K}\tilde \phi(t,x,v;u)}{iu}\biggr) du.
\end{align*}
\end{theorem}


\subsection{Вспомогательные результаты}
\subsubsection{Представление цены опциона через вероятности исполнения}

Следующая лемма выражает цену опциона колл через вероятности исполнения по двум мерам.
Отметим, что она выполнена не только в модели Хестона, а вообще в любой безарбитражной модели, при условии что безрисковая ставка равна 0, рисковый актив не выплачивает дивиденды, а его цена является настоящим мартингалом относительно меры $\P$.

Для заданного $T$ определим меру $\tilde\P$ с плотностью
\[
\frac{d\tilde\P}{d\P} = \frac{S_T}{S_0}. 
\]
% Обозначим процесс логарифма цены
% \[
% X_t = \ln S_t.
% \]
% Для краткости будем использовать следующие обозначения для условного ожидания и вероятности:
% \[
% \E_{t,x,v}(\,\cdot\,) = \E(\,\cdot\mid X_t=x, V_t=v), \qquad
% \P_{t,x,v}(\,\cdot\,) = \P(\,\cdot\mid X_t=x, V_t=v).
% \]
% Аналогичные обозначения будут использоваться и по мере $\tilde \P$.

\begin{lemma}
Цена опциона колл со страйком $K$ и временем исполнения $T$ в момент $t$ представима в виде 
\[
C_t = S_t \tilde\P(S_T\ge K\mid \F_t) - K\P(S_T\ge K\mid \F_t).
\]
% \eqref{hes:call-repr}--\eqref{hes:call-undiscounted} с функциями
% \begin{align*}
% &\Pi(t,x,v) = \P_{t,x,v}(X_T\ge \ln K),\\
% &\tilde\Pi(t,x,v) = \tilde\P_{t,x,v}(X_T\ge \ln K).
% \end{align*}
\end{lemma}

\begin{proof}
Заметим, что
\[
\E^\P((S_T - K)^+\mid \F_t) = \E^\P(S_T \I(S_T\ge K) \mid \F_t) - K\P(S_T\ge K\mid \F_t).
\]
Пусть $Z = S_T/S_0$ обозначает плотность $d\tilde\P/d\P$. 
Тогда из формулы пересчета для условных математических ожиданий%
\footnote{Для двух мер $\tilde\P \sim \P$ с плотностью $Z=d\tilde\P/d\P$, произвольной $\sigma$-алгебры $\G\subseteq\F$ и $\tilde\P$-интегрируемой случайной величины $\xi$ верно равенство (см.~\cite{Shiryaev04}, гл.~II, \S\,7, теорема 6)
\[
\E^{\tilde \P}(\xi\mid \G) = \frac{\E^{\P}(Z\xi\mid \G)}{\E^\P(Z\mid \G)}.
\]
}
находим
\begin{multline*}
\tilde\P(S_T\ge K\mid \F_t) = \E^{\tilde\P} (\I(S_T \ge K) \mid \F_t)
= \frac{\E^{\P}(Z \I(S_T \ge K)\mid \F_t)}{\E^{\P}(Z\mid \F_t)} \\= S_t \E^{\P}(S_T \I(S_T\ge K)\mid \F_t),
\end{multline*}
где воспользовались тем, что $\E(Z\mid \F_t) = S_t/S_0$, так как $S_t$ "--- мартингал относительно меры $\P$.
Подставляя получившееся равенство в формулу выше, получаем доказываемое утверждение.
\end{proof}
