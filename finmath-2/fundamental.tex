%!TEX root=finmath2.tex

\chapter{Фундаментальные результаты}
\label{ch:general}
\chaptertoc

Эта лекция посвящена фундаментальным теоретическим результатам.
Основная ее цель "--- показать, как определить справедливую цену платежного обязательства, и объяснить, почему ее можно вычислить в виде ожидания дисконтированной выплаты по эквивалентной мартингальной мере.


\section{Общая модель рынка}

Пусть дано фильтрованное вероятностное пространство $(\Omega,\F,(\F_t)_{t\in[0,T]},\P)$ со стандартным $d$-мерным броуновским движением $W_t=(W_t^1,\dots,W_t^d)$. 
Горизонт времени $T$ конечен.

Рынок в модели состоит из одного безрискового актива и $N$ рисковых активов.
Цена безрискового актива задается процессом
\[
d B_t = r_tB_t dt \qquad\Bigl[B_t = B_0 e^{\int_0^t r_s ds}\Bigr].
\] 
Без ограничения общности считается, что $B_0=1$.
В этой лекции процентная ставка $r_t$ случайная, хотя далее в курсе она, как правило, будет детерминированной (но не постоянной по времени).
Всегда будем предполагать, что $r_t$ ограничена снизу, \te\ $r_t\ge\underline r$ для некоторой константы $\underline r$.

Цены рисковых активов задаются процессами Ито
\begin{equation}
\label{gen:risky}
d S_t^n = \mu_t^n dt + \sum_{i=1}^d \sigma^{ni}_t d W_t^i,
\end{equation}
где $\mu_t^n$, $\sigma_t^{ni}$ "--- некоторые случайный процессы, для которых стохастические интегралы в формуле выше корректно определены.\footnote{Отметим, что здесь в коэффициенты сноса и диффузии не входит множитель $S_t$, как, например, в модели \bs.
Это дает большую общность рассуждений.}
Будем считать, что рисковые активы могут выплачивать дивиденды с интенсивностями, задаваемыми случайными процессами $q_t^n\ge0$.

\begin{remark}
Уравнение \eqref{gen:risky} представляет общую форму модели рынка, где цены рисковых активов задаются процессами Ито.
Однако, в дальнейших лекциях мы, в основном, будем иметь дело с моделями, содержащими один рисковый актив с ценой
\begin{equation}
\label{gen:risky-simple}
d S_t^n = \mu_t^n dt + \sigma_t d W_t,
\end{equation}
где $W_t$ "--- одномерное броуновское движение, при этом процесс $\sigma_t$ управляется другим броуновским движением, которое коррелированно с $W_t$.
\end{remark}

\begin{definition}
\label{gen:d:strategies}
\emph{Торговой стратегией} называется измеримый согласованный%
\footnote{В литературе часто считают, что стратегии задаются \emph{предсказуемыми} процессами.
Однако если цены задаются процессами Ито, то модели с согласованными и предсказуемыми стратегиями будут, в сущности, эквивалентны.}
процесс $\pi=(\pi_t)_{t\in[0,T]}$, где $\pi_t=(G_t,H_t^1,\dots,H_t^N)$.
\emph{Стоимостью портфеля} стратегии $\pi$ называется процесс
\[
V_t^\pi = G_tB_t + \sum_{n=1}^N H_t^n S_t^n.
\]
Стратегия $\pi$ называется \emph{самофинансируемой}, если корректно определены интегралы\footnote{Первые два интеграла понимаются как потраекторные интеграла Лебега, а третий интеграл "--- как стохастический интеграл по процессу Ито.
Они корректно определены при выполнении условий
$\int_0^T r_t B_t|G_t| dt < \infty$,
$\int_0^T \bigl(|\mu_t^n H_t^n| + |q_t^n H_t^n|\bigr)dt < \infty$,
$\int_0^T (\sigma_t^{ni} H_t^n)^2 dt < \infty$.}
$\int_0^t G_t d B_t$, $\int_0^t q_t^n H_t^n S_t^n dt$, $\int_0^t H_t^n dS_t^n$ и выполнено равенство (понимаемое в интегральном смысле)
\begin{equation}
\label{gen:sf}
d V_t^\pi = G_t d B_t + \sum_{n=1}^N H_t^n (q_t^n S_t^n dt + d S_t^n).
\end{equation}
Стратегия $\pi$ называется \emph{допустимой}, если существует константа $c$ такая, что $V_t^\pi \ge c$ для всех $t\in[0,T]$.
\end{definition}

Далее под словом <<стратегия>> мы будем всегда понимать допустимую самофинансируемую стратегию, если не оговорено иного.

Определение стоимости портфеля и условие самофинансируемости аналогичны соответствующим определениям, которые были даны для модели \bs\ в курсе <<Введение в финансовую математику>>, поэтому мы не будем подробно останавливаться на их интерпретации.

Что касается допустимости, то в том курсе использовалось более сильное условием (процесс $H_tS_t$ квадратично интегрируем относительно эквивалентной мартингальной меры), которое в общей модели не удобно, так как в ней нет явного выражения для ЭММ.
Напомним, что условие допустимости необходимо, чтобы избежать арбитражных возможностей.

\begin{definition}
\emph{Дисконтированными ценами рисковых активов с учетом дивидендов} называются процессы 
\[
\tilde S_t^n = e^{\int_0^t q_s^n ds}\frac{S_t^n}{B_t} = e^{\int_0^t (q_s^n -r_s)ds}S_t^n.
\]
\emph{Дисконтированной стоимостью} портфеля стратегии $\pi$ называется процесс 
\[
\tilde V_t^\pi = \frac{V_t^\pi}{B_t} = e^{-\int_0^t r_sds} V_t^\pi.
\] 
\end{definition}

% \begin{proposition}
% Стратегия $\pi$ является самофинансируемой тогда и только тогда, когда 
% \begin{equation}
% \label{gen:sf-discounted}
% d\tilde V_t^\pi = \sum_{n=1}^N e^{-\int_0^t q_s^n ds} H_t^nd\tilde S_t^n.
% \end{equation}
% \end{proposition}
% Доказательство этого результата практически повторяет доказательство аналогичного утверждения для модели \bs\ (см.~лекцию 10 во <<Введении в финансовую математику>>), поэтому мы его опустим.


\section{Отсутствие арбитража и безарбитражные цены}

Следующая задача "--- определить, что понимать под справедливой ценой платежного обязательства.
Так как большинство моделей, которые мы будем рассматривать, являются неполными, то понятие цены репликации, которым мы пользовались в модели \bs, здесь не является подходящим. 
Вместо него мы будем пользоваться понятием \emph{безарбитражной} цены.
Для этого сначала обсудим понятие арбитража и условия его отсутствия.


\subsection{Первая фундаментальная теорема финансовой математики}

Напомним, что один из центральных результатов для моделей рынка с дискретным временем "--- \emph{первая фундаментальная теорема} финансовой математики "--- утверждает, что  отсутствие арбитража равносильно существованию эквивалентной мартингальной меры.
В непрерывном времени ситуация становится сложнее, и для того, чтобы получить аналогичный результат, возникает необходимость ввести новые понятия.

% Определение наиболее близкое к статье DS-94
% \begin{definition}
% Будем говорить, что в модели рынка выполнено \emph{условие NFLVR} (no free lunch with vanishing risk "--- отсутствие бесплатного ланча с исчезающим риском) и называть такой рынок \emph{безарбитражным}, если не существует $\F_T$"=измеримой случайной величины $X\in L^\infty$ и последовательности $\F_T$"=измеримых случайных величин $X^n\in L^\infty$ таких, что 
% \begin{alphenum}
% \item $X\ge 0$;
% \item $\P(X>0) > 0$;
% \item $X^n \le V_T^{\pi^n}$ для некоторых стратегий $\pi^n$ c $V_0^{\pi^n} = 0$;
% \item $X^n \to X$ в $L^\infty$ при $n\to\infty$.
% \end{alphenum}
% \end{definition}
% \begin{remark}
% Напомним, что $L^\infty$ "--- это пространство классов эквивалентности существенно ограниченных величин с нормой $\|X\|_{L^\infty} = \inf\{c\in\R : |X| \le c\ \as\}$. 
% \end{remark}

% Эквивалентное, но более простое определение
\begin{definition}
Будем называть $\F_T$-измеримую случайную величину $X$ \emph{бесплатным ланчем с исчезающим риском} (free lunch with vanishing risk, FLVR), если
\begin{alphenum}
\item $X\ge 0$ \as,
\item $\P(X>0)>0$,
\item для любого $\epsilon>0$ найдется стратегия $\pi$ такая, что $V_0^\pi=0$, $V_T^\pi \ge X-\epsilon$ \as
\end{alphenum}

Будем говорить, что в модели рынка выполнено \emph{условие отсутствия бесплатного ланча с исчезающим риском} (no free lunch with vanishing risk, NFLVR) и называть такой рынок \emph{безарбитражным}, если бесплатного ланча с исчезающим риском не существует.
\end{definition}

Смысл условия NFLVR состоит в том, что, начиная из нулевого портфеля, нельзя получить прибыль с риском, стремящемся к нулю (прибыль равна $X$, а <<исчезающий>> риск равен $\epsilon$, который можно выбрать сколь угодно малым).

NFLVR является более сильным условием, чем простое условие отсутствие арбитража (NA).
Напомним, что NA означает, что не существует такой стратегии $\pi$, что $V_0^\pi = 0$, 
$V_T^\pi \ge 0$ и $\P(V_T^\pi > 0) > 0$.
Если NA не выполнено, то $X=V_T^\pi$ и будет бесплатным ланчем с исчезающим риском. 

\begin{definition}
\emph{Эквивалентной локальной мартингальной мерой} (ЭЛММ) называется вероятностная мера $\Q\sim\P$, относительно которой процессы $\tilde S^n$ являются локальными мартингалами.
\emph{Эквивалентной мартингальной мерой} (ЭММ) называется вероятностная мера $\Q\sim\P$, относительно которой процессы $\tilde S^n$ являются мартингалами.
\end{definition}

\begin{remark}
Любая ЭММ является ЭЛММ, но обратное не верно.
\end{remark}

Следующая теорема вытекает из классического варианта первой фундаментальной теоремы финансовой математики, доказанного Ф.~Дельбаном и В.~Шахермайером (F.~Delbaen, W.~Schachermayer, 1994) для моделей рынка, в которых цены активов задаются семимартингалами.

\begin{theorem}[первая фундаментальная теорема; теорема Дельбана"--~Шахермайера]
Свойство NFLVR равносильно существованию ЭЛММ.
\end{theorem}

Мы дадим доказательство только легкой импликации $\exists\,\Q \implies \text{NFLVR}$.
В другую сторону доказательство очень трудное.

\begin{proof}
Предположим, что ЭЛММ существует, но условие NFLVR не выполнено.
Возьмем последовательность стратегий $\pi^n$ таких, что $V_0^{\pi^n} = 0$ и $V_T^{\pi^n} \ge X - 1/n$, где $X$ "--- бесплатный ланч с исчезающим риском. 
Тогда
\[
0 < \E^{\Q}\frac{X}{B_T} \le \lim_{n\to\infty} \E^{\Q}\frac{V^{\pi^n}_T}{B_T} 
 \le \lim_{n\to\infty} V_0^{\pi^n} = 0.
\]
Здесь во втором неравенстве воспользовались тем, что $1/(nB_T) \to 0$ равномерно, так как предполагается, что процентная ставка $r_t$ ограничена снизу, что влечет ограниченность сверху величины $1/B_T$.
В третьем неравенстве воспользовались тем, что $\tilde V_t^{\pi^n}$ является локальным мартингалом, ограниченным снизу, и, следовательно, является супермартингалом (см.~раздел 1.7 в лекции 7 во <<Введении в финансовую математику>>).
Получаем противоречие: $0<0$.
% Старое доказательство для старого определения NFLVR
% Тогда для последовательности $X^n$, реализующей бесплатный ланч с исчезающим риском, имеем
% \[
% 0 < \E^{\Q}\frac{X}{B_T} = \lim_{n\to\infty} \E^{\Q}\frac{X^n}{B_T} 
% \le \lim_{n\to\infty} \E^{\Q} \frac{V_T^{\pi^n}}{B_T} \le \lim_{n\to\infty} V_0^{\pi^n} = 0.
% \]
% Здесь в первом неравенстве воспользовались тем, что $\P(X\ge 0)=1$, $\P(X>0)>0$, а также эти свойства выполнены и относительно $\Q$ в силу эквивалентности мер.
% В следующем за ним равенстве учли, что $X^n\to X$ в $L^\infty$, а, следовательно, $X^n/B_T\to X/B_T$, так как предполагается, что процентная ставка $r_t$ ограничена снизу, что влечет ограниченность сверху величины $1/B_T$.
% В последнем неравенстве воспользовались тем, что $\tilde V_t^{\pi^n}$ является локальным мартингалом, ограниченным снизу, и, следовательно, является супермартингалом. Остальные два перехода очевидны.
%
% В итоге получаем противоречие: $0<0$.
\end{proof}

Покажем, как преобразуются процессы цен рисковых активов при замене меры на ЭЛММ.

\begin{proposition}
\label{gen:p:emm-repr}
Пусть $\Q$ является ЭЛММ. Тогда верно представление
\begin{equation}
\label{gen:price-emm}
dS_t^n = (r_t-q_t^n) S_t^n dt + \sum_{i=1}^d \sigma_t^{ni} d\tilde W_t^i, 
\end{equation}
где $\tilde W_t = (\tilde W_t^1,\dots,\tilde W_t^d)$ "--- стандартное броуновское движение относительно $\Q$.
\end{proposition}

\begin{proof}
Заметим, что процессы $\tilde S_t^n$ относительно $\P$ имеют вид
\begin{multline*}
d\tilde S_t^n = d(e^{\int_0^t (q_s^n-r_s) ds}S_t^n) = (q_t^n-r_t)\tilde S_t^n dt + 
e^{\int_0^t (q_s^n-r_s) ds} dS_t^n \\ = 
\bigl((q_t^n-r_t)\tilde S_t^n + e^{\int_0^t (q_s^n-r_s)} \mu_t^n\bigr) dt + 
e^{\int_0^t (q_s^n-r_s) ds} \sum_{i=1}^d \sigma_t^{ni} d W_t^i.
\end{multline*}
При эквивалентной замене меры броуновское движение переходит в броуновское движение со сносом (по теореме Гирсанова для локальных мартингалов и теореме Леви, см.\ \cite{JacodShiryaev94}, гл.~III, теорема~3.11 и гл.~II, теорема 4.4), \te\ справедливо представление $W_t^i = A_t^i + \tilde W_t^i$, где $\tilde W_t = (\tilde W_t^1,\dots,\tilde W_t^d)$ "--- стандартное броуновское движение относительно $\Q$, а $A_t^i$ "--- непрерывный согласованный процесс ограниченной вариации.
Так как процесс $\tilde S_t^n$ должен иметь нулевой снос относительно $\Q$, то он имеет вид $d\tilde S_t^n = e^{\int_0^t (q_s^n-r_s) ds} \sum_{i=1}^d \sigma_t^{ni} d\tilde W_t^i$. 
Применяя формулу Ито к $S_t^n = e^{\int_0^t (r_s-q_s^n) ds} \tilde S_t^n$, получаем представление \eqref{gen:price-emm}.
\end{proof}

Чтобы найти ЭЛММ, можно воспользоваться теоремой Гирсанова.
Покажем, как это сделать на конкретном примере.

\begin{example}
Пусть модель задается уравнениями
\begin{align*}
&d S_t = \mu_t S_t dt + \sigma_t S_t d W^1_t,\\
& d \sigma_t = a_t dt + b_t d W^2_t,
\end{align*}
где $W_t^1$, $W_t^2$ "--- независимые броуновские движения. 

Определим броуновские движения со сносом
\[
d \tilde W_t^1 = \nu_t dt + dW_t^1, \qquad d \tilde W_t^2 = \rho_t dt + d W_t^2,
\]
где возьмем $\nu_t = (\mu_t - r_t+q_t)/\sigma_t$, a $\rho_t$ выберем произвольным.
Будем предполагать, что процессы $\nu_t$, $\rho_t$ таковы, что $\int_0^T |\nu_t| dt < \infty$ и $\int_0^T|\rho_t|dt < \infty$, и, следовательно, процессы $\tilde W_t^i$ корректно определены.
Тогда
\begin{align*}
&d S_t = (r_t-q_t) S_t dt + \sigma_t S_t d \tilde W^1_t,\\
& d \sigma_t = (a_t -b_t\rho_t) dt + b_t d \tilde W^2_t.
\end{align*}

Если применима теорема Гирсанова (например, когда выполнено условие Новикова), то можно найти меру $\Q\sim\P$, относительно которой $\tilde W_t$ будем двумерным броуновским движением без сноса.
Тогда $\Q$ будет ЭЛММ, что нетрудно увидеть, применяя формулу Ито к $\tilde S_t$, как это сделано в доказательстве предложения \ref{gen:p:emm-repr}.

Заметим, что в рассматриваемой модели из произвольности процесса $\rho$ видно, что ЭЛММ не единственна.
\end{example}


\subsection{Безарбитражные цены}

\begin{definition}
Пусть рынок из активов $(B,S^1,\dots,S^N)$ является безарбитражным.
Тогда \emph{безарбитражной ценой} платежного обязательства европейского типа с выплатой $X$ в момент времени $T$ называется процесс $V_t^X$ такой, что $V_T^X = X$ и расширенный рынок $(B,S^1,\dots,S^N,S^{N+1})$, где $S^{N+1}_t = V_t^X$ и актив $N+1$ не выплачивает дивиденды, является безарбитражным.
\end{definition}

\begin{remark}
Безарбитражная цена может быть не единственной.
\end{remark}

\begin{proposition}
\label{gen:p:na-price}
Если $\Q$ является ЭЛММ, то для любого платежного обязательства $X$ с $\E^\Q |X| < \infty$ процесс 
\[
V_t = B_t \E^\Q \left(\frac{X}{B_T} \;\bigg|\; \F_t\right)
\]
является безарбитражной ценой $X$.
\end{proposition}

\begin{proof}
Процесс $V_t/B_t$ является $\Q$-мартингалом (так как он является мартингалом Леви), а в расширенном рынке $\tilde S_t^{N+1} = V_t/B_t$. 
Следовательно, $\Q$ является ЭЛММ для расширенного рынка.
Тогда по первой фундаментальной теореме расширенный рынок безарбитражен.
\end{proof}

\begin{remark}
В предложении \ref{gen:p:na-price} есть небольшой пробел: в определении модели мы предполагаем, чтобы процессы цен рисковых активов являлись процессами Ито.
Поэтому, чтобы предложение было формально верным, необходимо еще потребовать, чтобы процесс $S_t^{N+1} = V_t$ был представим в таком виде, что в общем случае не гарантируется автоматически. 

Этого затруднения можно избежать, если изначально рассматривать модель, где цены рисковых активов задаются непрерывными локальными мартингалами: первая фундаментальная теорема остается верной для таких моделей, и, следовательно, верно также предложение \ref{gen:p:na-price}.
Однако тогда для формулировки условия самофинансирования пришлось бы обсудить конструкцию интеграла по локальному мартингалу, которая не очень-то проста, но в дальнейшем нам не потребуется.
\end{remark}

Предложение \ref{gen:p:na-price} ничего не говорит о том, какую ЭЛММ (если их несколько) использовать для оценки производных инструментов.
На практике применяют следующие соображения, которым мы тоже будем следовать в дальнейшем.
\begin{itemize}
\item Для вычисления цен всех деривативов используют одну и ту же ЭЛММ $\Q$.
Часто эту меру выбирают таким образом, чтобы цены ликвидных деривативов, вычисленные в модели, были как можно более близкими к наблюдаемым рыночным ценам (\te\ стремятся уменьшить статическую ошибку).
Другим критерием выбора ЭЛММ может быть, например, получение правильной динамики цен производных инструментов или динамики характеристик поверхности волатильности таких как наклон, кривизна и \tp

\item Наилучшая ЭЛММ выбирается не из всего множества ЭЛММ, а из некоторого параметрического подмножества, при этом обычно  ограничиваются рассмотрением только ЭММ.
Это приводит к задаче нахождения комбинации параметров, дающих наименьшую ошибку в ценах ликвидных инструментов (или наиболее правильную динамику цен).
\end{itemize}

\medskip
В заключение раздела приведем результат о том, что для реплицируемых платежных обязательств в качестве справедливой (\te\ безарбитражной) цены можно взять цену репликации.

\begin{proposition}
\label{gen:p:replication}
Пусть рынок из активов $(B,S^1,\dots,S^N)$ является безарбитражным, а платежное обязательство $X$ реплицируемо с помощью стратегии $\pi_t = (G_t,H_t^1,\dots,H_t^N)$, у которой процессы $G_t$ и $H_t^n$, $n=1,\dots,N$, ограничены.
Тогда процесс $V_t^\pi$ является безарбитражной ценой $X$.
\end{proposition}
\begin{proof}
Идея доказательства состоит в том, что если на расширенном рынке имелся бы арбитраж, то он имелся бы и на исходном рынке, так как любая стратегия на расширенном рынке может быть преобразована в стратегию на исходном рынке путем репликации актива $N+1$ активами из исходного рынка. 

Дадим строгое доказательство.
Предположим, что для расширенного рынка с ценой $S_t^{N+1} = V_t^\pi$ не выполнено условие NFLVR.
Пусть $X$ "--- бесплатный ланч с исчезающим риском.
Рассмотрим стратегии $\pi^k = (g_t^k,h_t^{k,1},\dots,h_t^{k,N+1})$ такие, что $V_0^{\pi^k} = 0$ и $V_T^{\pi^k} \ge X-1/k$.
Имеем
\[
V_t^{\pi^k} = g_t^k B_t + \sum_{n=1}^{N+1} h_t^{k,n} S_t^n
= (g_t^k + h_t^{k,N+1}G_t)B_t + \sum_{n=1}^{N} (h_t^{k,n} + h_t^{k,N+1}H_t^n)S_t^n,
\]
где воспользовались тем, что $S_t^{N+1} = G_tB_t + \sum_{n=1}^N H_t^nS_t^n$.

Пусть $\tilde \pi^k_t = (g_t^k + h_t^{k,N+1}G_t,\ h_t^{k,1} + h_t^{k,N+1}H_t^1,\ \dots,\ h_t^{k,N} + h_t^{k,N+1}H_t^N)$ "--- стратегии на исходном рынке.
Из ограниченности процессов $G_t$, $H_t^n$ следует, что интегралы, присутствующие в условии самофинансируемости для $\tilde \pi^k$, корректно определены, и, следовательно, $\tilde\pi^k$ являются самофинансируемыми, что следует из самофинансируемости $\pi^k$ и $\pi$.
Кроме того, из допустимости $\pi^k$ следует допустимость $\tilde \pi^k$.
Но тогда получаем, что $X$ "--- бесплатный ланч с исчезающим риском на исходном рынке. 
Противоречие.
\end{proof}

\begin{remark}
\label{gen:r:replication}
В общей модели рынка в дискретном времени цены репликации обладают важным свойством: на безарбитражном рынке у любого реплицируемого платежного обязательства цена репликации не зависит от реплицирующей стратегии; в частности, безарбитражная цена реплицируемого платежного обязательства однозначно определена и совпадает с ценой репликации (см.~лекцию 4 во <<Введении в финансовую математику>>).

В непрерывном времени это неверно.
Даже в модели \bs\ с классом допустимых стратегий как в определении \ref{gen:d:strategies} (\te\ $V_t^\pi\ge c$) и, для простоты, с нулевой безрисковой ставкой, можно построить допустимую самофинансируемую стратегию $\pi$ такую, что $V_0^\pi > 0$ и $V_T^\pi = 0$.

Для этого возьмем самофинансируемую стратегию $\pi$, у которой $V_0^\pi = 1$ и
\[
H_t = -\frac{1}{(1-t)S_t} \I(t\le\tau),
\]
где $\tau = \inf\{t: \int_0^t (1-s)^{-1} d W_s = 1\}$. 
Аналогично примеру в лекции 10 во <<Введении в финансовую математику>> получаем, что $V_T^\pi=0$.
Таким образом, платежное обязательство $X=0$ реплицируемо двумя способами: с помощью стратегии $\pi$ и с помощью нулевой стратегии.

Этой проблемы можно избежать, если использовать более сложное определение класса допустимых стратегий.
Подробное изложение можно найти в книге \cite{EberleinKallsen19}, гл.~11, разд.~7.
В нашем курсе этот материал не потребуется.
\end{remark}


\subsection{Полнота рынка и вторая фундаментальная теорема}

\begin{definition}
Безарбитражная модель рынка называется \emph{полной}, если любое $\F_T$-измеримое ограниченное платежное обязательство $X$ реплицируемо.
\end{definition}

В дискретном времени полнота рынка равносильна единственности ЭММ (\emph{вторая фундаментальная теорема} финансовой математики).
Приведем без доказательства результат, который представляет одну из версий второй фундаментальной теоремы для непрерывного времени (доказательство см.~в книге \cite{EberleinKallsen19}, теоремы 11.52 и 11.54).

\begin{theorem}
Пусть в модели рынка существует ЭММ.
Тогда модель полна если и только если ЭММ единственна.
\end{theorem}

Большинство моделей стохастической волатильности, которые мы будем рассматривать в курсе, будут неполными.
Поясним на примере модели Хестона, откуда возникает неполнота.

Модель Хестона задается уравнениями
\begin{align*}
&dS_t = \mu S_t dt + S_t\sqrt{V_t} d W_t^{1},\\
&d V_t = \kappa(\theta - V_t)dt + \sigma \sqrt{V_t} d W_t^{2},
\end{align*}
где $W^{1}$, $W^{2}$ "--- броуновские движения с коэффициентом корреляции $\rho \in (-1,1)$.

Рассмотрим платежное обязательство с выплатой $X=f(S_T)$.
В силу марковского свойства двумерного процесса $(S_t,V_t)$ цену этого платежного обязательства можно представить в виде $U_t = U(t,S_t,V_t)$ (будем использовать букву $U$ для обозначения цены, чтобы не путать с процессом стохастической дисперсии $V_t$).

Тогда по формуле Ито (при условии, что функция $f$ достаточно <<хорошая>>)
\[
d U_t = \alpha_t dt + \beta_t d W_t^{(1)} + \gamma_t d W_t^{2},
\]
где коэффициенты $\alpha_t$, $\beta_t$, $\gamma_t$ выражаются из формулы Ито, но конкретный их вид сейчас не важен. 

С другой стороны, если рассматривать некоторую самофинансируемую стратегию $\pi$, торгующую только безрисковым и рисковым активами, то стоимость ее портфеля будет иметь вид
\[
d U_t^\pi = G_t dB_t + H_tdS_t = \alpha'_t dt + \beta'_t dW_t^{1},
\]
а слагаемого c $d W_t^{2}$ не возникает.
Следовательно, в общем случае (если $\gamma_t\neq 0$) нельзя подобрать такие процессы $G_t$ и $H_t$, чтобы получить $d V_t^\pi = d U_t$ и, значит, репликация невозможна.

Однако, репликация становится возможной, если разрешить торговать каким"=нибудь производным инструментом с ценой, зависящей от процесса $V_t$.
Портфель такой стратегии будет иметь вид $\pi_t=(G_t,H_t,F_t)$, где $F_t$ "--- количество единиц этого инструмента в портфеле, а стоимость портфеля можно представить в виде
\[
d V_t^\pi = \alpha'_t dt + \beta'_t dW_t^{(1)} + \gamma_t' dW_t^{(2)},
\]
с коэффициентами $\alpha_t'$, $\beta_t'$, $\gamma_t'$, выражающимися через $G_t,H_t,F_t$.
Следовательно, имеется три неизвестных ($G_t$, $H_t$, $F_t$) и три уравнения ($\alpha_t=\alpha_t'$, $\beta_t=\beta_t'$, $\gamma_t=\gamma_t'$), которые, в принципе, позволяют их найти.

Таким образом, изначально неполную модель можно пополнить, если разрешить торговать производными инструментами.
В качестве таких инструментов могут быть использованы, например, опционы колл и пут на базовый актив.


%stop
\section{Модели в терминах форвардных цен}

Переход от спотовой цены базового актива $S_t$ к его форвардной цене часто позволяет избежать необходимости делать модельные предположения о динамике процентной ставки и дивидендной доходности (см.~лекцию 11 во <<Введении в финансовую математику>>).

Напомним, что форвардный контракт с временем исполнения $T$, заключаемый в момент $t$, представляет из себя соглашение о покупке единицы рискового актива $n$ в момент $T$ по цене $F_t$, определяемой в момент $t$. 
Расчеты, осуществляемые в момент $T$, можно отождествить с платежным обязательством $X=S_T^n - F_t$.
Так как в момент $t$ расчетов не происходит, то $F_t$ нужно выбрать так, чтобы цена такого платежного обязательства в момент $t$ была нулевой.
Это приводит к следующему определению.

\begin{definition}
\emph{$T$-форвардной ценой} рискового актива $n$ в момент времени $t$ называется $\F_t$-измеримая случайная величина $F_t$ такая, что для платежного обязательства с выплатой $X=S_T^n-F_t$ безарбитражной ценой в момент времени $t$ является нулевая цена.
\end{definition}

Если время $T$ ясно из контекста, будем говорить просто <<форвардная цена>>. 

\begin{proposition}
Предположим, что рынок безарбитражен, а $r_t$ и $q_t^n$ неслучайны.
Тогда $T$"=форвардной ценой рискового актива $n$ является
\begin{equation}
\label{gen:forward}
F_t =  e^{\int_t^T (r_s - q_s^n) ds} S_t^n.
\end{equation}
\end{proposition}

\begin{proof}
Платежное обязательство с выплатой $X=S_T^n - F_t$ можно реплицировать стратегией $\pi$, у которой $G_s \equiv -F_tB_t/B_T$, $H_s^n = e^{-\int_t^T q_s^n ds}$ при $s\ge t$ и $G_s=H_s^n=0$ при $s\le t$, а все остальные компоненты нулевые.
Нетрудно проверить, что если $F_t$ выбрать так, что $V_t^\pi = 0$, \te\ как в формуле \eqref{gen:forward}, то стратегия $\pi$ будет самофинансируемой, а нулевая цена будет безарбитражной для $X$ в момент времени $t$ согласно предложению \ref{gen:p:replication}.
\end{proof}

Далее под форвардной ценой мы всегда будем понимать выражение \eqref{gen:forward}.%
\footnote{Как уже было сказано в замечании \ref{gen:r:replication}, могут иметься другие безарбитражные цены, которые возникают из-за того, что используемое нами условие допустимости слишком слабое.
Если его усилить, как описано в \cite{EberleinKallsen19}, гл.~11, разд.~7, то именно \eqref{gen:forward} будет <<правильной>> форвардной ценой.}
Следующий результат показывает, что процессы волатильности спотовой и форвардной цены совпадают, если $r_t$ и $q_t$ неслучайны.
Таким образом, в этом случае задачи моделирования волатильности спотовой и форвардной цены эквивалентны.
Для простоты мы рассмотрим рынок только с одним рисковым активом, но в общем случае результат аналогичен.


\begin{proposition}
\label{gen:c:forward-vol}
Пусть рынок безарбитражен, $r_t$ и $q_t$ неслучайны, а процесс цены рискового актива имеет вид \eqref{gen:risky-simple}.
Тогда $T$-форвардная цена этого актива относительно любой ЭЛММ имеет вид
\[
d F_t = \sigma_t F_t d \tilde W_t,
\] 
где $\tilde W_t$ "--- броуновское движение относительно этой ЭЛММ.
\end{proposition}

\begin{proof}
Согласно \eqref{gen:price-emm}, имеем $dS_t = (r_t-q_t) S_t dt + \sigma_t S_t d\tilde W_t$. 
Применяя формулу Ито к выражению \eqref{gen:forward}, получаем доказываемое утверждение.
\end{proof}


\summary

\begin{itemize}
\item Безарбитражность рынка равносильна существованию эквивалентной локальной мартингальной меры (первая фундаментальная теорема).

\item Безарбитражная цена интегрируемого платежного обязательства задается условным математическим ожиданием его дисконтированной выплаты относительно ЭЛММ:
\[
V_t = B_t \E^\Q \left(\frac{X}{B_T} \;\bigg|\; \F_t\right).
\]
Безарбитражные цены, в общем случае, не единственны.

\item Относительно ЭЛММ имеются два эквивалентных описания процесса цены рискового актива в терминах спотовых и форвардных цен:
\[
dS_t^n = (r_t-q_t) S_t dt + \sigma_t S_t d W_t
\quad\iff\quad
d F_t = \sigma_t F_t d W_t,
\]
где $W$ "--- броуновское движение относительно ЭЛММ.
\end{itemize}
