%!TEX root=finmath2.tex

\chapter{Модели CEV и SABR}
\label{ch:sabr}
\chaptertoc


\section{Модель CEV}
\subsection{Свойства процесса цены и связь с процессами Бесселя}

Как и при рассмотрении модели Хестона, будем считать, что исходная вероятностная мера уже является мартингальной, безрисковая процентная ставка равна нулю, а рисковый актив не выплачивает дивиденды (или моделируется процесс форвардной цены базового актива).

В модели CEV цена рискового актива задается стохастическим дифференциальным уравнением
\begin{equation}
\label{sabr:cev}  
dS_t = \alpha S_t^{\beta} dW_t, 
\end{equation}
где $\alpha>0$, $\beta \ge 0$ "--- параметры модели. Начальное значение $S_0$ положительно и неслучайно. Заметим, что если $\beta=1$, то получается модель \bs, а если $\beta=0$, то модель Башелье. 

Корректность уравнения \eqref{sabr:cev} уже обсуждалась в лекции \ref{ch:sde}: оно имеет неотрицательное сильное решение, причем решение единственно, если $\beta \ge 1/2$. При $\beta\ge 1$ решение строго положительно, а при $\beta < 1$ достигает нуля с положительной вероятностью.

При $\beta< 1/2$ решение единственно до момента достижения нуля, но глобальной единственности нет, так как после достижения нуля возможны различные продолжения.
Далее мы всегда будем считать, что выбирается такое решение, что после достижения нуля процесс остается в нуле, а иначе в модели бы возник арбитраж (можно купить базовый актив за нулевую цену и продать за положительную). 

Напомним также, что при $\beta\le 1$ процесс $S_t$ является мартингалом, а при $\beta>1$ он является лишь локальным мартингалом.

Далее мы ограничимся рассмотрением только случая $\beta\in(0,1)$.

\begin{remark}
Одна из мотиваций введения модели CEV состоит в том, что она позволяет учесть так называемый \emph{эффект рычага} (\emph{leverage effect}), который означает, что волатильность растет при падении цены базового актива (в предположении $\beta <1$). 
А именно, если записать уравнение для цены в виде $dS_t = \sigma(S_t)S_t d W_t$, где $\sigma(s) = \alpha s^{\beta-1}$ "--- локальная волатильность, то видно, что локальная волатильность как раз и обладает таким поведением.
\end{remark}

\begin{remark}
Скажем подробнее о модели CEV с ненулевой безрисковой ставкой. 
Если считать, что цена $S_t$ описывается уравнением
\[
dS_t = rS_t dt + \alpha S_t^\beta dW_t,
\]
то $T$-форвардная цена $F_t = e^{r(T-t)} S_t$ будет удовлетворять уравнению
\[
dF_t = \alpha e^{r(1-\beta)(T-t)} F_t^\beta dW_t,
\]
в чем можно убедиться с помощью формулы Ито.
Возникающий экспоненциальный множитель в волатильности делает неудобным применение получаемых далее результатов к процессу $F_t$. 
Избавиться от него можно, если считать, что уравнению \eqref{sabr:cev} удовлетворяет сама форвардная цена $F_t$ (и тогда уже уравнение для $S_t$ будет содержать аналогичный множитель в волатильности).
\end{remark}

Изучение процесса цены в модели CEV можно свести к хорошо развитой теории \emph{процессов Бесселя}.
Сначала дадим определение процессов Бесселя.

\begin{definition}
Пусть $W=(W_t)_{t\ge 0}$ "--- броуновское движение в $\R^n$, $n\ge 2$, и $x\in \R^n$. Тогда процесс 
\[
X_t = \|W_t+x\|, \qquad t\ge 0,
\]
называется \emph{процессом Бесселя} размерности $n$ с начальным условием $x_0=\|x\|$.
Здесь $\|\cdot\|$ "--- евклидова норма.
\end{definition}

Таким образом, процесс Бесселя равен расстоянию броуновского движения в $\R^n$ от начала координат, в предположении, что броуновское движение начинается из точки $x$.
Нетрудно показать, что распределение процесса Бесселя не зависит от того, каково именно начальное значение $x$, а зависит только от его нормы%
\footnote{
Это можно увидеть из того, что характеристическая функция величины $(W_t^1 + x^1)^2 + \ldots + (W_t^n + x^n)^2$ зависит от $x$ только посредством зависимости от $(x^1)^2 + \ldots + (x^n)^2$.}.
Квадрат этого процесса, $Y_t=X_t^2$, имеет \emph{нецентральное распределение хи-квадрат} с $n$ степенями свободы и параметром нецентральности $x_0^2$. 

Название \emph{процесс Бесселя} объясняется тем, что плотность нецентрального распределения хи-квадрат выражается через модифицированные функции Бесселя первого рода.

Используя формулу Ито и теорему Леви для броуновского движения, можно показать, что $Y_t$ удовлетворяет стохастическому дифференциальному уравнению
\[
dY_t = ndt + 2\sqrt{Y_t}d\tilde W_t,
\]
где $\tilde W_t$ "--- некоторое другое броуновское движение%
\footnote{
По формуле Ито $dY_t = ndt + 2\sqrt{Y_t} d\tilde W_t$, где $\tilde W_t = \sum_i \int_0^t W_s^i/\|W_s\| \I(W_s\neq 0) ds$ является непрерывным локальным мартингалом.
Показывается, что квадратическая характеристика $\qc{\tilde W}_t = t$. Следовательно, $\tilde W_t$ "--- броуновское движение.
}.
Это представление подсказывает, как можно определить процесс Бесселя произвольной размерности.

\begin{definition}
\emph{Квадратом процесса Бесселя} размерности $\delta \in \R$ c начальным условием $y_0\ge0$ называется процесс, являющийся решением стохастического дифференциального уравнения 
\begin{equation}
\label{sabr:bessel}
dY_t = \delta dt + 2\sqrt{|Y_t|} d W_t, \qquad Y_0=y_0,
\end{equation}
где $W_t$ "--- одномерное броуновское движение.

Тогда \emph{процессом Бесселя} размерности $\delta \in \R$ (или процессом Бесселя \emph{с индексом $\nu=\delta/2-1$}) c начальным условием $x_0=\sqrt{y_0}$ называется процесс $X_t = \sqrt Y_t$, определенный до момента времени, когда $Y_t$ становится отрицательным.
\end{definition}

Известны следующие результаты о поведении квадрата процесса Бесселя (см.~\cite{ChernyEngelbert}, прил.~A.3; эти результаты можно получить из теорем Энгельберта"--~Шмидта, Ямады"--~Ватанабе и критерия Феллера):
\begin{itemize}
\item уравнение \eqref{sabr:bessel} имеет единственное сильное решение;
\item при $\delta \ge 2$ решение строго положительно для всех $t>0$;
\item при $0<\delta<2$ решение всегда неотрицательно, достигает нуля с вероятностью 1, при этом мера времени, проводимого в нуле, равна нулю (\te\ $\int_0^\infty \I(Y_t=0) dt = 0$ \as);
\item при $\delta =0$ решение достигает нуля с вероятностью 1 и затем остается в нуле;
\item при $\delta <0$ решение достигает нуля с вероятностью 1, а затем ведет себя как минус квадрат процесса Бесселя размерности $|\delta|$.
\end{itemize}

\begin{proposition}
Пусть $S_t$ "--- процесс цены рискового актива в модели CEV.
Для $\delta = (1-2\beta)/(1-\beta)$ определим процесс
\[
X_t = \frac{S_t^{1-\beta}}{\alpha(1-\beta)}.
\]
Тогда $Y_t=X_t^2$ удовлетворяет уравнению
\[
d Y_t = \delta \I(Y_t>0) dt + 2\sqrt{Y_t}d \tilde W_t.
\]
\end{proposition}
Таким образом, процесс $X_t$ является процессом Бесселя до момента достижения нуля, а индикатор в коэффициенте сноса обеспечивает то, что, попав в ноль, процесс там и остается.
Доказательство этого предложения сводится к применению формулы Ито.
%TODO написать, как аккуратно применить формулу Ито из-за недифференцируемости в нуле

Заметим, что интервалу значений параметра $\beta\in(0,1/2)$ соответствует $\delta \in (0,1)$, a интервалу $\beta\in[1/2,1)$ соответствует $\delta\in(-\infty,0]$.

Детальное изложение теории процессов Бесселя можно найти в~книге~\cite{RevuzYor}, гл.~XI, \S\,1 для случая $\delta\ge 0$ и обзорной статье~\cite{GoingYor03} для $\delta\in\R$.
Из этих результатов получаются свойства процесса CEV. 
Приведем выражение для его переходной вероятности и вероятности достижения нуля (подробнее см.~\cite{Schroder89}).

\begin{proposition}
Переходная вероятность процесса $S_t$ имеет вид (при $s_0>0$)
\[
\P(S_t \le x \mid S_0=s_0) = \P(S_t=0) + \int_0^x p(s_0,s,t) ds,
\]
с плотностью
\[
p(s_0,s,t) =
\frac{s^{\frac 12-2\beta}s_0^{\frac12}}{\alpha^2(1-\beta)t}
\exp\biggl(-\frac{s^{2(1-\beta)} 
  + s_0^{2(1-\beta)}}{2\alpha^2(1-\beta)^2t}\biggr)
I_{\frac1{2(1-\beta)}}\biggl(\frac{s^{1-\beta}s_0^{1-\beta}}{\alpha^2(1-\beta)^2t}\biggr)
,
\]
где $I_\nu(x)= \sum_{n\ge0}\frac{(x/2)^{2n+\nu}}{k!\Gamma(n+\nu+1)}$ "--- модифицированная функция Бесселя первого рода порядка $\nu$.
Вероятность достижения нуля до времени $t$ равна
\[
\P(S_t = 0 \mid S_0=s_0) = 
G\biggl(\nu,\ 
  \frac{s_0^{2(1-\beta)}}{2\alpha^2(1-\beta)^2t}
\biggr),
\]
где $G(\nu,x) = \Gamma(\nu)^{-1}\int_x^\infty u^{\nu-1}e^{-u}du$, и $\Gamma(\nu)=\int_0^\infty u^{\nu-1}e^{-u}du$ -- гамма-функция.
\end{proposition}


\subsection{Цены опционов и подразумеваемая волатильность}
Интегрируя переходную плотность процесса $S_t$, можно получить явную формулу для цены европейских опционов. Приведем формулу для опционов колл; цену опционов пут можно найти аналогично или из паритета цен колл--пут.

\begin{proposition}
Цена опциона колл $C(T,K)=\E(S_T-K)^+$ в модели CEV равна
\[
C(T,K) = S_0 (1-F(a; 2+(1-\beta)^{-1}, b)) -  K F(b; (1-\beta)^{-1},a),
\]
где
\[
a=\frac{K^{2(1-\beta)}}{\alpha^2(1-\beta)^2T}, \qquad
b=\frac{S_0^{2(1-\beta)}}{\alpha^2(1-\beta)^2T},
\]
и $F(x; k, \lambda)$ обозначает функцию распределения нецентрального распределения хи-квадрат с $k$ степенями свободы и параметром нецентральности $\lambda$.
\end{proposition}

\begin{remark}
Плотность нецентрального распределения хи-квадрат с $k>0$ степенями свободы и параметром нецентральности $\lambda>0$ имеет вид
\[
f(x;k,\lambda) = \frac12 e^{-\frac{x+\lambda}{2}}\Bigl(\frac x\lambda\Bigr)^{\frac k4-\frac12} I_{\frac k2-1}(\sqrt{\lambda x}), \qquad x>0.
\]
\end{remark}

На рис.~\ref{sabr:f:cev-smile} показаны улыбки волатильности в модели CEV.
В частности, видно, что модель позволяет воспроизвести асимметрию улыбок, выражающуюся в более крутом левом хвосте, при этом, чем меньше значение $\beta$ тем более ассиметрична улыбка. 

\begin{figure}[ht]
\centering
\includegraphics{pic/cev.png}
\caption{Улыбки волатильности в модели CEV при $\alpha=0.3$ и различных значениях $\beta$. Время до исполнения $T=1$, начальная цена $S_0=1$.}
\label{sabr:f:cev-smile}
\end{figure}


\section{Модель SABR}
\subsection{Уравнения, задающие модель}

В модели SABR процесс волатильности задается геометрическим броуновским движением:
\begin{align}
\label{sabr:sabr-price}
&d S_t = \alpha_t S_t^\beta d W_t^{(1)},\\
\label{sabr:sabr-vol}
&d\alpha_t = \nu\alpha_t d W_t^{(2)},\\
\label{sabr:sabr-corr}
&dW_t^{(1)} dW_t^{(2)} = \rho dt,
\end{align}
где $\nu>0,\beta\in[0,1],\rho\in (-1,1)$ "--- параметры модели, а начальные значения $S_0$ и $\alpha_0$ положительны. Если $\nu=0$, то $\alpha_t=\alpha_0$ для всех $t\ge0$, и получается модель CEV.

Далее будем считать, что если процесс $S_t$ достигает нуля, то он там и остается. 
Покажем, что с таким предположением уравнения \eqref{sabr:sabr-price}--\eqref{sabr:sabr-vol} имеют единственное сильное решение.

С процессом $\alpha_t$ трудностей не возникает, так как это обычное геометрическое броуновское движение. Однако к процессу $S_t$ не применимы общие результаты о существовании и единственности решений, которые обсуждались в лекции \ref{ch:sde}, поэтому мы докажем сильное существование и единственность непосредственно.
Для этого нам потребуется следующая теорема о представлении непрерывных локальных мартингалов через \emph{замену времени} для броуновского движения, имеющая самостоятельный интерес.

\begin{proposition}[\emph{теорема Дамбиса"--~Дубинса"--~Шварца}]
\label{sabr:t:dds}
Пусть непрерывный локальный мартингал $M=(M_t)_{t\ge0}$, $M_0=0$, задан на фильтрованном вероятностном пространстве $(\Omega,\F,(\F_t)_{t\ge0},\P)$, причем $M_0 = 0$.
Тогда на некотором расширении исходного вероятностного пространства найдется броуновское движение $\tilde W_t$ такое, что $M_t = \tilde W_{\qc M_t}$.
\end{proposition}

\begin{remark}
Процесс $\qc M_t$ "--- это квадратическая характеристика мартингала $M$ (\te\ такой неубывающий непрерывный процесс, что $M_t^2 - \qc M_t$ является локальным мартингалом; ее существование следует из \emph{разложения Дуба"--~Мейера}).
Если $M_t$ является интегралом Ито вида $M_t = \int_0^t \sigma_s d W_s$, то $\qc M_t = \int_0^t \sigma_s^2 ds$.

Подробнее о том, как строится указанное расширение вероятностного пространства, см., например,~\cite{BarndorffNielsenShiryaev15}, гл.~1, \S\,4.
Отметим, что в расширении нет необходимости, если $\qc M_t\to\infty$ \as\ при $t\to\infty$.
В таком случае достаточно определить семейство моментов остановки $\tau_t = \inf\{s\ge 0: \qc M_s \ge t\}$, и тогда процесс $\tilde W_t = M_{\tau_t}$, $t\ge0$, является броуновским движением относительно фильтрации $\tilde\F_t = \F_{\tau_t}$, $t\ge 0$, причем $M_t = \tilde W_{\qc M_t}$ \as
\end{remark}

% \begin{remark}
% У теоремы Дамбиса"--~Дубинса"--~Шварца имеется обобщение "--- теорема Монро (I.~Monroe, 1978), которая утверждает, что непрерывный справа и имеющий пределы слева процесс $X$ является семимартингалом тогда и только тогда, когда найдется броуновское движение $\tilde W$ и неубывающее и непрерывное справа (\as) семейство моментов остановки $\tau_t$, $t\ge 0$, такое, что процессы $X_t$ и $W_{\tau t}$ одинаково распределены.
% \end{remark}

\begin{theorem}
Уравнения \eqref{sabr:sabr-price}--\eqref{sabr:sabr-vol} имеют единственное сильное решение $(S_t,\alpha_t)$ для любых начальных условий $s_0>0$ и $\alpha_0 > 0$.
\end{theorem}

\begin{proof}
Пусть $W_t = (W_t^{(1)}, W_t^{(2)})$ "--- двумерное броуновское движение с корреляцией $\rho$ между компонентами. Процесс $\alpha_t$ строится по $W_t^{(2)}$ явно как геометрическое броуновское движение, причем такой процесс единственен в сильном смысле.
Далее определим локальный мартингал
\[
M_t = \int_0^t \alpha_s d W_s^{(1)}.
\]
Его квадратическая характеристика $\qc M_t = \int_0^t \alpha_s^2 ds$ и тогда по теореме Дамбиса"--~Дубинса"--~Шварца найдется (одномерное) броуновское движение $\tilde W_t$ такое, что $M_t = \tilde W_{\qc M_t}$.
Построим по нему сильное решение уравнения для процесса CEV 
\begin{equation}
\label{sabr:aux-cev}
d\tilde S_t = \tilde S_u^\beta d \tilde W_u, \qquad \tilde S_0 = s_0,
\end{equation}
где в случае $\beta<1/2$ считается, что процесс остается в нуле после его достижения.
Положим $S_t = \tilde S_{\qc M_t}$. 
Тогда%
\footnote{Во втором равенстве нужно воспользоваться формулой замены времени в стохастическом интеграле, см.~\cite{KaratzasShreve91}, гл.~3, предл.~4.8.}
\[
S_t = s_0 + \int_0^{\qc M_t} \tilde S_u^\beta d \tilde W_u = 
s_0 + \int_0^{t} \tilde S_{\qc M_t}^\beta d \tilde W_{\qc M_t} =
s_0 + \int_0^t S_t^\beta d M_t =
s_0 + \int_0^t \alpha_t S_t^\beta d W_t^{(1)},
\]
и, следовательно, $S_t$ является сильным решением уравнения \eqref{sabr:sabr-price}.

Сильная единственность следует из того, что предыдущие рассуждения можно обратить и тогда, если бы существовало два решения $S_t$ и $S_t'$, по ним можно было бы построить два решения уравнения \eqref{sabr:aux-cev}, что невозможно в предположении, что после достижения нуля выбирается нулевое продолжение.
\end{proof}


\subsection{Приближенная формула для подразумеваемой волатильности}
Самым известным результатом, связанным с моделью SABR и делающей ее популярной, является \emph{приближенная формула} для подразумеваемой волатильности в модели, впервые полученная в работе П.~Хэгана и соавторов \cite{Hagan+02} (так называемая \emph{формула Хэгана}). 

Пусть $\hat\sigma(T,K)$ обозначает подразумеваемую волатильность в модели SABR для времени исполнения $T$ и страйка $K$ (\te\ такая волатильность в модели Блэка, что цена опциона с заданными $T,K$ в ней и в модели SABR совпадают).

\begin{theorem}
Зафиксируем $K>0$ и положим $x=\ln(S_0/K)$. Тогда при $T\to0$ верно представление
\[
\hat\sigma(T,K) = I_0(x)(1+ I_1(x)T) + O(T^2),
\]
где
\begin{align*}
&I_0(x) = \nu x \Big/ \ln\left(\frac{\sqrt{1-2\rho z + z^2} + z-\rho}{1-\rho}\right),\\
&I_1(x) = \frac{(1-\beta)^2}{24}\frac{\alpha_0^2}{(S_0K)^{1-\beta}} + \frac{\rho\nu\beta\alpha_0}{(S_0K)^{(1-\beta)/2}} + \frac{2-3\rho^2}{24}\nu^2,
\end{align*}
с обозначением
\[
z = \frac{\nu}{\alpha_0}\frac{S_0^{1-\beta} - K^{1-\beta}}{1-\beta}.
\]
Для $x=0$, $\nu=0$ или $\beta=1$ возникающие в формуле для $I_0(x)$ неопределенности раскрываются так:
\begin{align*}
&I_0(0) = \alpha_0 K^{\beta-1},\\
&I_0(x) = \frac{x\alpha_0(1-\beta)}{S_0^{1-\beta} - K^{1-\beta}}\ \text{при}\ \nu=0,\\
&z = \frac{\nu}{x}{\alpha_0}\ \text{при}\ \beta=1.
\end{align*}
\end{theorem}

\begin{remark}
Исходная формула, полученная в работе \cite{Hagan+02} оказалась не вполне верной в выражении для $I_0(x)$ для $\beta<1$ и была потом подправлена в работах \cite{Berestycki+04,Labordere05} (короткое резюме результатов и обсуждение см.~в~\cite{Obloj07}).
Доказательства являются крайне трудными во всех трех работах, мы их здесь не приводим.
\end{remark}
