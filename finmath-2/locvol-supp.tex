%!TEX root=finmath2.tex
\stopchaptertoc

\part*{Дополнения к лекциям}
\titleformat{\chapter}[display]{\bfseries\Large}{\underline{Дополнение~\thechapter}}{0.5em}{}
\titlecontents{chapter}[0em]{}{\textbf{Дополнение\ \thecontentslabel.}\hspace{2mm}}{}{\dotfill\contentspage}
\setcounter{chapter}{0}
\renewcommand{\theHchapter}{A\arabic{chapter}}%


\chapter{К модели локальной волатильности}
\label{ch:lv-s}

Здесь собраны результаты о модели локальной волатильности, которые не вошли в лекцию \ref{ch:locvol}, но представляют определенный интерес. 

\section{Формула Дюпира через подразумеваемую волатильность}

Формулу Дюпира для локальной волатильности можно переписать так, чтобы вместо поверхности цен опционов она содержала поверхность \emph{полной подразумеваемой дисперсии}.
Полная подразумеваемая дисперсия опциона (total implied variance) равна величине $\hat w = \hat\sigma^2 T$, где $\hat\sigma$ "--- его подразумеваемая волатильность.
Представление локальной волатильности через $\hat w$ может быть удобно, когда дана поверхность подразумеваемой волатильности, а не непосредственно цены опционов.

Сделаем замену координат и вместо страйка $K$ будем использовать \emph{лог"=форвардную денежность} $y=\ln (K/F_0^T)$, где $F_0^T = B_T S_0$ обозначает $T$"=форвардную цену в момент времени $t=0$.
Величины $S_0$ и функция $B_t$ считаются фиксированными; безрисковая процентная ставка неслучайна.

Нетрудно видеть, что имеется взаимно-однозначное соответствие между переменными $(T,K,\sigma)$ и $(T,y,w)$, причем подразумеваемая волатильность $\hat \sigma(T,K)$ и полная подразумеваемая дисперсия $\hat w(T,y)$ связаны по формулам 
\[
\hat w(T,y) = \hat\sigma^2(T, e^yF_0^T)T, \qquad \hat\sigma(T,K) = \sqrt{\hat w(T,\ln(K/F_0^T))/T}.
\]

\begin{proposition}
Пусть для заданной поверхности цен опционов $\hat C(T,K)$ выполнены условия теоремы \ref{lv:t:dupire} из лекции \ref{ch:locvol}.
Тогда локальную волатильность можно представить в виде $\sigma(t,s) = \sqrt{v(t, \ln(s/F_0^t))}$, где функция $v(t,y)$ связана с $\hat w(t,y)$ по формуле
\begin{equation}
\label{lv:v}
v(t,y) = \frac{\hat w'_T(t,y)}{1 - \frac{y}{\hat w(t,y)} \hat w'_y(t,y) - \frac14\Bigl(\frac14 + \frac1{\hat w(t,y)} - \frac{y^2}{\hat w^2(t,y)}\Bigr)(\hat w'_y(t,y))^2 + \frac12 \hat w''_{yy}(t,y)}.
\end{equation}
\end{proposition}

\begin{proof}
В переменных $(y,w)$ формула Блэка для цены опциона колл принимает вид
\begin{align*}
&c(y, w) = S_0(\Phi(d_1) - e^y \Phi(d_2)),\\
&d_1 = -\frac{y}{\sqrt w}  + \frac{\sqrt{w}}{2}, \qquad d_2 = d_1 - \sqrt w.
\end{align*}
Путем непосредственного дифференцирования проверяются следующие соотношения между производными:
\begin{align*}
%&c'_T = c\,r_T,& & & &\\
&c'_y = c - S_0\Phi(d_1),& &c'_w = \frac{S_0}{2\sqrt w}\phi(d_1),& &\\
&c''_{yy} = c'_y + 2c'_w,& &c''_{wy}= \Bigl(\frac12 - \frac yw\Bigr)c'_w,& 
&c''_{ww} = -\Bigl(\frac18 + \frac1{2w} - \frac{y^2}{2w^2}\Bigr) c'_w,\\
&y'_T = -r_T,& &y'_K = \frac1K,& &y''_{KK} = -\frac1{K^2}.
\end{align*}
Далее в формулу Дюпира
\begin{equation}
\label{lv:dupire-KT}  
\sigma^2(T,K) = 2\frac{\prt{}T C(T,K) + r_TK\prt{}KC(T,K)}{K^2\prtt{}K C(T,K)}
\end{equation}
подставим $C(T,K) = c(y(T,K), \hat w(T, y(T,K)))$ и вычислим производные:
\begin{align*}
\prt CT &=  c'_y y'_T + c'_w(\hat w'_T + \hat w'_y y'_T) = -r_T (c'_y + c'_w \hat w'_y) + c'_w \hat w'_T,\\[0.3em]
\prt CK &= (c'_y + c'_w\hat w'_y) y'_K = (c'_y + c'_w\hat w'_y) \frac1K,\\[0.3em]
\prtt CK &= c''_{yy}(y'_K)^2 + 2 c''_{yw} y'_K \prt wK + c'_y y''_{KK} + c''_{ww}y'_K\prt wK + c'_w \Bigl(\prt wK\Bigr)^2 + c'_w \prtt wK\\[0.3em]
&= \frac2{K^2} c'_w\biggl(1 - \frac yw w'_y -\frac14\Bigl(\frac14 + \frac1w - \frac{y^2}{w^2}\Bigr)(w'_y)^2 + \frac12 w''_{yy}\biggr).
\end{align*}
Из первых двух равенств находим, что
\[
\prt{}T C(T,K) + r_TK\prt{}KC(T,K) = c'_w w'_T.
\]
Подставляя полученные выражения в \eqref{lv:dupire-KT}, получаем формулу \eqref{lv:v}.
\end{proof}




\section{Локальная волатильность форвардной цены}
Покажем, что в модели локальной волатильности форвардная цена тоже допускает представление в виде уравнение, где коэффициент волатильности зависит только от времени и форвардной цены. 

\begin{proposition}
Пусть цена рискового актива задается моделью локальной волатильности $dS_t = r_tS_t dt + \sigma(t,S_t) S_t d W_t$, где процентная ставка $r_t$ неслучайна. Тогда $T$-форвардная цена $F_t^T = S_tB_T/B_t$, где $t\in[0,T]$, удовлетворяет уравнению
\[
d F_t^T = \sigma_f(t,F_t^T) F_t^T d W_t, \qquad F_0^T = S_0B_T,
\]
с функцией
\[
\sigma_f(t,f) = \sigma(t,f B_t/B_T),
\]
а процесс приращения логарифма форвардной цены $Y_t^T = \ln F_t^T - \ln F_0^T$ удовлетворяет уравнению
\[
d Y_t^T = -\frac12 v(t, Y_t^T) dt + \sqrt{v(t, Y_t^T)} d W_t, \qquad X_0^T=0,
\]
где $v(t,y)$ "--- функция из \eqref{lv:v}.
\end{proposition}

\begin{proof}
Как было показано в лекции \ref{ch:general} (см.~следствие \ref{gen:c:forward-vol}), коэффициент волатильности у процессов $F_t$ и $S_t$ одинаковый (в том смысле, что это один и тот же случайный процесс), а коэффициент сноса у $F_t$ нулевой. Таким образом,
\[
d F_t = \sigma(t,S_t)F_t d W_t = \sigma(t, F_tB_t/B_T) F_tdW_t,
\]
что и требовалось доказать.

Чтобы получить уравнение для $Y_t^T$, заметим, что из соотношения $\sigma^2(t,s) = v(t, \ln(s/F_0^t))$ следует, что
\[
\sigma^2_f(t,F_t^T) = \sigma^2(t,F_t^TB_t/B_T) = v(t, \ln(F_t^TB_t/(B_TF_0^t))) = v(t,Y_t^T).
\]
Применяя далее формулу Ито к $Y_t^T$, получаем требуемое уравнение.
\end{proof}


%%%%%%%%%%%%%%%%%%%%%%%%%%%%%%%%%%%%%%%%%%%%%%%%%%%%%%%%%%%%%%%%%%%%%
%%% Этот фрагмент нужно перенести в следующий курс, в материал про LSV
%%%%%%%%%%%%%%%%%%%%%%%%%%%%%%%%%%%%%%%%%%%%%%%%%%%%%%%%%%%%%%%%%%%%%

% \section{Локальная волатильность из стохастической волатильности}

% Предположим, что в некоторой модели рынка цена рискового актива задается уравнением
% \begin{equation}
% \label{lv:sv}
% d S_t = r_tS_t dt + \sigma_t S_t dW_t,
% \end{equation}
% где $r_t$ "--- детерминированная процентная ставка, а $\sigma_t$ "--- случайный процесс (\emph{стохастическая волатильность}) такие, что соответствующие интегралы корректно определены. 

% Следующая теорема показывает, как получить функцию локальной волатильности для такой модели: грубо говоря, нужно взять
% \begin{equation}
% \label{lv:lv-from-sv}
% \sigma(t,s) = \sqrt{\E(\sigma_t^2 \mid S_t=s)}.
% \end{equation}
% (точное утверждение см.~в замечании \ref{lv:r:lv-formula} ниже).

% Далее сформулируем общую теорему, из которой эта формула будет следовать при выполнении условия на интегрируемость коэффициентов.

% \begin{theorem}[теорема Дьёндя]
% \label{lv:t:gyoungy}
% Пусть на некотором фильтрованном вероятностном пространстве $(\Omega,\F,\FF,\P)$ задан $n$-мерный процесс Ито $X_t$ вида
% \[
% dX_t^i = b_t^i dt + \sum_{j=1}^d\sigma_t^{ij} dW_t^j,\qquad X_0 = x_0\in\R,
% \]
% где $b_t$ и $\sigma_t$ является согласованными измеримыми процессами со значениями в $\R^n$ и $\R^{n\times d}$, удовлетворяющими условиям $\E\int_0^t \|b_s\| ds <\infty$ и $\E\int_0^t \|\sigma_s\sigma'_s\| ds < \infty$ для всех $t\ge 0$ ($\sigma'_s$ "--- транспонированная матрица). 

% Тогда найдутся функции $b(t,x)\colon \R_+\times \R^n\to\R^n$ и $\sigma(t,x)\colon\R_+\times\R^n\to \R^{n\times d}$ такие, что для почти всех (по мере Лебега) $t \ge0$ выполнены равенства
% \begin{equation}
% \label{lv:gyongy}
% b(t,X_t) = \E(b_t\mid X_t), \quad \sigma(t,X_t)\sigma'(t,X_t) = \E(\sigma_t\sigma'_t\mid X_t) \quad \as,
% \end{equation}
% и стохастическое дифференциальное уравнение
% \[
% d \hat X_t = b(t,\hat X_t) dt + \sigma(t,\hat X_t) d W_t, \qquad \hat X_0=x_0,
% \]
% имеет слабое решение, у которого одномерные распределения совпадают с распределениями процесса $X_t$, \te\ $\hat X_t \stackrel{d}{=} X_t$ для всех $t\ge 0$.
% \end{theorem}

% \begin{remark}
% \label{lv:r:lv-formula}
% Если теперь применить эту теорему к модели \eqref{lv:sv} и найти процесс $\hat S_t$ с такими же одномерными распределениями, как у $S_t$, то эти два процесса будут давать одинаковые цены опционов колл, так как они определяются только одномерными распределениями.
% Из равенства \eqref{lv:gyongy} тогда заключаем, что $\sigma^2(t,S_t) = \E(\sigma_t^2\mid S_t)$ \as\ для п.\,в.\ $t\ge 0$.
% Отсюда следует, что формула \eqref{lv:lv-from-sv} верна $\mu_{S_t}$-\as\ для п.\,в.\ $t\ge 0$, где $\mu_{S_t}$ "--- вероятностная мера на $\R_+$, являющаяся распределением величины $S_t$.  
% \end{remark}

% \begin{remark}
% И.~Дьёндь (I.~Gy\"ongy) доказал эту теорему в работе \cite{Gyongy86} для случая, когда процессы $b_t$ и $\sigma_t$ ограничены, причем матричный процесс $\sigma_t\sigma'_t$ равномерно положительно определен.
% Приведенное выше обобщение, не использующее дополнительные условия, взято из работы \cite{BrunickShreve13}.
% \end{remark}

% Доказательство теоремы довольно трудное.
% Главную сложность в нем представляет доказательство \emph{существования} процесса $\hat X_t$. 
% Далее мы сформулируем и докажем упрощенную версию этой теоремы, применимую в контексте модели локальной волатильности, в которой уже предполагается, что нужный процесс существует, и находится формула для локальной волатильности.

% \newcounter{theoremold}
% \setcounter{theoremold}{\value{theorem}}
% \let\oldthetheorem\thetheorem
% \renewcommand\thetheorem{\ref{lv:t:gyoungy}\,$'$}

% \begin{theorem}
% \label{lv:t:lv-from-sv}
% Пусть процессы $S_t$ и $\hat S_t$ имеют вид 
% \begin{align*}
% &d S_t = r_tS_t dt + \sigma_t S_t dW_t,\\
% &d \hat S_t = r_t\hat S_t dt + \sigma(t,\hat S_t) \hat S_t dW_t
% \end{align*}
% с детерминированной процентной ставкой $r_t$, одинаковым начальным условием $S_0=\hat S_0 > 0$ и одним и тем же броуновским движением $W$, причем дисконтированные процессы $S_t/B_t$ и $\hat S_t/B_t$ являются (настоящими) мартингалами.

% Предположим, что $\E(S_T-K)^+ = \E (\hat S_T-K)^+ < \infty$ для всех $T\ge 0$ и $K>0$ (и, следовательно, у обоих процессов совпадают цены опционов колл).
% Тогда для всех $t\ge 0$, за исключением, быть может, множества нулевой меры Лебега, выполнено равенство $\sigma^2(t,S_t) = \E(\sigma_t^2\mid S_t)$ \as
% \end{theorem}

% % Чтобы сделать номер теоремы с штрихом
% \let\thetheorem\oldthetheorem
% \setcounter{theorem}{\value{theoremold}}

% \begin{remark}
% Нетрудно показать, что совпадение цен опционов колл равносильно совпадению одномерных распределений двух положительных процессов с конечными математическими ожиданиями.
% \end{remark}

% \begin{proof}[Доказательство теоремы \ref{lv:t:lv-from-sv}]
% Достаточно рассмотреть случай $r_t=0$, иначе можно перейти к дисконтированным ценам, так же, как это было сделано при доказательстве формулы Дюпира в лекции \ref{ch:lv1}. 

% Пусть $C(T,K) = \E(S_T-K)^+$.
% Из следствия \ref{lv:с:expectation} в лекции \ref{ch:lv1} имеем $C(T,K) = (S_0-K)^+ + \frac 12 \E L_T^K(S)$. 
% Тогда для любой гладкой функции $h(s)$ с компактным носителем
% \begin{multline*}
% \int_\R h(K) C(T,K) dK = \int_\R h(K) (S_0-K)^+ dK + \frac12 \E \int_0^T h(S_t) \sigma_t^2 S_t^2 dt \\
% = \int_\R h(K) (S_0-K)^+ dK  + \frac12  \int_0^T \E(h(S_t) \E(\sigma^2_t\mid S_t) S_t^2)dt,
% \end{multline*}
% где в первом равенстве воспользовались формулой времени пребывания как в доказательстве формулы Дюпира, далее поменяли местами математическое ожидание и интеграл по теореме Фубини и воспользовались телескопическим свойством условного ожидания.
% С помощью аналогичных рассуждений получаем, что 
% \[
% \int_\R h(K) C(T,K) dK = \int_\R h(K) (S_0-K)^+ dK + \frac12 \int_0^T \E(h(\hat S_t)  \sigma^2(t,\hat S_t) \hat S_t^2) dt.
% \]
% Так как эти равенства выполняются для всех $T$, то подынтегральные функции должны совпадать за исключением, быть может, множества нулевой меры по Лебегу, \te\ $\E(h(S_t) \E(\sigma^2_t\mid S_t) S_t^2) = \E (h(\hat S_t)\sigma^2(t,\hat S_t) \hat S_t^2)$ для п.\,в.\ всех $t\ge 0$. 
% Представляя математические ожидания как интегралы по распределению $\mu_{S_t}$ величины $S_t$ (совпадающему с распределением $\hat S_t$), приходим к равенству
% \[
% \int_\R h(s) \E(\sigma_t^2 \mid S_t = s) s^2 \mu_{S_t}(ds) = \int_\R h(s) \sigma^2(t,s) s^2 \mu_{S_t}(ds).
% \]
% Из произвольности функции $h$ заключаем, что подынтегральные выражения должны быть равны по мере $\mu_{S_t}$, что и доказывает теорему.
% \end{proof}
%%%%%%%%%%%%%%%%%%%%%%%%%%%%%%%%%%%%%%%%%%%%%%%%%%%%%%%%%%%%%%%%%%%%%

\section{Аппроксимация подразумеваемой волатильности}
\label{lv-s:s:hagan}
\subsection{Формулировка результата}

В этом разделе мы приведем приближенную аналитическую формулу Хэгана"--~Вудвард \cite{HaganWoodward99} для подразумеваемой волатильности в модели вида
\begin{equation}
\label{lv:homogenous}
dS_t = a(t)A(S_t) dW_t.
\end{equation}
Будем считать, что  функции $a(t)$ и $A(s)$ принимают неотрицательные значения.
Заметим, что множитель $S_t$ для удобства включен в коэффициент $A(S_t)$.


\begin{theorem}[П.~Хэган, Д.~Вудвард]
\label{lv:t:hagan-woodward}
Пусть $\hat\sigma(T,K)$ "--- подразумеваемая волатильность в модели \eqref{lv:homogenous}.
Положим 
\[
\tilde s = (S_0 + K)/2,\qquad
\alpha = \sqrt{\frac{1}{T}\int_0^T a^2(t)dt}.
\]

Тогда верна следующая приближенная формула:
\begin{multline}
\label{lv:hw}
\hat\sigma(T,K) = \frac{\alpha A(\tilde s)}{\tilde s} 
\biggl\{1 
+ \frac1{24} 
  \biggl(
    \frac{A''(\tilde s)}{A(\tilde s)} - 2\biggl(\frac{A'(\tilde s)}{A(\tilde s)}\biggr)^2 + \frac{2}{\tilde s^2}
  \biggr) (S_0-K)^2\\
+ \frac1{24}
  \biggl(
    2\frac{A''(\tilde s)}{A(\tilde s)} - \biggl(\frac{A'(\tilde s)}{A(\tilde s)}\biggr)^2 + \frac1{\tilde s^2}
  \biggr) \alpha^2 A^2(\tilde s)T + \ldots \biggr\}.
\end{multline}
\end{theorem}

\begin{remark}
Это не строгий результат.
Во-первых, \emph{приближенная} формула означает, что есть некоторый метод, который позволяет написать следующие члены в разложении \eqref{lv:hw}, но готовой аналитической формулы для них нет.
Метод основан на введении параметра в уравнение с частными производными для цены опциона и разложении решения по степеням этого параметра.
Во-вторых, разложение проводится формальным образом, не заботясь о сходимости получающихся рядов. Тем не менее, для <<не слишком плохих>> функций $a(t)$ и $A(s)$ численно показывается, что приближенная формула дает хорошее соответствие с настоящей подразумеваемой волатильностью.
\end{remark}

\begin{remark}
В случае ненулевой процентной ставки будет справедлив аналогичный результат, если предположить, что модели \eqref{lv:homogenous} удовлетворяет $T$"=форвардная цена.
Тогда в формуле \eqref{lv:hw} нужно заменить $S_0$ на $F_0$ и $\tilde s$ на $\tilde f = (F_0+K)/2$.
Детали см.~в статье \cite{HaganWoodward99}.
\end{remark}

\begin{remark}
Из формулы \eqref{lv:hw} можно увидеть, как изменяется подразумеваемая волатильность при изменении цены базового актива и, в частности, объяснить, почему в модели локальной волатильности получается неверная динамика цен опционов и подразумеваемой волатильности.

Пусть $\hat\sigma_0(T,K)$ обозначает улыбку волатильности для фиксированного момента исполнения $T$ в момент времени $t=0$.
Если рассматривать страйки $K$ вблизи значения $S_0$, то в формуле \eqref{lv:hw} можно ограничиться только первым членом в правой части (вклад остальных будет мал). 
Тогда получим
\[
A(s) \approx \frac{1}{\alpha} \hat\sigma(T, 2 s - S_0) s.
\]
Если заменить цену $S_0$ на $S_0+\epsilon$, то новая улыбка волатильности станет равна
\[
\hat\sigma_{n}(T, K) 
\approx \frac{A((S_0+ \epsilon + K)/2)}{\alpha (S_0+\epsilon+K)/2}
\approx \hat\sigma(T, K + \epsilon).
\]
Следовательно, если $\epsilon>0$, то улыбка сдвигается влево, а если $\epsilon<0$, то вправо.
В реальности наблюдается движение в другую сторону (см.~обсуждение в разделе \ref{lv:s:discussion} лекции \ref{ch:locvol}).
\end{remark}

%%%%%%%%%%%%%%%%%%%%%%%%%%%%%%%%%%%%%%%%%%%%%%%%%%%%%%%%%%%%%%%%%%%%%
%%% Старая картинка
%%%%%%%%%%%%%%%%%%%%%%%%%%%%%%%%%%%%%%%%%%%%%%%%%%%%%%%%%%%%%%%%%%%%%
% \medskip
% \begin{figure}
% \centering
% \begin{tikzpicture}
% \draw (1,0) -- (3,0);
% \draw[->] (4,0)--(5.5,0)  node[below] {\footnotesize $K$};
% \draw plot [smooth] coordinates {(2,2.5) (3,0.5) (4.5,2.5)} 
%   node[right] {\footnotesize $\hat\sigma_0$};
% \draw[dashed] plot [smooth] coordinates {(1,2.5) (2,0.5) (3.5,2.5)}
%   node[right] {\footnotesize $\hat\sigma_{\Delta t}$};
% \draw (3,0) node[below] {\footnotesize $S_0$};
% \draw (4,0) node[below,xshift=5mm] {\footnotesize $S_0+\epsilon$};
% \draw[blue,thick] (3,0.05)--(3,-0.05);
% \draw[blue,thick,->] (3,0) -- (4,0);
% \draw[red,thick,->] (2.18,2) -- (1.18,2);
% \end{tikzpicture}
% \caption{Движение улыбки волатильности при изменении цены базового актива.}
% \label{lv:f:smile}
% \end{figure}

\subsection{Схема <<доказательства>>}

Для краткости покажем лишь, как получить первые два члена в приближенной формуле.
Для последующих членов рассуждения аналогичны, но быстро становятся слишком громоздкими.

Зафиксируем страйк $K$ и время исполнения $T$.
Как следует из формулы Фейнмана"--~Каца, цена опциона $C(t,s) = \E((S_T-K)^+\mid S_t=s)$ удовлетворяет уравнению
\begin{equation}
\label{lv:fk}
\left\{
\begin{aligned}
&C_t(t,s) + \frac{A^2(s)}{2} C_{ss}(t,s) = 0,\\
&C(T,s) = (s-K)^+,
\end{aligned}
\right.
\end{equation}
где нижние индексы означают дифференцирование по соответствующим переменным (для удобства будем опускать сам знак производной).
Положим $\epsilon = A(K)$ и сделаем замену переменных
\begin{equation}
\label{lv:hw-change}
\tau = T-t, \qquad x = \frac{s-K}{\epsilon}, \qquad 
Q(\tau,x) = \frac{C(t(\tau), s(x))}{\epsilon}.
\end{equation}
Тогда уравнение \eqref{lv:fk} запишется в виде
\begin{equation}
\label{lv:fk-2}
\left\{
\begin{aligned}
&Q_\tau(\tau,x) - \frac{A^2(K+\epsilon x)}{2A^2(K)} Q_{xx}(\tau,x) = 0,\\
&Q(0,x) = x^+.
\end{aligned}
\right.
\end{equation}
Обозначим $\nu_1 = A'(K)/A(K)$ и разложим в ряд, формально считая $\epsilon$ малым параметром: 
\begin{gather*}
A(K+\epsilon x) 
  = A(K)(1+\epsilon\nu_1 x + \ldots),\\
Q(\tau,x) 
  = Q^{(0)}(\tau,x) + \epsilon Q^{(1)}(\tau,x)+ \ldots,
\end{gather*}
где $Q^{(0)}$, $Q^{(1)}$ "--- неизвестные функции. Подставляя эти разложения в уравнение \eqref{lv:fk-2} и группируя члены при степенях $\epsilon$, получаем уравнения
\[
\left\{
  \begin{aligned}
  &Q^{(0)}_\tau - \frac12 Q^{(0)}_{xx} = 0,\\
  &Q^{(0)}(0,x) = x^+,
  \end{aligned}
\right.\hspace{1cm}
\left\{
  \begin{aligned}
  &Q^{(1)}_\tau - \frac12 Q^{(1)}_{xx} = \nu_1 x Q^{(0)}_{xx},\\
  &Q^{(1)}(0,x) = 0
  \end{aligned}
\right.
\]
(для членов при $\epsilon^2$, $\epsilon^3$ и \td\ будут получаться аналогичные системы, которые мы не приводим для краткости изложения).
Решая эти уравнения, находим\footnote{В том, что это действительно решения, можно убедиться подстановкой, используя соотношения
\[
G_\tau(\tau, x) = \frac{1}{2\sqrt\tau}\phi\left(\frac{x}{\sqrt\tau}\right), \qquad 
G_{xx}(\tau, x) = \frac{1}{\sqrt\tau}\phi\left(\frac{x}{\sqrt\tau}\right), \qquad 
G_{\tau x} = -\frac{x}{\tau} G_\tau.
\]
}
\begin{align*}
&Q^{(0)}(\tau,x) = G(\tau,x) 
  := x \Phi(x/\sqrt\tau) + \sqrt\tau \phi(x/\sqrt{\tau}),\\
&Q^{(1)}(\tau,x) = \nu_1\tau x G_\tau(\tau,x).
\end{align*}
Таким образом, 
\[
Q = Q^{(0)} + \epsilon Q^{(1)} +\ldots = G + \epsilon\nu_1\tau x G'_\tau + \ldots
\]
Теперь возьмем такое $\tilde\tau$, что $Q(\tau,x) = G(\tilde \tau,x)$:
\[
\tilde \tau = \tau(1 + \epsilon\nu_1x + \ldots).
\]
Из того, как мы определили $Q$ в \eqref{lv:hw-change}, следует, что 
\[
C(t,s) = \epsilon G(\tilde \tau,x) = G(\epsilon^2\tilde\tau, \epsilon x) =
G(\tau^*, s-K),
\]
где
\[
\tau^* = \epsilon^2\tilde\tau 
= A^2(K)\tau (1 + \nu_1(s-K) + \ldots). 
\]
Из разложения $\sqrt{1+z} = 1 + \frac z2 - \frac{z^2}{8} + \ldots$, получаем
\[
\sqrt{\tau^*} = A(K)\sqrt{\tau} 
\biggl( 1 + \frac12 \nu_1 (s-K) + \ldots
\biggr).
\]
Раскладывая $A(K)$ в ряд в точке $\tilde s=\frac12 (s+K)$, получаем
\begin{equation}
\label{lv:tau}
\sqrt{\tau^*} = A(\tilde s) \sqrt\tau 
\biggl(1 
  + \frac{\gamma_2 - 2\gamma_1^2}{24}(s-K)^2 
  + \ldots 
\biggr).
\end{equation}
где $\gamma_1 = A'(\tilde s)/A(\tilde s)$, $\gamma_2 = A''(\tilde s)/A(\tilde s)$ и \td

Таким же образом для модели \bs\  (нужно взять $A(s) = \hat\sigma s$), находим
\begin{equation}
\label{lv:tau-bs}
\sqrt{\tau^*_{BS}} = \hat\sigma \tilde s \sqrt{\tau} 
\biggl(1 
  - \frac{1}{12\tilde s^2} (s-K)^2
  + \ldots
\biggr).
\end{equation}
Равенство $C(t,s) = C_{BS}(t,s)$ влечет $G(\tau^*,s-K) = G(\tau^*_{BS}, s-K)$, и, следовательно, $\tau^* = \tau^*_{BS}$ в силу монотонности $G$. Отсюда получаем, что $\sqrt{\tau^*} = \sqrt{\tau^*_{BS}}$.
Выражая теперь $\hat\sigma$ из \eqref{lv:tau}--\eqref{lv:tau-bs}, получаем
\begin{multline*}
\hat\sigma = \frac{A(\tilde s)}{\tilde s} \biggl(1 + \frac{\gamma_2 - 2\gamma_1^2}{24}(s-K)^2  + \ldots  \biggr)
\biggl(1 - \frac{1}{12\tilde s^2} (s-K)^2 + \ldots \biggr)^{-1} \\= 
\frac{A(\tilde s)}{\tilde s} 
\biggl\{1 
+ \frac1{24} 
  \biggl(
    \frac{A''(\tilde s)}{A(\tilde s)} - 2\biggl(\frac{A'(\tilde s)}{A(\tilde s)}\biggr)^2 + \frac{2}{\tilde s^2}
  \biggr) (s-K)^2 + \ldots\biggr\},
\end{multline*}
где во втором равенстве воспользовались разложением $(1-z+\ldots)^{-1} = 1 + z + \ldots$
Полагая $s=S_0$, получаем доказываемую формулу.
