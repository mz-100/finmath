%!TEX root=finmath2.tex

\chapter{Симуляция модели SABR}
\label{ch:sabr-sim}

Рассмотрим схему симуляции модели SABR, которая является более эффективной по сравнению со схемой Эйлера. 
Идея ее реализации напоминает схему QE для модели Хестона и основана на нахождении хорошего приближения интеграла от квадрата стохастической волатильности.

Мы ограничимся случаями $\beta=0$ и $\beta=1$ как наиболее простыми; в работе \cite{ChenOsterlee12} можно найти остальные.

Представим модель SABR в виде
\begin{align*}
&d S_t = \alpha_t S_t^\beta (\rho dW_t^{(1)} + \sqrt{1-\rho^2} dW_t^{(2)}),\\
&d \alpha_t = \nu\alpha_t d W_t^{(1)}, 
\end{align*}
где броуновские движения $W^{(1)}$ и $W^{(2)}$ независимы.

Как обычно, процесс $S_t$ симулируют последовательно в точках $0=t_0<t_1<\ldots<t_n$.
Пусть $\hat S_{t_i}$ обозначает симулированное значение в $t_i$.
Покажем, как из $\hat S_{t_i}$ получить $\hat S_{t_{i+1}}$. Заметим, что процесс $\alpha_t$ симулировать просто (это геометрическое борновское движение), поэтому можно считать, что нам уже даны значения $\alpha_{t_i}$, а нужно получить значение величины $\hat S_{t_{i+1}}$ из ее условного распределения при условии значений $\hat S_{t_i}$ и $\alpha_{t_i}$, $\alpha_{t_{i+1}}$.

\medskip
\textit{Случай $\beta =0$}.
Заметим, что
\[
S_{t_{i+1}} = S_{t_i} + \rho\int_{t_i}^{t_{i+1}} \alpha_s d W_s^{(1)} + \sqrt{1-\rho^2}\int_{t_i}^{t_{i+1}} \alpha_s dW_s^{(2)}
= \rho I^{(1)}_i + \sqrt{1-\rho^2}I^{(2)}_i,
\]
где для краткости $I^{(1)}_i$ и $I^{(2)}_i$ обозначают соответствующие интегралы.
Из уравнения для $\alpha_t$ следует, что 
\[
\alpha_{t_{i+1}} = \alpha_{t_i} + \nu\int_{t_i}^{t_{i+1}}\alpha_s d W_s^{(1)} = \alpha_{t_i} + \nu I^{(1)}_i.
\]
Таким образом,
\[
I^{(1)}_i = \frac1\nu(\alpha_{t_{i+1}} - \alpha_{t_i}).
\]
Так как процесс $\alpha_t$ не зависит от $W_t^{(2)}$, то условное распределение $I^{(2)}_i$ при условии траектории $\alpha_s$, $s\in[t_i,t_{i+1}]$, является нормальным с нулевым средним и дисперсией $A_i$, где
\[
A_i = \int_{t_i}^{t_{i+1}} \alpha_s^2 ds.
\]
Далее приблизим условное распределение $\Law(A_i\mid \alpha_{t_i},\alpha_{t_{i+1}})$ лог-нормальным распределением с такими же условными средним и дисперсией.

Обозначим $\xi_i = W_{t_{i+1}}^{(2)} - W_{t_i}^{(2)}$, причем заметим, что эта величина однозначно определяется по значениям $\alpha_{t_i}$ и $\alpha_{t_{i+1}}$ из равенства
\[
\alpha_{t_{i+1}} = \alpha_{t_i} e^{\nu\xi_i - \frac{\nu^2}{2} (t_{i+1}-t_i)}.
\]
В работе \cite{ChenOsterlee12} показывается, что для $m_i:= \E (A_t \mid W_t)$ и $v_i := \D( A_t \mid W_t)$ верны приближенные формулы
\begin{align*}
m_i &\approx \alpha_{t_i}^2\Delta t 
  \biggl(1 + \nu \xi_i + \frac{\nu^2}3 \left(2 \xi_i^2 - \frac{\Delta t}2\right) 
    + \frac{\nu^3}3 (\xi_i^3 - \Delta t\xi_i) \\&\qquad+ \frac{\nu^4}5 \left(\frac23 \xi_i^4 - \frac32 \xi_i^2 + 2(\Delta t)^2\right)\biggr),\\
v_i &\approx \frac 13 \alpha_{t_i}^4 \nu^2 (\Delta t)^3,
\end{align*}
где $\Delta t = t_{i+1} - t_i$.
Тогда будем симулировать
\[
\hat A_i = e^{\mu_i + \sigma_i N_i},
\]
где $N_i$ "--- стандартная нормальная случайная величина, не зависящая от $\alpha_t$, а параметры $\mu_i,\sigma_i$ выбираются так, что $\E A_i = m_i$, $\D A_i = v_i$, для чего нужно взять
\[
\mu_i = \ln(m_i) -\frac12 \ln\left(1+\frac{v_i}{m_i^2}\right), \qquad
\sigma_i^2 = \ln\left(1+\frac {v_i}{m_i^2}\right).
\]
Собирая вместе вышесказанное, получаем, что для симуляции $\hat S_{t_{i+1}}$ следует использовать формулу
\[
\hat S_{t_{i+1}} = \frac{\rho}{\nu}(\alpha_{t_{i+1}} - \alpha_{t_i}) + \sqrt{1-\rho^2} e^{\frac12(\mu_i+\sigma_i N_i)} \zeta_i,
\]
где $N_i,\zeta_i$ "--- независимые стандартные нормальные случайные величины.


\medskip
\textit{Случай $\beta=1$.}
По явной формуле для стохастической экспоненты процесс $S_t$ допускает представление
\begin{align*}
S_{t_{i+1}} &= S_{t_i}\exp\left(\rho \int_{t_i}^{t_{i+1}} \alpha_s d W_s^{(1)} 
  + \sqrt{1-\rho^2} \int_{t_i}^{t_{i+1}} \alpha_s d W_s^{(2)} 
  - \frac12 \int_{t_i}^{t_{i+1}} \alpha_s^2 ds 
\right) \\
&= S_{t_i}\exp\left(\rho I_i^{(1)} + \sqrt{1-\rho^2} I_i^{(2)} - \frac12 A_i\right),
\end{align*}
где $I_i^{(1)}$, $I_i^{(2)}$, $A_i$ такие же, как и выше.
Тогда из рассуждений для случая $\beta=0$ получаем, что можно симулировать значение $\hat S_{t_{i+1}}$ используя формулу
\[
\hat S_{t_{i+1}}= \hat S_{t_i}\exp\left(\frac\rho\nu(\alpha_{t_{i+1}}-\alpha_{t_i}) + \sqrt{1-\rho^2} e^{\frac12(\mu_i+\sigma_i N_i)} \zeta_i - \frac12 e^{\mu_i+\sigma_i N_i}\right).
\]
