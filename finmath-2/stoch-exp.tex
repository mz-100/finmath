%!TEX root=finmath2.tex
\chapter{Стохастическая экспонента и условие Новикова}
\label{ch:stoch-exp}


Пусть $X_t$ "--- одномерный процесс Ито, заданный на некотором фильтрованном вероятностном пространстве и имеющий вид
\[
dX_t = a_t dt + b dW_t,
\]
где $a_t,b_t$ "--- некоторое случайные процессы, для которых соответствующие интегралы корректно определены.

\begin{definition}
\emph{Стохастической экспонентой} процесса $X$ называется непрерывный согласованный процесс $Y$, удовлетворяющий уравнению
\begin{equation}
\label{hes:exp}
dY_t = Y_t dX_t,\qquad Y_0=1.
\end{equation}
Обозначение: $Y_t = \mathcal{E}(X)_t$. 
\end{definition}

\begin{remark}
Уравнение \eqref{hes:exp}, как обычно, нужно понимать в интегральном смысле, \te\
\[
Y_t = 1 + \int_0^t Y_s d X_s = 1 + \int_0^t b_sY_s ds + \int_0^t \sigma_s Y_s W_s.
\]
Стохастическую экспоненту можно определить для любого семимартингала $X$, но мы далее для простоты ограничимся только процессами Ито. 
\end{remark}

\begin{example}
Геометрическое броуновское движение $dS_t = \mu S_t dt + \sigma S_t dW_t$ является стохастической экспонентой броуновского движения со сносом $X_t = \mu t + \sigma W_t$. 
\end{example}

\begin{proposition}[\emph{формула Долеан"--~Дэд}]
Уравнение \eqref{hes:exp} имеет единственное сильное решение
\[
Y_t = \exp\left(X_t - X_0 - \int_0^t \sigma_s^2 ds\right).
\]
\end{proposition}

Тот факт, что такой процесс $Y_t$ является решением уравнения \eqref{hes:exp} проверяется непосредственно по формуле Ито. Доказательство единственности см., например, в книге \cite{JacodShiryaev94}, гл.~I, \S\,4f.

\medskip
Если у процесса $X_t$ коэффициент сноса $b_t=0$, то стохастическая экспонента $Y_t$ является локальным мартингалом.
Возникает вопрос, а когда она является настоящим мартингалом? 
Стандартным способом проверки является следующее условие Новикова, которое уже упоминалось в курсе <<Введение в финансовую математику>> при обсуждении условий выполнения теоремы Гирсанова.
Доказательство см.~в книге \cite{KaratzasShreve91}, гл.~3.

\begin{proposition}[\emph{условие Новикова}]
\label{stochexp:novikov}
Предположим, что $d X_t = \sigma_t dW_t$ и выполнено условие 
\[
\E e^{\int_0^T \sigma_t^2 dt} < \infty.
\]
Тогда стохастическая экспонента $Y_t = \mathcal{E}(X)_t$ является мартингалом.
\end{proposition}

\begin{corollary}[см.~\cite{KaratzasShreve91}, гл.~3, следствие 5.14]
\label{stochexp:novikov-corollary}
Стохастическая экспонента процесса $d X_t = \sigma_t dW_t$, $t\in[0,T]$, является мартингалом, если найдется $\epsilon>0$ такое, что для всех $t\in[0,T-\epsilon]$ выполнено неравенство
\[
\E e^{\int_t^{t+\epsilon} \sigma_s^2 ds} < \infty.
\]  
\end{corollary}
