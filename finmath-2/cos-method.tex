%!TEX root=finmath2.tex
\chapter{Метод косинусов для модели Хестона}
\label{ch:cos-method}

\emph{Метод косинусов}, предложенный в работе \cite{FangOosterlee09}, позволяет эффективно находить цены платежных обязательств, не зависящих от траектории цены базового актива, в моделях, где имеется аналитическое представление для характеристической функции распределения цены базового актива.
Здесь мы рассмотрим метод косинусов применительно к модели Хестона.


\section{Напоминание: ряды Фурье}

Метод косинусов основан на разложении плотности распределения цены базового актива в ряд Фурье, поэтому сначала приведем общие сведения о рядах Фурье, которые потребуются далее.

\emph{Рядом Фурье} для периодической функции $p(x)$ с периодом $2\pi$ называется разложение
\[
p(x) = \frac{A_0}{2} + \sum_{n=1}^\infty (A_n\cos nx + B_n \sin nx)
\]
c \emph{коэффициентами Фурье}
\[
A_n = \frac{1}{\pi}\int_{-\pi}^\pi p(x) \cos(nx) dx, \quad 
B_n = \frac{1}{\pi}\int_{-\pi}^\pi p(x) \sin(nx) dx.
\]
Если $p\in L^2([-\pi,\pi])$, то ряд Фурье сходится к $p(x)$ в $L^2$. Если $p(x)$ дифференцируема в точке $x$, то ряд Фурье в этой точке сходится к $p(x)$; более точное условие поточечной сходимости дает признак Дини.

Если функция $p(x)$ задана на отрезке $[0,\pi]$, то ее можно разложить в ряд, состоящий только из членов с косинусами (при условии сходимости ряда)
\[
p(x) = \frac{A_0}{2} + \sum_{n=1}^\infty A_n \cos(nx), \qquad
A_n = \frac{2}{\pi}\int_0^\pi p(x) \cos(nx) dx.
\]
Эта формула следует из разложения периодического продолжения четной функции $\tilde p(x) = p(|x|)$, $x\in[-\pi,\pi]$.

Если же функция $p(x)$ задана на произвольном отрезке $[a,b]$, то с помощью замены переменной получается разложение
\begin{equation}
\label{cos:p-fourier}
p(x) = \frac{A_0}{2} + \sum_{n=1}^\infty A_n \cos\left(n\pi\frac{x-a}{b-a}\right), \quad
A_n = \frac{2}{b-a}\int_a^b p(x) \cos\left(n\pi\frac{x-a}{b-a}\right) dx,
\end{equation}
при этом заметим, что коэффициенты $A_n$ можно представить в виде
\begin{equation}
\label{cos:A}
A_n = \frac{2}{b-a} \mathrm{Re}\left\{ \tilde\phi\left(\frac{n\pi}{b-a}\right)\exp\left(-i\frac{na\pi}{b-a}\right) \right\},
\end{equation}
где
\[
\tilde\phi(u) = \int_a^b e^{iux} p(x) dx.
\]


\section{Формула для цен платежных обязательств}

Рассмотрим модель рынка, в которой рисковый актив имеет цену $S_t$. 
Пусть $\phi(u)$ "--- характеристическая функция распределения случайной величины $X_T = h(S_T)$ в заданный момент исполнения $T$, где $h(s)$ выбирается из соображений удобства (например, для вычисления европейских опционов в модели Хестона $h(s) = \ln(S_T/K)$).
Будем считать, что безрисковая процентная ставка равна нулю, рисковый актив не выплачивает дивиденды, а исходная вероятностная мера $\P$ уже является мартингальной\footnote{Модель с произвольной детерминированной безрисковой ставкой сводится к этому случаю, см.~раздел \ref{gen:s:forward} в лекции \ref{ch:general}.}.
Кроме того, пусть $X_T$ имеет абсолютно непрерывное распределение с плотностью $p(x)$. 

Рассмотрим европейское платежное обязательство с выплатой $f(X_T)$.
Нахождение его цены $V$ в начальный момент времени сводится к вычислению интеграла
\[
V = \E f(X_T) = \int_\R f(x) p(x) ds.
\]
Чтобы вычислить этот интеграл, сделаем приближения, позволяющие свести задачу к вычислению суммы, члены которой включают значения характеристической функции $\phi(x)$ в разных точках.
Контроль точности будет обсуждаться в следующем разделе.

Сначала выберем достаточно большой отрезок $[a,b]$, так что можно приблизить
\begin{equation}
\label{cos:V1}
V \approx V^{(1)} := \int_{a}^{b} f(x) p(x) dx.
\end{equation}
Подставим сюда разложение \eqref{cos:p-fourier} для $p(x)$ и возьмем конечное число членов ряда:
\begin{multline}
\label{cos:V2}
V^{(1)} \approx V^{(2)} := \int_a^b f(x) \biggl( \frac{A_0}{2} + \sum_{n=0}^\infty A_n \cos\left(n\pi\frac{x-a}{b-a}\right) \biggr) dx \\
= \frac{b-a}{2} \left( \frac{A_0V_0}{2} + \sum_{n=1}^\infty A_nV_n  \right)
\approx \frac{b-a}{2} \left( \frac{A_0V_0}{2} + \sum_{n=1}^{N} A_nV_n \right),
\end{multline}
где
\[
V_n = \frac{2}{b-a}\int_a^b f(x) \cos\left(n\pi\frac{x-a}{b-a}\right) dx.
\]
Далее, принимая во внимание формулу \eqref{cos:A}, приблизим 
\begin{equation}
\label{cos:F}
A_n \approx F_n := \frac{2}{b-a} \mathrm{Re}\left\{ \phi\left(\frac{n\pi}{b-a}\right)\exp\left(-i\frac{na\pi}{b-a}\right) \right\},
\end{equation}
где $\phi(u)=\int_\R e^{iux}p(x)dx$ "--- характеристическая функция $X_T$ (аппроксимация здесь состоит в том, что интеграл по $[a,b]$ в определении $\tilde\phi$ заменяется на интеграл по $\R$).
Таким образом, получаем 
\[
V^{(2)} \approx V^{(3)} := \frac{b-a}{2} 
\left(\frac{F_0V_0}{2} + \sum_{k=1}^N F_nV_n\right).
\]
В итоге получаем следующую формулу для цены платежного обязательства:
\[
V\approx 
\frac{V_0}{2} 
  + \sum_{n=1}^N \mathrm{Re}\left\{ 
    \phi\left(\frac{n\pi}{b-a}\right)\exp\left(-i\frac{na\pi}{b-a}\right) 
  \right\} V_n.
\]

Для модели Хестона воспользуемся тем, что характеристическая функция $\phi$ имеет вид
\[
\phi(u) = \phi(u; 0,0,v_0) e^{iux_0},
\]
где выражение для $\phi(u;t,x,v) = \E(e^{iu\ln S_{T-t}} \mid \ln S_0=x, V_0=v)$ было дано в разделе \ref{hes:s:stetement} лекции \ref{ch:heston-formula}.
Отсюда находим
\[
V \approx 
  \frac{V_0}{2} 
    + \sum_{n=1}^N \mathrm{Re}\left\{
      \phi\left(\frac{n\pi}{b-a};0,0,v\right) \exp\left(in\pi \frac{x_0-a}{b-a}\right) 
    \right\} V_n,
\]
где $x_0 = \ln(S_0/K)$.

Заметим, что коэффициенты $V_n$ не зависят от модели рынка, а зависят только от функции выплаты производного инструмента и выбора отрезка $[a,b]$.
Для конкретных функций выплат их можно найти явно.

Например, для опциона колл со страйком $K$ возьмем $h(s) = \ln(S_T/K)$.
Тогда в качестве функции выплаты $f(x)$ нужно взять $f(x) = K(e^x-1)^+$.
Считая, что $a<0<b$, получаем
\[
V_n^\text{call} = K\frac{2}{b-a}\int_0^b (e^x-1) \cos\left(n\pi\frac{x-a}{b-a}\right) dx.
\]
Полученный интеграл вычисляется аналитически:
\[
V_n^\text{call} = K\frac{2}{b-a}(\chi_n(0,b) - \psi_n(0,b))
\]
с функциями
\begin{multline*}
\chi_n(c,d) = \frac{1}{1+\left(\frac{n\pi}{b-a}\right)^2}\biggl[\cos\left(n\pi \frac{d-a}{b-a}\right)e^d - \cos\left(n\pi\frac{c-a}{b-a}\right)e^c \\
+\frac{n\pi}{b-a}\sin\left(n\pi\frac{d-a}{b-a}\right)e^d - \frac{n\pi}{b-a}\sin\left(n\pi\frac{c-a}{b-a}\right)e^c\biggr],
\end{multline*}
а также $\psi_0(c,d) = d-c$ и
\[
\psi_n(c,d) = \left[\sin\left(n\pi\frac{d-a}{b-a}\right) - \sin\left(n\pi\frac{c-a}{b-a}\right)\right]\frac{b-a}{np}, \qquad n\ge 1.
\]
Аналогично, для опциона пут
\[
V_n^\text{put} = K\frac{2}{b-a}(-\chi_n(a,0) + \psi_n(a,0)).
\]


\section{Контроль точности}

Вычисление цены $V$ включает в себя ошибки аппроксимации при замене интегрирования по всей прямой на интегрирование по отрезку $[a,b]$ и обратно в формулах \eqref{cos:V1} и \eqref{cos:F}, а также в отбрасывании хвоста ряда в формуле \eqref{cos:V2}.
Что касается последней ошибки, то в работе \cite{FangOosterlee09} показано, что ряд сходится экспоненциально быстро, если плотность $p(x)$ гладкая.
На практике можно поступить так: продолжать вычислять следующие члены ряда, пока они не станут меньше определенного порога, после чего оставшиеся члены отбросить.
Таким образом, остается определить, как выбрать отрезок $[a,b]$.
В \cite{FangOosterlee09} для модели Хестона в качестве эмпирического правила предложены значения
\[
a_1 = c_1 - 12 \sqrt{|c_2|}, \qquad b = c_1 + 12\sqrt{|c_2|},
\]
где $c_1,c_2$ "--- первый и второй семиинварианты для величины $X=\ln (S_T/K)$, \te
\[
c_1 = \prt{}{u} \ln(\E e^{uX}) \Big|_{u=0}, \qquad
c_2 = \prtt{}{u} \ln(\E e^{uX}) \Big|_{u=0}.
\]
В явном виде
\begin{align*}
c_1 &= (1-e^{-\kappa T})\frac{\theta-v_0}{2\kappa} - \frac12\theta T,\\
c_2 &=\frac{1}{8\kappa^3}\Bigl(\sigma T \kappa e^{-\kappa T}(v_0-\theta)(8\kappa\rho-4\sigma) 
+\kappa\rho\sigma(1-e^{-\kappa T})(16\theta-8v_0)\\
&+2\theta\kappa T(-4\kappa\rho\sigma+\sigma^2+4\kappa^2)
+\sigma^2\bigl((\theta-2v_0)e^{-2\kappa T} + \theta(6e^{-\kappa T}-7)+2v_0\bigr)\\
&+8\kappa^2(v_0-\theta)(1-e^{-\kappa T})\Bigr).
\end{align*}
