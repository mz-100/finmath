%!TEX root=finmath2.tex

\chapter{Стохастические дифференциальные уравнения}
\label{ch:sde}
\chaptertoc

Модели стохастической волатильности описываются с помощью стохастических дифференциальных уравнений, задающих процессы цены рискового актива и волатильности.
В этой лекции мы обсудим общие вопросы, связанные со стохастическими дифференциальными уравнениями: существование и единственность решения, а также свойства, которыми решение обладает.

Большинство результатов даны без доказательств.
Их можно найти в гл.~5 книги \cite{KaratzasShreve91} и гл.~1 книги \cite{ChernyEngelbert}.


\section{Определения}
\subsection{Слабые и сильные решения}

\emph{Стохастическим дифференциальным уравнением} (СДУ) называется уравнение вида
\begin{equation}
\label{sde:sde}
d X_t = a(t,X_t)dt + b(t,X_t)dW_t, \qquad X_0=x,
\end{equation}
где $X_t$ "--- неизвестный случайный процесс, а $W_t$ "--- броуновское движение.
В общем случае эти процессы могут быть многомерными: $X_t = (X_t^1,\dots,X_t^n)$ и $W_t=(W_t^1,\dots,W_t^d)$.
Тогда $a(t,x)$ является измеримой функцией из $\R_+\times\R^n$ в $\R^n$, а $b(t,x)$ "--- измеримой функцией из $\R_+\times\R^n$ в $\R^{n\times d}$, и уравнение \eqref{sde:sde} нужно понимать в векторном смысле (точный смысл см.\ в следующем определении).
Начальное условие $x\in\R^n$ будем считать неслучайным.

Функция $a(t,x)$ называется \emph{коэффициентом сноса}, а функция $b(t,x)$ "--- \emph{коэффициентом диффузии}.

\begin{definition}
\emph{Решением} (часто говорят \emph{слабым решением}) уравнения \eqref{sde:sde} называется совокупность $(\Omega,\F,(\F_t)_{t\ge0},\P,W,X)$, где
\begin{itemize}
\item $(\Omega,\F,(\F_t)_{t\ge0},\P)$ "--- фильтрованное вероятностное пространство,
\item $W$ "--- броуновское движение, определенное на этом пространстве,
\item $X$ "--- непрерывный согласованный процесс, определенный на этом пространстве, и такой, что для любого $t\ge 0$ выполнено неравенство%
\footnote{Это неравенство требуется, чтобы интегралы в формуле \eqref{sde:strong} были корректно определены.}
\begin{equation}
\label{sde:strong-ineq}
\int_0^t \|a(s,X_s)\|ds + \int_0^t \|b(s,X_s)\|^2ds < \infty,
\end{equation}
а также для любого $t\ge 0$ и $n=1,\dots,N$ выполнено равенство
\begin{equation}
\label{sde:strong}
X_t^n = x + \int_0^t a^n(s,X_s)ds + \sum_{i=1}^d\int_0^t b^{ni}(s,X_s)dW_s^i.
\end{equation}
\end{itemize}
\end{definition}

Далее для краткости мы будем говорить просто <<решение $X$>>, имея ввиду всю совокупность $(\Omega,\F,(\F_t)_{t\ge0},\P,W,X)$.

\begin{definition}
Решение уравнения \eqref{sde:sde} называется \emph{сильным}, если процесс $X$ согласован с фильтрацией $\F_t^W = \sigma(W_s, s\le t)$, порожденной броуновским движением и пополненной\footnote{Пополнение означает, что в $\F_0$ добавлены подмножества событий нулевой вероятности.}.
\end{definition}

Поясним различие между сильным и слабым решениями.
В слабом решении ищется и сам неизвестный процесс $X$, и вероятностное пространство, на котором определен этот процесс и броуновское движение $W$.
Если же существует сильное решение, то можно показать, что оно является функционалом от траекторий броуновского движения (см.\ \cite{ChernyEngelbert}, предложение 1.5), \te\ найдется измеримое отображение $\Phi\colon C(R_+) \to C(\R_+)$ такое, что $X_t = \Phi(W)_t$.
Отсюда следует, что сильное решение можно построить на любом вероятностном пространстве c броуновским движением $W$.
Если решения является слабым, но не является сильным, то это означает, что для его построения недостаточно лишь одного источника случайности в виде броуновского движения (оно содержит дополнительную <<рандомизацию>>).

Модели стохастической волатильности, с которыми мы будем работать далее, будут задаваться сильными решениями СДУ.
Понятие слабого решения нам нужно затем, что для доказательства существования сильного решения часто сначала проще показать, что существует слабое, а затем воспользоваться результатами о том, когда слабое решения является сильным.

\begin{example}
\emph{Уравнение Танаки} 
\[
dX_t = \sgn(X_t) d W_t, \qquad X_0=0,
\]
где $\sgn(x) = 1$ при $x> 0$ и $\sgn(x) = -1$ при $x\le0$, имеет слабое решение, являющееся броуновским движением. Его можно построить так: в качестве $X_t$ возьмем броуновское движение, заданное на каком-то вероятностном пространстве, и положим $W_t = \int_0^t \sgn(X_s) d X_s$. По \emph{теореме Леви}%
\footnote{Непрерывный локальным мартингал $W_t$ с $W_0=0$ является броуновским движением тогда и только тогда, когда $W_t^2 - t$ является локальным мартингалом.}
$W$ является броуновским движением, и, следовательно, процессы $(X,W)$ задают слабое решение.
Можно показать, что сильного решения это уравнение не имеет (см.~\cite{ChernyEngelbert}, гл.~1, пример 1.18).
\end{example}


\subsection{Слабая и сильная единственность}

\begin{definition}
Уравнение \eqref{sde:sde} обладает свойством \emph{слабой единственности} решения (также говорят \emph{единственности по распределению}), если любые два решения $X$ и $\tilde X$ совпадают по распределению, \te\ равны все конечномерные распределения $(X_{t_1},\dots,X_{t_k}) \stackrel{d}{=} (\tilde X_{t_1},\dots,\tilde X_{t_k})$
\end{definition}

\begin{definition}
Уравнение \eqref{sde:sde} обладает свойством \emph{сильной единственности} решения (также говорят \emph{потраекторной единственности}), если любые два решения $X$ и $\tilde X$, заданные на одном и том же вероятностном пространстве и удовлетворяющее уравнению \eqref{sde:sde} c одним и тем же броуновским движением, совпадают с вероятностью~1, \te\ $X_t=\tilde X_t$ \as\ для всех $t\ge 0$.
\end{definition}

Следующая теорема Ямады"--~Ватанабе будет очень полезна в дальнейшем: она утверждает, что существование сильного решения можно получить из существования слабого и свойства сильной единственности (а сильная единственность будет иметь место, если выполнены условия другой теоремы Ямады"--~Ватанабе, см.~далее).

\begin{theorem}[Т.~Ямада, C.~Ватанабе]
\label{sde:t:yw-1}
Из сильной единственности следует слабая единственность.
Из сильной единственности и существования слабого решения следует следует существование сильного решения.
\end{theorem}

Далее для краткости мы будем говорить, что у некоторого СДУ <<существует единственное слабое решение>>, если слабое решение существует и выполнено свойство слабой единственности.
Аналогично, фраза <<существует единственное сильное решение>> означает существование сильного решения и выполнение свойства сильной единственности.


\section{Условия существования и единственности решения}
\subsection{Сильные решения}

\begin{theorem}[К.~Ито]
Пусть существует константа $C$ такая, что для всех $t\ge 0$ и $x,y\in \R^n$ выполнены неравенства
\begin{align*}
&\|a(t,x) - a(t,y)\| + \|b(t,x) - b(t,y)\| \le C \|x-y\|,\\
&\|a(t,x)\| + \|b(t,x)\| \le C(1+\|x\|).
\end{align*}
Тогда уравнение \eqref{sde:sde} имеет единственное сильное решение для любого начального условия $x_0$.
\end{theorem}

\begin{remark}
Можно сформулировать версию этой теоремы (а также и последующих теорем) для решения на отрезке $[0,T$] "--- нужно лишь потребовать, чтобы неравенства выполнялись для $t\in[0,T]$.
\end{remark}

\begin{example}
По теореме Ито уравнение для геометрического броуновского решения $d S_t = \mu S_t dt + \sigma S_t dW_t$  имеет единственное сильное решение  с любым начальным условием $S_0=s_0$.
Аналогично, единственное сильное решение имеет уравнения для процесса Орнштейна"--~Уленбека $d X_t = \lambda(\theta- X_t) dt + \sigma dW_t$.
\end{example}

\begin{theorem}[Т.~Ямада, C.~Ватанабе]
\label{sde:t:yw-2}
Пусть существуют константа $C$ и функция $h(x)\colon \R_+\to(0,\infty)$ со свойством $\int_0^\epsilon h^{-2}(x) dx = \infty$ для любого $\epsilon >0$, такие, что
\begin{align*}
&\|a(t,x) - a(t,y)\| \le C\|x-y\|,\\
&\|b(t,x) - b(t,y)\| \le h(\|x-y\|).
\end{align*}
Тогда уравнение \eqref{sde:sde} обладает свойством сильной единственности.
\end{theorem}

% \begin{example}
% Рассмотрим одномерное уравнение $d X_t = \kappa(\theta - X_t)dt + \sigma\sqrt{|X_t|}dW_t$.
% Теорема Ито здесь не применима, так как коэффициент диффузии не является липшицевым.
% Однако по теореме Ямады"--~Ватанабе с $h(x) = \sigma\sqrt{x}$ это уравнение обладает свойством сильной единственности решения. В разделе \ref{sde:s:cir} мы покажем, что у него существует слабое решение и тогда, по другой теореме Ямады"--~Ватанабе, на самом деле, существует и единственно сильное решение.
% \end{example}


\subsection{Слабые решения}
\subsubsection{Неоднородные уравнения}

\begin{theorem}[А.\,B.~Скороход]
Пусть функции $a(t,x)$ и $b(t,x)$ ограничены и непрерывны.
Тогда уравнение \eqref{sde:sde} имеет слабое решение.
\end{theorem}

\begin{theorem}[Д.~Струк, C.~Варадан]
Пусть $d=n$. Предположим, что функция $a(t,x)$ ограничена, а $b(t,x)$ непрерывна и удовлетворяет следующему условию: для любых $t\ge 0$, $x\in \R^n$ найдется константа $\epsilon(t,x)>0$ такая, что $\|b(t,x)y\| \ge \epsilon(t,x) \|y\|$ для всех $y\in\R^n$.
Тогда уравнение \eqref{sde:sde} имеет единственное слабое решение.
\end{theorem}

\begin{remark}
Для $d=n=1$ условия теоремы Струка"--~Варадана на функцию $b(t,x)$ означают, что она непрерывна и строго положительна.
\end{remark}


\subsubsection{Одномерные однородные уравнения: нулевой снос}

В этом и следующем разделах мы будем рассматривать одномерное однородное СДУ, \te\ уравнение вида
\begin{equation}
\label{sde:homog}
d X_t = a(X_t) dt + b(X_t) d W_t, \qquad X_0=x_0.
\end{equation}

\begin{theorem}[Х.-Ю.~Энгельберт, В.~Шмидт]
\label{sde:t:es-1}
Пусть $a\equiv 0$. Определим множества
\[
I = \left\{ x \in \R :
  \int_{x-\epsilon}^{x+\epsilon} \frac{1}{b^2(y)} dy = \infty\ \text{для любого}\ \epsilon>0\right\}, \quad
Z = \{x \in \R : b(x) = 0\}.
\]
Тогда
\begin{alphenum}
\item уравнение \eqref{sde:homog} имеет слабое решение для любого начального условия $x_0$ тогда и только тогда, когда $I\subseteq Z$;
\item уравнение \eqref{sde:homog} имеет единственное слабое решение для любого начального условия $x_0$ тогда и только тогда, когда $I=Z$;
\end{alphenum}
\end{theorem}

При нулевом коэффициенте сноса решение уравнения \eqref{sde:homog} всегда является локальным мартингалом (\tk\ стохастический интеграл является локальным мартингалом).
Следующий результат дает критерий мартингальности решения.

\begin{theorem}[А.~Миятович, М.\,А.~Урусов]
\label{sde:t:mu}
Пусть $a\equiv0$, функция $b(x)$ такова что $b(0)=0$, $b(x) > 0$ при $x>0$ и $\int_x^y b^{-2}(z)dz<\infty$ для любых $0<x<y$, и  уравнение \eqref{sde:homog} имеет слабое решение.

Возьмем такое решение, что, если оно достигает 0, то остается в нуле {\normalfont(}\te\ $\P(\exists\, t>s : X_s=0,\ X_t\neq0) = 0)${\normalfont)}.
Тогда оно является мартингалом в том и только том случае, когда
\[
\int_1^\infty \frac{x}{b^2(x)} dx = \infty.
\]
\end{theorem}


\subsubsection{Одномерные однородные уравнения: произвольный снос}

Чтобы сформулировать теорему о достаточных условиях для существования и единственности решения уравнения \eqref{sde:homog} с ненулевым сносом, нам потребуется ввести понятие решения до момента выхода из интервала.

\begin{definition}
Решением уравнения \eqref{sde:homog} в интервале $(l,r)$ называется совокупность $(\Omega,\F,(\F_t)_{t\ge0},\P,W,X)$, где $W$ "--- броуновское движение, определенное на фильтрованном вероятностном пространстве $(\Omega,\F,(\F_t)_{t\ge0},\P)$, а $X$ "--- согласованный непрерывный процесс, для которого найдутся последовательности $l_n\downarrow l$ и $u_n\uparrow u$ такие, что для моментов остановки $\tau_n = \inf\{t\ge0 : X_t \not\in(l_n,u_n)\}$ с вероятностью 1 выполнено неравенство
\[
\int_0^{\tau_n} |a(X_s)|ds + \int_0^{\tau_n} b^2(X_s)ds < \infty,
\]
а также с вероятностью 1 выполнено равенство%
\footnote{Условие <<при $t\le\tau_n$>> в этом равенстве означает, что если умножить обе его части на индикатор $\I(t\le\tau_n)$, то полученное равенство будет выполнено с вероятностью 1.}
\[
X_t^n = x_0 + \int_0^t a(X_s)ds + \int_0^t b(X_s)dW_s\ \text{при $t\le \tau_n$}.
\]
\end{definition}



\begin{theorem}[Х.-Ю.~Энгельберт, В.~Шмидт]
\label{sde:t:es-2}
Для существования единственного слабого решения уравнения \eqref{sde:homog} в интервале $(l,r)\ni x_0$ достаточно выполнения следующих условий:
\begin{alphenum}
\item $b(x)\neq 0$ для всех $x\in(l,r)$;
\item $\int_x^y (1+|a(z)|)b^{-2}(z) dz <\infty$ для любых $x,y$ таких, что $l<x<y<r$.
\end{alphenum}
\end{theorem}

Если решение существует в интервале $(l,r)$ то возникает вопрос, выходит ли оно когда-нибудь из этого интервала (заметим, что если не выходит, то оно будет существовать для всех $t\ge 0$).
Следующая теорема, называемая \emph{критерием взрыва У.~Феллера}%
\footnote{Такое название объясняется случаем $l=-\infty$, $r=+\infty$: выход решения из него за конечное время "--- это выход <<на бесконечность>>, \te\ <<взрыв>>.}
дает необходимое и достаточное условие невыхода из интервала за конечное время.

Будем считать, что для уравнения \eqref{sde:homog} выполнены условие предыдущей теоремы о существовании решения в интервале $(l,r)$.

Пусть $\tau_l = \inf\{t\ge 0: X_t \notin(l,r)\}$ "--- момент первого достижения нижней границы (где $\inf\emptyset = \infty$), и аналогично определим $\tau_r$, а также положим $\tau = \min(\tau_l,\tau_r)$ "--- момент первого выхода из интервала.
Для произвольного $c\in(l,r)$ определим функцию
\[
v(x) = \int_c^x \int_c^y \exp\left(-2\int_z^y \frac{a(u)}{b^2(u)} du \right)\frac{dz}{b^2(z)} dy, \qquad x \in(l,r).
\]

\begin{theorem}[У.~Феллер]
\label{sde:t:feller}
Решение не выходит на нижнюю границу интервала ($\tau_l=\infty$ \as) тогда и только тогда, когда $v(c) \to \infty$ при $c\to l$.
Решение не выходит на верхнюю границу интервала ($\tau_r=\infty$ \as) тогда и только тогда, когда $v(c)\to \infty$ при $c\to r$.
\end{theorem}


\subsection{Строго марковское свойство}

Если уравнение \eqref{sde:sde} имеет единственное слабое решение, то оно является строго марковским процессом.
Чтобы строго сформулировать этот результат, введем дополнительные понятия.

Будем рассматривать уравнение \eqref{sde:sde}, но с начальным условием $X_{t}=x$, заданным в произвольный момент времени $t\ge 0$.
Это уравнение нужно понимать в интегральном виде:
\begin{equation}
\label{sde:sde-markov}
X_u^n = x + \int_{t}^u a^n(s,X_s)ds + \sum_{i=1}^d \int_{t}^u b^{ni}(s,X_s) dW_s^i, \qquad u\ge t.
\end{equation}
Под (слабым) решением этого уравнения понимается фильтрованное вероятностное пространство с определенным на нем броуновским движением и процесс $X=(X_s)_{s\ge t}$, удовлетворяющий уравнению.
Слабая единственность решения понимается как совпадение по распределению любых двух решений.

Рассмотрим пространство $C(\R_+,\R^N)$ непрерывных вектор-функций $\theta_s$, определенных на $\R_+$ и принимающих значения в $\R^N$.
Снабдим его борелевской $\sigma$"=алгеброй\footnote{Борелевская $\sigma$-алгебра порождена всеми открытыми подмножествами $C(\R_+,\R^N)$, а открытые подмножества определяются по метрике $\rho(\theta,\nu) = \sup_{s\ge 0} \|\theta_s-\nu_s\|$.}, что позволит говорить об измеримых отображениях $F\colon C(\R_+,\R^N) \to \R$.
Например, таким отображением является $F(\theta) = \theta_1^i$ ($i$-я координата функции $\theta$ в момент времени $1$); если его применить к траекториям процесса $\{X_{t+s}, s\ge 0\}$, то получим случайную величину $X_{t+1}^i$.

Когда нужно будет говорить о математическом ожидании функций от процесса $X$ и подчеркнуть, что $X$ является решением уравнения \eqref{sde:sde-markov} с начальным условием $X_t=x$, будем использовать запись $\E(\,\cdot\mid X_t=x)$.
Если начальное условие ясно из контекста, то указание начального условия будем опускать.

\begin{theorem}[см.~\cite{KaratzasShreve91}, гл.~5.4.C, теорема 4.20]
\label{sde:t:strong-markov}
Пусть уравнение \eqref{sde:sde-markov} имеет единственное слабое решение для всех $t\ge0$ и $x\in \R^N$, причем коэффициенты $a(t,x)$ и $b(t,x)$ ограничены на любом компакте в $\R_+\times\R^N$.

Для некоторых $t_0\ge0$, $x_0\in\R^N$ рассмотрим решение с начальным условием $X_{t_0}=x_0$, заданное на фильтрованном вероятностном пространстве $(\Omega,\F,\FF,\P)$ с броуновским движением $W$.
Пусть $\tau\ge t_0$ "--- произвольный момент остановки фильтрации $\FF$.
Тогда для любого измеримого отображения $F\colon C(\R_+,\R^N) \to \R$ выполнено \emph{строго марковское свойство}%
\begin{multline}
\label{sde:strong-markov}
\E (F(X_{\tau + s}, s\ge 0) \mid \F_\tau) = \E (F(X_{\tau + s}, s\ge 0) \mid X_\tau) \\
= \E(F(X_{t+s}, s\ge 0) \mid X_t=x) \big|_{t=\tau,x=X_\tau},
\end{multline}
в предположении, что условные математические ожидания определены.
\end{theorem}

Поясним суть равенства \eqref{sde:strong-markov}.
В левой его части стоит случайная величина, измеримая относительно $\F_\tau$.
В середине "--- величина, измеримая относительно $\sigma$-алгебры $\sigma(X_\tau)\subseteq \F_\tau$.
В правой части "--- функция $v(t,x) = \E(F(X_{t+s}, s\ge 0) \mid X_t=x)$, в которую подставляется $\tau$ вместо $t$ и $X_\tau$ вместо $x$.
Строго марковское свойство утверждает, что они равны, и, таким образом, будущие значения процесса $X$ зависят только от текущего значения, но не от прошлых.

\medskip
Приводимое далее следствие будет полезно при вычислении цен платежных обязательств.

\begin{corollary}
В предположениях предыдущей теоремы рассмотрим случайный процесс $V_t = \E(g(X_T) \mid \F_t)$, $t\in[0,T]$, где функция $g\colon\R^N\to\R$ такова, что $\E |g(X_T)| < \infty$.
Тогда $V_t = v(t,X_t)$, где $v(t,x) = \E(g(X_T) \mid X_t=x)$.
\end{corollary}
\begin{proof}
Зафиксируем $t\in[0,T]$ и применим теорему к $F(\theta) = g(\theta_{T-t})$, так что $F(X_{t+s}, s\ge 0) = g(X_T)$.
\end{proof}


\section{Многомерная формула \fc}

Приведем многомерный аналог формулы \fc, которая нам уже встречалась в курсе \intro.

Пусть заданы функции 
\[
a(t,x)\colon \R_+\times \R^n \to \R^n, \quad
b(t,x)\colon \R_+\times \R^n \to \R^{n\times n}, \quad
c(t,x)\colon \R_+\times \R^n \to \R.
\]
Здесь, как и ранее, $x=(x^1,\dots,x^n)\in\R^n$ обозначает $n$-мерную пространственную переменную, а $t\ge 0$ "--- время.

Рассмотрим задачу Коши для параболического уравнения с частными производными вида
\begin{align}
\label{sde:fc-cauchy-1}
&\frac{\partial V}{\partial t}(t,x) + \sum_{i=1}^n a^i(t,x)\frac{\partial V}{\partial x^i}(t,x) + 
\sum_{i,j=1}^n b^{ij}(t,x)\frac{\partial^2 V}{\partial x^i \partial x^j}(t,x)
= c(t,x)V(t,x)\\
\label{sde:fc-cauchy-2}
&V(T,x) = f(x),
\end{align}
где $t\in[0,T]$, $x\in\R^n$.
Формула \fc\ дает \emph{вероятностное представление} решения этой задачи в терминах процесса $X_t$, удовлетворяющего уравнению 
\begin{equation}
\label{sde:fc-process}
dX_t = a(t,X_t) dt + b(t,X_t) dW_t.
\end{equation}

\begin{theorem}[формула Фейнмана--Каца]
Пусть функции $a(t,x)$, $b(t,x)$, $c(t,x)$, $f(x)$ достаточно <<хорошие>>.
Тогда решение задачи  \eqref{sde:fc-cauchy-1}--\eqref{sde:fc-cauchy-2} представимо в виде
\begin{equation}
\label{sde:fc-v}
V(t,x) = \E\left(e^{-\int_t^T c(s,X_s)ds} f(X_T) \;\bigg|\; X_t = x\right),
\end{equation}
и, в частности, это решение единственно в классе функций $C^{1,2}$ на $[0,T)\times \R^n$.
\end{theorem}

\begin{remark}
Достаточные условия <<хорошести>> функций $a,b,c,f$, при которых справедлива формула \fc, следующие (см., например, \cite{KaratzasShreve91}, гл.~5, разд.~5.7.B):
\begin{enumerate}
\item $f$ непрерывна, а также либо $|f(x)| \le L(1+ \|x\|^p)$ для некоторых констант $L,p>0$, либо $f(x)\ge 0$ для всех $x$;
\item $a$ и $b$ непрерывны и удовлетворяют условию линейного роста $\|a(t,x)\| + \|b(t,x)\| \le K(1+|x|)$ с константой $K$ для всех $t,x$;
\item уравнение \eqref{sde:fc-process} имеет единственное слабое решение для всех $t\in[0,T]$ и $x\in\R^n$;
\item задача \eqref{sde:fc-cauchy-1}--\eqref{sde:fc-cauchy-2} имеет решение $V(t,x)$ являющееся функцией класса $C^{1,2}$ на $[0,T)\times\R^n$, причем $|V(t,x)| \le M(1+\|x\|^q)$ с константами $M,q>0$.
\end{enumerate}
\end{remark}


\section{Примеры}
\subsection{Линейное уравнение}

По теореме Ито линейное одномерное уравнение 
\[
dX_t = (\alpha + \beta X_t)dt + (\gamma + \delta X_t) d W_t, \qquad X_0=x_0,
\]
с постоянными коэффициентами %
%\footnote{Для линейных уравнений с коэффициентами, зависящими от времени, получаются, в целом, аналогичные результаты; см.~\cite{GikhmanSkorokhod68}, гл.~II, \S\,5, пример 3.}
$\alpha$, $\beta$, $\gamma$, $\delta$ имеет единственное сильное решение для любого начального условия.
Непосредственно применяя формулу Ито, можно убедиться, что решение задается формулой
\[
X_t = e^{\delta W_t + (\beta-\frac{\delta^2}{2})t}
\left(x_0 + (\alpha - \gamma\delta) \int_0^t e^{-\delta W_s - (\beta-\frac{\delta^2}{2})s} ds
+ \gamma \int_0^t e^{-\delta W_s - (\beta-\frac{\delta^2}{2})s} dW_s\right).
\]


\subsection{Процесс CEV}

Рассмотрим уравнение
\begin{equation}
\label{sde:cev}
d X_t = \sigma |X_t|^\beta d W_t, \qquad X_0=x_0\in \R,
\end{equation}
где $\beta>0$ "--- параметр.
Это уравнение задает процесс цены рискового актива в модели CEV (Constant Elasticity of Variance) в случае, когда безрисковая процентная ставка равна нулю.
Исследуем свойства этого уравнения.

\begin{itemize}
\item По теореме Энгельберта"--~Шмидта (теорема \ref{sde:t:es-1}) существует слабое решение.

\item Для $\beta\ge1/2$ по теореме Ямады"--~Ватанабе (теорема \ref{sde:t:yw-2}) имеет место сильная единственность.
Следовательно, в этом случае по другой теореме Ямады"--~Ватанабе (теорема \ref{sde:t:yw-1}) существует единственное сильное решение.

\item Если же $\beta< 1/2$, то по теореме Энгельберта"--~Шмидта решение не единственно.
Например, если $x_0=0$, то $X_t\equiv0$ является решением, но можно показать, что существует и ненулевое решение (\emph{пример И.\,В.~Гирсанова}, см.~\cite{ChernyEngelbert}, гл.~1, пример 1.22).
% \footnote{
% Пусть $\tilde W$ "--- броуновское движение на вероятностном пространстве $(\Omega,\F,\P)$.
% Положим $A_t = \int_0^t |\tilde W_s|^{-2\beta} ds$, $T_t = \inf\{t\ge 0 : A_s > t\}$ и определим процесс $ X_t = \tilde W_{T_t}$.
% Показывается, что процесс $T_t$ принимает конечные значения, непрерывен и строго возрастает, а также $A_t\to\infty$ \as\ при $t\to\infty$.
% Тогда $X_t$ является непрерывным локальным мартингалом по отношению к фильтрации $\F^{\tilde W}_{T_t}$ с квадратической характеристикой $\qc X_t = T_t$.
% Отсюда следует, что $W_t = \int_0^t|X_s|^{-\beta} dX_s$ является непрерывным локальным мартингалом по отношению к той же фильтрации и имеет квадратическую характеристику $\qc W_t = t$.
% Следовательно, $W$ является броуновским движением по теореме Леви.
% Таким образом, $(\Omega,\F,(\F_{T_t}^{\tilde W})_{t\ge 0}, \P, W, X)$ "--- слабое решение.}).

\item Для функции $v(x)$ из критерия Феллера имеем $v(\infty) = \infty$ при любом $\beta>0$ и $v(0) = \infty$ при $\beta\ge 1$, $v(0)<\infty$ при $\beta <1$.
Следовательно, решение никогда не уходит на бесконечность за конечное время при любом $\beta$.
Решение не достигает нуля при $\beta\ge1$, и достигает нуля при $\beta<1$.
В последнем случае из самого уравнения видно, что можно взять такое решение, что после того, как оно достигает нуля, оно остается в нуле.

\item По теореме Миятовича"--~Урусова (теорема \ref{sde:t:mu}) решение является мартингалом тогда и только тогда, когда $\beta\le 1$ (то решение, которое остается в нуле после его достижения).
\end{itemize}


\subsection{Процесс CIR}

Рассмотрим уравнение
\begin{equation}
\label{sde:cir}
d X_t = \kappa(\theta-X_t) dt + \xi\sqrt{X_t}dW_t, \qquad X_0=x_0>0
\end{equation}
с параметрами $\kappa>0$, $\theta>0$, $\xi>0$.
Это уравнение задает процесс стохастической безрисковой ставки в модели Кокса"--~Ингерсолла"--~Росса (Cox"--~Ingersoll"--~Ross, CIR), а также процесс стохастической дисперсии в модели Хестона.
В теории случайных процессов он также называется процессом квадратного корня.

\begin{itemize}
\item По теореме Энгельберта"--~Шмидта (теорема \ref{sde:t:es-2}) существует слабое решение  в интервале $(0,\infty)$.

\item Для функции $v(x)$ из критерия Феллера нетрудно увидеть, что $v(\infty)=\infty$. 
Таким образом, решение не уходит на бесконечность за конечное время.

\item При $x\to 0$ поведение функции $v(x)$ следующее (где знак $\simeq$ означает асимптотическую эквивалентность):
\begin{multline*}
v(x) = \int_x^1 \int_y^1 e^{2\int_y^z \frac{\kappa(\theta-u)}{\xi^2 u} du} \frac{dz}{\xi^2 z} dz dy
\simeq \int_x^1 \int_y^1 e^{\frac{2\kappa}{\xi^2}(\theta\ln\frac zy - (z-y))} \frac{dz}z dy \\
\simeq \int_x^1 \int_y^1 \left(\frac zy\right)^{\frac{2\kappa\theta}{\xi^2}} \frac {dz}{z} dy
\simeq \int_x^1 \left(y^{-\frac{2\kappa\theta}{\xi^2}} -1\right)dy.
\end{multline*}
Отсюда видно, что $v(0) = \infty$, если $2\kappa\theta\ge\xi^2$, и $v(0)<\infty$ иначе.

\item По критерию Феллера при $2\kappa\theta\ge\xi^2$ решение не достигает нуля и остается всегда положительным.
Из теорем Ямады"--~Ватанабе в этом случае следует существование единственно сильного решения (теорему \ref{sde:t:yw-1} нужно применить с функцией $h(x)=\xi\sqrt{x}$).

\item Если $2\kappa\theta<\xi^2$, то решение достигает нуля за конечное время, при этом можно показать, что оно существует для всех $t\ge 0$, является неотрицательным, а мера Лебега множества моментов времени, в которых решение проводит в нуле, равна 0 \as%
\footnote{Для доказательства существования можно воспользоваться теоремой Скорохода, применяя ее к <<усеченным>> коэффициентам сноса и диффузии, полагая их равными нулю вне отрезка $[0,n]$.
Это даст решение до момента выхода из этого отрезка, а затем нужно устремить $n\to\infty$.
Чтобы доказать неотрицательность, заменим коэффициент диффузии на $\eta\sqrt{X_t^+}$.
Тогда, если предположить, что решение проходит через 0, получим противоречие, так как в отрицательной полуоси оно должно являться строго возрастающей функцией.
Аналогично, решение не может проводить в нуле положительное время.}
\end{itemize}


\summary
В теории стохастических дифференциальных уравнений существуют понятия сильного и слабого решения, которые отличаются тем, что в сильном решении неизвестный процесс $X$ измерим относительно фильтрации, порожденной броуновским движением, а в слабом решении этого не требуется.
Решение называется единственным в сильном смысле, если любые два решения, построенные с одним и тем же броуновским движением, совпадают потраекторно \as\ 
Решение называется единственном с слабом смысле, если любые два решения совпадают по распределению.

В лекции приведены следующие теоремы.
\begin{itemize}
\item Теорема Ито: достаточное условие сильного существования и сильной единственности.
\item Теорема Ямады"--~Ватанабе: сильная единственность и слабое существование влекут сильное существование.
\item Теорема Ямады"--~Ватанабе: достаточное условие сильной единственности.
\item Теорема Скорохода: достаточное условие слабого существования.
\item Теорема Струка"--~Варадана: достаточное условие слабого существования и слабой единственности.
\item Теорема Энгельберта"--~Шмидта (нулевой снос): необходимое и достаточное условие слабого существования и слабой единственности для одномерного однородного уравнения;
\item Теорема Энгельберта"--~Шмидта (произвольный снос): достаточное условие слабого существования и слабой единственности решения интервале.
\item Теорема Миятовича"--~Урусова: критерий мартингальности решения.
\item Теорема Феллера: критерий выхода из интервала.
\item Формула \fc: вероятностное представление решения задачи Коши для параболического уравнения с частными производными.
\end{itemize}
