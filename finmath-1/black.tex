%!TEX root=finmath1.tex
\chapter{Модель Блэка для рынка фьючерсов}
\label{ch:black}

Здесь более подробно излагается модель Блэка для рынка фьючерсов (она кратко упоминалась в замечании \ref{bs2:r:black} в лекции \ref{ch:bs2}), в которой моделируется непосредственно цена фьючерса, без введения в рассмотрение базового актива.

\section{Описание модели}
В модели имеются два актива "--- безрисковый актив с детерминированной процентной ставкой и рисковый актив, являющийся фьючерсом с моментом экспирации $T'\ge T$.
Клиринг происходит <<непрерывно>> во времени.

Основные объекты и понятия, вводимые далее, будут такими же, как в модели \bs, но определение стоимости портфеля будет иметь другой вид: оно будет учитывать, что открыть и закрыть позицию по фьючерсу в любой момент можно без издержек (см.~пояснение в разделе \ref{fut:ss:assets} лекции \ref{ch:futures-discrete}).

Итак, пусть задано фильтрованное вероятностное пространство $(\Omega,\F,\FF,\P)$ с фильтрацией $\FF=(\F_t)_{t\in[0,T]}$, которая порождена броуновским движением $W=(W_t)_{t\in[0,T]}$ и пополнена.

Цены безрискового актива и фьючерса имеют вид
\[
d B_t = r(t) B_t dt,\quad B_0=1, \qquad d F_t = \mu F_t dt + \sigma F_t dW_t, \quad F_0 = f_0>0.
\]
Функция $r(t)$ считается измеримой и ограниченной.

Торговой стратегией называется процесс $\pi_t=(G_t,H_t)$, где $G\in \PP_T^1$, $H\in\PP_T^2$.
Стоимость портфеля стратегии $\pi$ по определению равна
\begin{equation}
\label{bl:portfolio-value}
V_t^\pi = G_tB_t
\end{equation}
(заметим, что стоимость фьючерсной позиции равна нулю). 
Дисконтированная стоимость портфеля $\tilde V_t^\pi = V_t^\pi/B_t$ совпадает с процессом $G_t$.

Стратегия называется самофинансируемой, если
\begin{equation}
\label{bl:sf}
d V_t^\pi = G_t d B_t + H_t dF_t.
\end{equation}

\begin{proposition}
В модели Блэка существует единственная вероятностная мера $\Q\sim\P$ такая, что процесс $F$ является мартингалом относительно $\Q$:
\[
d F_t = \sigma F_t d W_t^{\Q},
\] 
где $W_t^Q$ "--- броуновское движение относительно $\Q$. 

Кроме того, процесс $G$ любой самофинансируемой стратегии (и, следовательно, дисконтированная стоимость ее портфеля) является локальным мартингалом относительно $\Q$.
При этом $G$ является квадратично интегрируемым мартингалом, если $HF\in \L^2_T(\Q)$.
\end{proposition}

\begin{proof}
Кратко перечислим лишь основные шаги доказательства, так как оно идейно повторяет доказательства аналогичных результатов в модели \bs.

Мера $\Q$ строится по теореме Гирсанова.
Если $\pi$ "--- самофинансируемая стратегия, то, применяя формулу Ито к \eqref{bl:sf}, показывается, что дисконтированная стоимость $\tilde V_t^\pi = V_t^\pi/B_t\;(=G_t)$ удовлетворяет уравнению
\[
d\tilde V_t^\pi = \frac{H_t}{B_t} dF_t = \frac{\sigma H_t F_t}{B_t}dW_t^{\Q},
\]
откуда следует, что это локальный мартингал.
Если выполнено условие $HF \in \L^2_T(\Q)$, то он квадратично интегрируем по свойству интеграла Ито.
\end{proof}
\begin{remark}
Полученная цена фьючерса имеет точно такой же вид относительно ЭММ, как в разделе \ref{bs2:ss:futures} лекции \ref{ch:bs2}.
Зачем же нам тогда понадобилось рассматривать новую модель?

Отличие между моделью там и здесь состоит в том, что теперь мы не предполагаем, что на рынке торгуется сам базовый актив, с которым связан фьючерс, и, как следствие, не требуем, чтобы дисконтированная цена базового актива являлась мартингалом относительно ЭММ. 
%
% В качестве примера можно привести фьючерсы на индекс волатильности VIX\footnote{Индекс VIX вычисляется по подразумеваемой волатильности опционов на индекс S\&P\,500, исполняющихся через месяц, и показывает <<насколько волатилен>> сейчас рынок.
% Более подробно мы его коснемся в курсе <<Модели стохастической волатильности>>.}. 
% Этот индекс не является торгуемым инструментом, однако торгуются фьючерсы на VIX.
% В частности, в безарбитражной модели процесс значения индекса VIX (он сам или дисконтированный) не обязан быть мартингалом, а цена фьючерса обязана.
\end{remark}


\section{Цены платежных обязательств. Формула Блэка}

\subsection{Премиальные платежные обязательства}
Будем отождествлять платежные обязательства с $\F_T$"=измеримыми случайными величинами $X$ такими, что $\E^\Q X^2 < \infty$.
В этом разделе мы рассмотрим премиальные платежные обязательства, по которым, как обычно, покупатель платит продавцу премию (цену обязательства) в момент $t$, а получает выплату $X$ в момент $T$.
В следующем разделе будут рассмотрены маржируемые контракты.

\begin{proposition}
В модели Блэка любое квадратично интегрируемое платежное обязательство $X$ является реплицируемым, причем реплицирующая стратегия единственна.
Цена репликации (стоимость портфеля реплицирующей стратегии) равна
\[
V_t^X = B(t,T)\E^\Q (X \mid \F_t),
\]
где $B(t,T) = e^{-\int_t^T r(s)ds}$ "--- цена бескупонной облигации.
\end{proposition}

В качестве частного случая рассмотрим европейские опционы колл и пут на фьючерс "--- это контракты, дающие право покупателю в будущий момент времени $T$ открыть длинную (опцион колл) или короткую (опцион пут) фьючерсную позицию по фиксированной цене $K$.
Считается, что экспирация фьючерса происходит не раньше экспирации опционов.
Эти опционы можно отождествить с выплатами, производимыми в момент $T$ и равными
\[
X^\text{call} = (F_T-K)^+, \qquad X^\text{put} = (K-F_T)^+,
\]
где $F_T$ "--- цена фьючерса с временем экспирации $T'\ge T$. 

\begin{corollary}[формула Блэка]
\label{bl:c:black-formula}
Цены опционов на фьючерс в модели Блэка имеют вид $V_t=V(t,F_t)$, где
\[
\begin{aligned}
&\VC(t,f) = B(t,T) (f\Phi(d_1) - K\Phi(d_2)), \\
&\VP(t,s) = B(t,T) (K\Phi(-d_2) - f\Phi(-d_1)),\\
&d_1 = \frac{1}{\sigma\sqrt{T-t}} \left(\ln\frac{f}{K} + \frac{\sigma^2}{2}(T-t)\right), \quad
d_2 = d_1 - \sigma\sqrt{T-t}.
\end{aligned}
\]
\end{corollary}

Отметим, что полученная формула имеет такой же вид, как формула Блэка из лекции \ref{ch:bs2}, но там $f$ выражает $T$-форвардную цену базового актива, а здесь "--- цену фьючерса (они совпадают в модели \bs, если $T=T'$).
Величина $T'$ не входит в формулу.


\subsection{Маржируемые контракты}

Для маржируемого контракта в модели Блэка величина $X$ задает расчетную цену в момент $T$.
Ценой репликации контракта называется такой процесс $V^X = (V_t^X)_{t\in[0,T]}$, что существует стратегия $\pi_t=(G_t,H_t)$ с компонентами $G\equiv 0$ и $H\in \PP_T^2$ (не являющаяся в общем случае самофинансируемой), удовлетворяющая условию
\[
d V_t^X = H_t d F_t.
\]
Это равенство означает, что изменение стоимости портфеля стратегии в точности равно вариационной марже.
Поясним, что условие $G_t\equiv 0$ возникает по той причине, что стоимость портфеля реплицирующей стратегии должна быть нулевой (так как позицию по маржируемому контракту можно открывать и закрывать ьез издержек), а стоимость портфеля определяется формулой \eqref{bl:portfolio-value}.
Таким образом, реплицирующая стратегия торгует только фьючерсом.

\begin{proposition}
В модели Блэка реплицирующая стратегия любого маржируемого контракта $X$ единственна $($в классе стратегий с $HF\in \L^2_T(\Q)$$)$, а цена репликации равна
\[
V_t^X = \E^\Q (X \mid \F_t).
\]
\end{proposition}

\begin{corollary}[формула Блэка для маржируемых опционов]
Цены маржируемых опционов колл и пут $($\te\ маржируемых контрактов с расчетными ценами $X^\text{call} = (F_T - K)^+$ и $X^\text{put} = (K-S_T)^+$$)$ имеют вид $V_t = V(t,F_t)$, где
\[
\VC(t,f) = f\Phi(d_1) - K\Phi(d_2), \qquad
\VP(t,s) = K\Phi(-d_2) - f\Phi(-d_1),
\]
а величина $d$ такая, же как в следствии \ref{bl:c:black-formula}.
\end{corollary}
