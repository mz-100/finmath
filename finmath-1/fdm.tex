%!TEX root=finmath1.tex
\part{Численные методы}
\chapter{Метод конечных разностей}
\label{ch:fdm}
\chaptertoc

Как было показано в предыдущих лекциях, цены европейских платежных обязательств удовлетворяют уравнению \bs\ с частными производными.
В этой лекции содержится базовое введение в метод конечных разностей для численного решения уравнений с частными производными применительно к модели \bs.


\section{Обзор метода конечных разностей}
\subsection{Три схемы аппроксимации уравнения с частными производными}

На прямоугольнике $\Pi =\{(t,x): 0\le t \le T,\ x_\text{min} \le x \le x_\text{max}\}$ рассмотрим уравнение с частными производными параболического типа%
\footnote{Мы рассматриваем только параболические уравнения, так как именно они возникают в задачах оценки производных инструментов.}
\begin{align}[left=\empheqlbrace]
\label{13:pde}
&v'_t(t,x) + a(t,x) v'_x(t,x) + b(t,x) v''_{xx}(t,x) = 0, \\
\label{13:initial}
&v(T,x) = f(x), \\
\label{13:boundary-lower}
&v(t,x_\text{min}) = g(t),\\
\label{13:boundary-upper}
&v(t,x_\text{max}) = h(t).
\end{align}
Решением этого уравнения будем называть функцию $v(t,x)$, непрерывную на $\Pi$, дифференцируемую по $t$ и дважды дифференцируемую по $x$ на внутренности $\Pi$, а также удовлетворяющую уравнению \eqref{13:pde} внутри $\Pi$, терминальному условию \eqref{13:initial} для $x\in[x_\text{min},x_\text{max}]$ и граничным условиям \eqref{13:boundary-lower} и \eqref{13:boundary-upper} для $t\in[0,T)$.

Далее будет предполагаться, что уравнение \eqref{13:pde}--\eqref{13:boundary-upper} имеет единственное решение, причем оно является достаточно гладким (дифференцируемым столько раз, сколько нам потребуется). 
Отметим, что в уравнении \eqref{13:pde}--\eqref{13:boundary-upper} задано терминальное условие $v(T,x) = f(x)$, а не начальное (как в физических задачах).

\begin{remark}
Дальнейшие рассуждения остаются справедливыми для уравнения \eqref{13:pde} с дополнительными слагаемыми $c(t,x)v(t,x) + d(t,x)$ в левой части.
\end{remark}

\emph{Метод конечных разностей} для численного решения уравнения \eqref{13:pde}--\eqref{13:boundary-upper} заключается в замене производных функции $v(t,x)$ на разности ее значений в близких точках.

Зададим на прямоугольнике $\Pi$ равномерную сетку $\Omega_{m,n} = \{(t_i, x_j)\}$, разбив его стороны на $m$ и $n$ отрезков соответственно%
\footnote{Т.е.\ по оси $t$ возьмем точки $t_i = i \Delta t$, где $i=0,\dots,m$ и $\Delta t = T/m$, а по оси $x$ возьмем точки $x_j = x_\text{min} + j \Delta x$, где $j = 0,\ldots,n$ и $\Delta x = (x_\text{max} - x_\text{min})/n$.}, и будем искать значения $v_{i,j} = v(t_i, x_j)$ в узлах сетки.
В зависимости от того, как аппроксимировать производные, будут получаться разные системы уравнений для неизвестных $v_{i,j}$.

Далее рассмотрим три основных схемы аппроксимации.
Все они основаны на идее, что для первой и второй производных достаточно гладкой функции $u(z)$ при $\Delta z \to 0$ выполнены равенства (как следует из  формулы Тейлора)
\begin{align*}
&u'(z) = \frac{u(z + \Delta z) - u(z)}{\Delta z} + O(\Delta z)\ \text{или}\ 
u'(z) = \frac{u(z + \Delta z) - u(z - \Delta z)}{2\Delta z} + O(\Delta z^2),\\
&u''(z) = \frac{u(z + \Delta z) - 2u(z) + u(z - \Delta z)}{\Delta z^2} + O(\Delta z^2).
\end{align*}


\subsubsection{Явная схема}
Будем использовать аппроксимации
\begin{align}
\label{13:explicit-t}
&v'_t(t_i, x_j) \approx \frac{v_{i,j} - v_{i-1,j}}{\Delta t},\\
\label{13:explicit-x}
&v'_x(t_i, x_j) \approx \frac{v_{i,j+1} - v_{i,j-1}}{2\Delta x},\\
\label{13:explicit-xx}
&v''_{xx}(t_i, x_j) \approx \frac{v_{i,j+1} - 2v_{i,j} + v_{i,j-1}}{\Delta x^2}.
\end{align}
Тогда уравнение \eqref{13:pde} примет вид
\begin{equation}
\label{13:pde-explicit}
\frac{v_{i,j} - v_{i-1,j}}{\Delta t} + a_{i,j} \frac{v_{i,j+1} - v_{i,j-1}}{2\Delta x} + b_{i,j} \frac{v_{i,j+1} - 2v_{i,j} + v_{i,j-1}}{\Delta x^2}=0,
\end{equation}
где $a_{i,j} = a(t_i, x_j)$, $b_{i,j} = b(t_i, x_j)$ и $i=1,\ldots,m$, $j=1,\ldots,n-1$.
Терминальное условие \eqref{13:initial} и граничные условия \eqref{13:boundary-lower}--\eqref{13:boundary-upper} дают соотношения
\begin{align}
\label{13:initial-explicit}
&v_{m,j} = f_j,\quad j=0,\ldots,n,\\
\label{13:boundary-explicit}
&v_{i,0} = g_i,\quad v_{i,n} = h_i,\quad i=0,\ldots,m-1,
\end{align}
где $f_j = f(x_j)$, $g_i = g(t_i)$, $h_i = h(t_i)$. 

Система уравнений \eqref{13:pde-explicit}--\eqref{13:boundary-explicit} легко решается по слоям по времени.
Действительно, значения $v_{m,j}$ известны из условия \eqref{13:initial-explicit}.
Далее, если определены значения $v_{i,j}$ на слое $i$, то значения $v_{i-1,j}$ на предыдущем слое для $j=1,\dots,n-1$ находятся из уравнения \eqref{13:pde-explicit} по формуле
\[
v_{i-1,j} = v_{i,j} + \Delta t \left( a_{i,j} \frac{v_{i,j+1} - v_{i,j-1}}{2\Delta x} + b_{i,j} \frac{v_{i,j+1} - 2v_{i,j} + v_{i,j-1}}{\Delta x^2}\right),
\]
а значения $v_{i-1,0}$ и $v_{i-1,n}$ определяются из условий \eqref{13:boundary-explicit}.
В силу того, что имеется явная формула для $v_{i-1,j}$, рассматриваемая схема и называется явной.


\subsubsection*{Неявная схема}
В этой схеме для производной по времени используется аппроксимация
\[
v'_t(t_i, x_j) \approx \frac{v_{i+1,j} - v_{i,j}}{\Delta t},\\
\]
а для производной по пространству та же аппроксимация \eqref{13:explicit-x}--\eqref{13:explicit-xx}, что и в явной схеме.
Тогда уравнение \eqref{13:pde} принимает вид
\[
\frac{v_{i+1,j} - v_{i,j}}{\Delta t} + a_{i,j} \frac{v_{i,j+1} - v_{i,j-1}}{2\Delta x} + b_{i,j} \frac{v_{i,j+1} - 2v_{i,j} + v_{i,j-1}}{\Delta x^2} = 0,\\
\]
где $i=0,\ldots,m-1$, $j=1,\ldots,n-1$.
Удобно сдвинуть индекс $i$ на 1 влево и переписать это уравнение так, чтобы первое слагаемое было таким же как в явной схеме:
\begin{equation}
\label{13:pde-implicit}
\frac{v_{i,j} - v_{i-1,j}}{\Delta t} + a_{i-1,j} \frac{v_{i-1,j+1} - v_{i-1,j-1}}{2\Delta x} + b_{i-1,j} \frac{v_{i-1,j+1} - 2v_{i-1,j} + v_{i-1,j-1}}{\Delta x^2} = 0,
\end{equation}
где теперь $i=1,\ldots,m$, $j=1,\ldots,n-1$.
Начальные и граничные условия по-прежнему задаются формулами \eqref{13:initial-explicit}--\eqref{13:boundary-explicit}.

Значения $v_{i-1,j}$ здесь тоже можно находить последовательно по слоям.
Однако, в отличие от явной схемы, если уже найдены значения на слое $i$, то для того, чтобы найти значения на слое $i-1$, придется решать систему линейных уравнений, так как в левую часть \eqref{13:pde-implicit} входят неизвестные $v_{i-1,j-1}$, $v_{i-1,j}$, $v_{i-1,j+1}$. 

Обозначив вектор $V = (v_{i-1,0},\dots,v_{i-1,n})$, с учетом граничных условий получаем линейную систему $UV = W$ c трехдиагональной матрицей
\[
U = \begin{pmatrix}
u_{0,0} & 0       & 0       & 0       & \dots  & 0      & 0       & 0\\ 
u_{0,1} & u_{1,1} & u_{1,2} & 0       & \dots  & 0      & 0       & 0\\
0       & u_{2,1} & u_{2,2} & u_{2,2} & \dots  & 0      & 0       & 0\\
\vdots  & \vdots  & \vdots  &  \vdots & \ddots & \vdots & \vdots & \vdots  \\
0       & 0       & 0       & 0       & \dots  & u_{n-1,n-2} & u_{n-1,n-1} & u_{n-1,n}\\
0       & 0       & 0       & 0       & \dots  & 0           & 0           & u_{n,n}
\end{pmatrix}
\]
с элементами
\begin{align*}
&u_{0,0} = u_{n,n} = 1,\\
&u_{j,j+1} = \left(-\frac{a_{i-1,j}}{2\Delta x} - \frac{b_{i-1,j}}{\Delta x^2}\right)\Delta t,& 
  &j=1,\dots,n-1,\\
&u_{j,j} = 1 + 2\frac{b_{i-1,j}}{\Delta x^2}\Delta t, & &j=1,\dots,n-1,\\
&u_{j,j-1} = \left(\frac{a_{i-1,j}}{2\Delta x} - \frac{b_{i-1,j}}{\Delta x^2}\right)\Delta t,& 
  &j=1,\dots,n-1,
\end{align*}
и правой частью $W=(w_0,\dots,w_n)$, где
\[
w_0 = g_{i-1},\qquad
w_j = v_{i,j}, \ j=1,\dots,n-1,\qquad
w_n = h_{i-1}.
\]
Для линейной системы с трехдиагональной матрицей существует эффективный метод решения (\emph{метод прогонки}), который позволяют найти значения $v_{i-1,j}$ на очередном слое за время $O(n)$. 

\subsubsection*{Схема \cn}
Эта схема является комбинацией явной и неявной схем и использует полусумму уравнений \eqref{13:pde-explicit} и \eqref{13:pde-implicit}, что приводит к уравнению
\begin{multline}
\label{13:pde-cn}
\frac{v_{i,j} - v_{i-1,j}}{\Delta t} + \frac12 \Biggl(
  a_{i,j} \frac{v_{i,j+1} - v_{i,j-1}}{2\Delta x} + b_{i,j} \frac{v_{i,j+1} - 2v_{i,j} + v_{i,j-1}}{\Delta x^2} \\
  + a_{i-1,j} \frac{v_{i-1,j+1} - v_{i-1,j-1}}{2\Delta x} + b_{i-1,j} \frac{v_{i-1,j+1} - 2v_{i-1,j} + v_{i-1,j-1}}{\Delta x^2} \Biggr) = 0.
\end{multline}
Для значений $v_{i+1,j}$ опять получается трехдиагональная система  $UV = W$, где матрица $U$ имеет элементы
\begin{align*}
&u_{0,0} = u_{n,n} = 1,\\
&u_{j,j+1} = \left(-\frac{a_{i-1,j}}{4\Delta x} - \frac{b_{i-1,j}}{2\Delta x^2}\right)\Delta t, & &j=1,\dots,n-1,\\
&u_{j,j} = 1 + \frac{b_{i-1,j}}{\Delta x^2}\Delta t, & &j=1,\dots,n-1,\\
&u_{j,j-1} = \left( \frac{a_{i-1,j}}{4\Delta x} - \frac{b_{i-1,j}}{2\Delta x^2}\right)\Delta t,& &j=1,\dots,n-1,
\end{align*}
а правая часть $W$ имеет элементы
\begin{align*}
&w_0 = g_{i-1},\\
&w_j =  v_{i,j} + \left( a_{i,j} \frac{v_{i,j+1} - v_{i,j-1}}{4\Delta x} + b_{i,j} \frac{v_{i,j+1} - 2v_{i,j} + v_{i,j-1}}{2\Delta x^2} \right) \Delta t,\\
&w_{N} = h_{i-1}.
\end{align*}


\subsection{Сходимость разностных схем}

Пусть $v_{i,j}$ обозначают значения, полученные с помощью какой-либо разностной схемы  на сетке $\Omega_{m,n}$, а $v(t,x)$ обозначает настоящее решение.
Под \emph{сходимостью} разностной схемы понимают сходимость $v_{i,j} \to v(t,x)$, когда $t_i\to t$ и $x_j\to x$ при $\Delta t,\Delta x \to 0$.
В этом разделе мы приведем основные определения и результаты, связанные со сходимостью разностных схем для рассматриваемого уравнения с частными производными на нестрогом уровне и без углубления в детали.
Подробности можно найти, например, в гл.~10 книги \cite{BakhvalovZhidkovKobelkov}.

По \emph{теореме Лакса--Рябенького--Филиппова} сходимость следует из свойств аппроксимации и устойчивости.
Дадим соответствующие определения.

Будем использовать обозначение $L = \prt{}t + a(t,x)\prt{}x + b(t,x)\prtt{}x$ для дифференциального оператора в левой части уравнения \eqref{13:pde}, и обозначение $L_\delta$ для разностного оператора, соответствующего некоторой разностной схеме с шагом $\delta=(\Delta t, \Delta x)$.
Например, для явной схемы 
\begin{multline*}
L_\delta u(t,x) = \frac{u(t,x) - u(t-\Delta t, x)}{\Delta t} + a(t,x) \frac{u(t,x+\Delta x) - u(t,x-\Delta x)}{2\Delta x} \\
+ b(t,x) \frac{u(t,x+\Delta x) - 2u(t,x) + u(t,x-\Delta x)}{\Delta x^2},
\end{multline*}
и аналогичные выражения имеются для неявной схемы и схемы \cn.

Говорят, что разностная схема \emph{аппроксимирует} уравнение \eqref{13:pde} с порядком $p$ по времени и порядком $q$ по пространству, если для любой достаточно гладкой функции $u$ верно, что  $\|L u - L_h u\| = O((\Delta t)^p + (\Delta x)^q)$ при $\Delta t, \Delta x \to 0$ c какой-либо выбранной нормой $\|\cdot\|$ (часто используют норму $L^2$ или $L^\infty$).

Разностная схема \emph{устойчива}, если найдутся константы $c,\epsilon_t,\epsilon_x$ такие, что любой  функции $u$, заданной в узлах сетки с шагом $\delta=(\Delta t, \Delta x)$, где $\Delta t < \epsilon_t$, $\Delta x < \epsilon_x$, выполнено неравенство
\[
\|L_\delta u\| \le c \|u\|.
\]

Для рассмотренных выше разностных схем известны следующие факты.
Явная схема имеет порядок аппроксимации 1 по времени и 2 по пространству и является условно-устойчивой, т.е.\ устойчива, только если $\Delta t$ достаточно мало по сравнению с $(\Delta x)^2$ (в случае постоянного коэффициента $b$ должно быть выполнено неравенство $\Delta t < (\Delta x)^2 / (2b)$).
В частности, для уменьшения шага по времени требуется квадратичное уменьшение шага по пространству, что часто делает использование этой схемы неэффективным.
Если условие устойчивости не выполнено, то схема будет давать решение с ошибкой, увеличивающейся  при $\Delta t,\Delta x\to 0$.

Неявная схема и схема \cn\ являются безусловно устойчивыми. При этом неявная схема имеет порядок аппроксимации 1 по времени и 2 по пространству, а схема \cn\ имеет порядок аппроксимации 2 и по времени, и по пространству.

Из этого следует, что, по крайней мере для задач с гладкими функциями $f,g,h$, схема \cn\ является предпочтительной.
Однако функции выплат опционов, как правило, не являются гладкими (например, это так даже в случае обычных опционов колл и пут -- функции $(s-K)^+$ и $(K-s)^+$ имеют разрыв производной в точке $s=K$), что может приводить к ошибкам при использовании схемы \cn; см.~примеры вычисления гаммы в практическом занятии.


\section{Примеры оценки опционов}
\subsection{Ванильные опционы}

Чтобы продемонстрировать работу разностных схем на простом примере, применим их к задаче вычисления цен европейских опционов колл и пут и сравним получающиеся ответы с формулой \bs.
Далее будем рассматривать только опционы колл (для опционов пут все аналогично).

Пусть безрисковая процентная ставка задается функцией $r(t)$, а дивидендная доходность функцией $q(t)$.
Без ограничения общности будем считать, что исходная вероятностная мера уже является мартингальной, и опускать верхний индекс $\Q$ у знака математического ожидания.
Пусть $B(t,T) = e^{-\int_t^T r(s) ds}$ обозначает коэффициент дисконтирования (цену бескупонной облигации).

Напомним, что цена опциона колл $V(t,s) = B(t,T) \E((S_T-K)^+ \mid S_t = s)$ удовлетворяет уравнению
\begin{align*}[left=\empheqlbrace]
&V'_t(t,s) + (r(t)-q(t))sV'_s(t,s) + \frac{\sigma^2}{2} s^2 V''_{ss}(t,s) = r(t)V(t,s),\\
&V(T,s) = (s-K)^+.
\end{align*}
Удобно сделать замену координат и вместо цены рискового актива $S_t$ работать с ее с логарифмом $X_t=\ln S_t$.
Тогда $V(t,s) = B(t,T) v(t, \ln s)$, где
$v(t,x) = \E ((e^{X_T} - K)^+ \mid X_t = x)$, причем функция $v(t,x)$ удовлетворяет уравнению%
\footnote{Это уравнение можно получить непосредственными преобразованиями, используя равенство $\prt{}s = s^{-1}\prt{}x$, или применить формулу Фейнмана--Каца к функции $v(t,x)$ с учетом того, что по формуле Ито $d X_t = (r(t)-q(t) - \sigma^2/2) dt + \sigma d W_t$.}
\begin{align}[left=\empheqlbrace]
\label{13:log-pde}
&v'_t(t,x) + a(t)v'_x(t,x) + b v''_{xx}(t,x) = 0,\\
\label{13:log-initial}
&v(T,x) = (e^x - K)^+
\end{align}
с коэффициентами $a(t) = r(t) - q(t) - \sigma^2/2$ и $b = \sigma^2/2$.
То обстоятельство, что теперь коэффициент $b$ константный, повысит точность разностной схемы.

Чтобы применить разностную схему, необходимо еще добавить граничные условия.
Для опциона колл воспользуемся тем, что $V(t,s) \to 0$ при $s\to 0$ и $V(t,s) \to s - B(t,T)K$ при $s\to\infty$ (см.~лекцию~\ref{ch:bs}), и, следовательно, $v(t,x) \to 0$ при $x\to -\infty$ и $v(t,x) \to e^x/B(t,T) - K$ при $x\to\infty$.
Таким образом, можно взять малое значение $x_\text{min}$, большое значение $x_\text{max}$, и положить
\begin{equation}
\label{13:log-boundary}
v(t,x_\text{min}) = 0,\quad v(t,x_\text{max}) = \frac{e^x}{B(t,T)} - K.
\end{equation}
В качестве ``малого'' и ``большого'' подойдут такие значения, что процесс $X_t$ достигнет их с малыми вероятностями, например $x_\text{min} = X_0-5\sigma\sqrt{T}$ и $x_\text{max} = X_0 + 5\sigma\sqrt{T}$.

Теперь уравнение \eqref{13:log-pde}--\eqref{13:log-boundary} готово к применению метода конечных разностей.
Примеры представлены в практикуме.


\subsection{Барьерные опционы}

\emph{Барьерные опционы} дают право купить (опцион колл) или продать (опцион пут) базовый актив в фиксированный момент времени в будущем по фиксированной цене при условии, что до момента исполнения цена пересечет (опцион типа \emph{вход}) или не пересечет (опцион типа \emph{выход}) некоторый уровень, называемый \emph{барьером}. 

Более подробно, пусть $K$ -- заданная цена страйк, а $H$ -- барьер.
Тогда определены следующие разновидности барьерных опционов колл.
\begin{itemize}
\item \emph{Вверх-и-выход} (\emph{up-and-out}): опцион можно исполнить, только если цена базового актива все время находилась ниже барьера.
\item \emph{Вверх-и-вход} (\emph{up-and-in}): опцион можно исполнить, только если цена базового актива хотя бы раз достигла или поднялась выше барьера.
\item \emph{Вниз-и-выход} (\emph{down-and-out}): опцион можно исполнить, только если цена базового актива все время находилась выше барьера.
\item \emph{Вниз-и-вход} (\emph{down-and-in}): опцион можно исполнить, только если цена базового актива хотя бы раз достигла или опустилась ниже барьера.
\end{itemize}
Случайные величины, задающие выплаты по опционам колл, имеют вид
\begin{align*}
&X^\text{call,u-o} = (S_T-K)^+\I(\max_{u\le t} S_u < H)&   &(\text{вверх-и-выход}),\\
&X^\text{call,u-i} = (S_T-K)^+\I(\max_{u\le t} S_u \ge H)& &(\text{вверх-и-вход}),\\
&X^\text{call,d-o} = (S_T-K)^+\I(\min_{u\le t} S_u > H)&   &(\text{вниз-и-выход}),\\
&X^\text{call,d-i} = (S_T-K)^+\I(\min_{u\le t} S_u \le H)& &(\text{вниз-и-вход}).
\end{align*}

\begin{remark}
Для случая постоянной безрисковой процентной ставки и дивидендной доходности известны аналитические формулы для цен барьерных опционов (см., например, \cite{Haug06}).
Разностные схемы, однако, позволяют учитывать произвольные функции $r(t)$ и $q(t)$.
\end{remark}

Чтобы получить уравнение с частными производными для цен барьерных опционов, применим следующее обобщение формулы Фейнмана--Каца.
Рассмотрим <<достаточно хорошее>> замкнутое множество  $D \subseteq \R_+\times \R$ с  границей $\partial D$. 
Пусть стохастическое дифференциальное уравнение
\[
d X_t = \mu(t,X_t) dt + \sigma(t,X_t) d W_t
\]
имеет единственное решение для любого начального условия $X_t=x$, где $(t,x)\in D$, причем решение является марковским процессом (достаточно потребовать выполнения условий теоремы Ито о существовании и единственности решения).
Рассмотрим функцию
\[
V(t,x) = \E (f(\tau_D, X_{\tau_D}) \mid X_t = x), \qquad (t,x) \in D,
\]
где $\tau_D = \inf\{s\ge t : X_s \not\in D\}$ -- момент первого выхода из множества $D$. 
Тогда, если функции $\nu(t,x)$, $\sigma(t,x)$ и $f(t,x)$ достаточно хорошие, то $V(t,x)$ удовлетворяет уравнению
\begin{align*}[left=\empheqlbrace]
&V'_t(t,x) + \mu(t,x)V'_x(t,x) + \frac12 \sigma^2(t,x) V''_{xx}(t,x) = 0, \qquad (t,x)\in\mathrm{int} D,\\
&V(t,x) = f(t, x), \qquad (t,x)\in \partial D,\ t>0.
\end{align*}
\begin{remark}
Обычная формула Фейнмана--Каца получается отсюда при $D=[0,T]\times \R_+$.
\end{remark}

Для примера рассмотрим барьерный опцион колл вверх-и-выход.
Пусть $V(t,s)$ обозначает его цену в момент времени $t$ при цене базового актива $S_t=s<H$ при условии, что до момента $t$ цена базового актива не достигала барьера.
Тогда 
\[
\begin{split}
V(t,s) &= B(t,T) \E((S_T-K)^+\I(\max_{t\le u \le T} S_u < H \mid S_t = s) \\
&= B(t,T) \E(f(\tau_D, S_{\tau_D}) \mid S_t=s)
\end{split}
\]
с функцией 
\[
f(t,s) = \begin{cases}
0, &t<T,\\
(s-K)^+\I(s<H), &t=T.
\end{cases}
\]
и множеством $D = [0,T]\times[0, H]$.
Отсюда следует, что цена такого барьерного опциона удовлетворяет уравнению
\begin{align*}[left=\empheqlbrace]
&V'_t(t,s) + (r(t)-q(t))sV'_s(t,s) + \frac{\sigma^2}{2} s^2 V''_{ss}(t,s) = r(t)V(t,s), \\ &\qquad\qquad t\in(0,T),\ s\in(0,H),\\
&V(T,s) = (s-K)^+, \quad s\in (0, H)\\
&V(t,H) = 0, \quad t\in[0,T].
\end{align*}

Как и для ванильных опционов, сделаем замену переменных и перейдем к логарифму цены.
Тогда можно представить $V(t,s) = B(t,T) v(t, \ln s)$, где функция $v(t,x)$ задана на $[0,T]\times(-\infty, \ln H]$ и удовлетворяет уравнению
\begin{align*}[left=\empheqlbrace]
&v'_t(t,x) + a(t)v'_x(t,x) + b v''_{xx}(t,x)=0,\\
&v(T,x) = (e^x-K)^+\I(x<\ln H),\\
&v(t, \ln H) = 0.
\end{align*}
с функциями $a(t) = r(t) - q(t) - \sigma^2/2$ и $b = \sigma^2/2$.
Для применения разностных схем необходимо еще добавить граничное условие на нижней границе, которое аналогично ванильному опциону можно взять $v(t, x_\text{min}) = 0$ для малого значения $x_\text{min}$. 

Цена опциона колл вниз-и-выход вычисляется аналогично.
Вычисление цен опционов типа вход можно свести к уже рассмотренным задачам, если воспользоваться \emph{паритетом вход-выход}, согласно которому цены опционов с одинаковым временем исполнения, страйком и барьером связаны соотношениями
\begin{align*}
V_\text{call,u-o}(t,s) + V_\text{call,u-i}(t,s) = V_\text{call}(t,s),\\
V_\text{call,d-o}(t,s) + V_\text{call,d-i}(t,s) = V_\text{call}(t,s).
\end{align*}
Справедливость этих соотношений нетрудно увидеть из следующих равенств (для опционов вверх; для опционов вниз аналогично):
\begin{align*}
V^\text{call,u-o}(t,s) + V^\text{call,u-i}(t,s) 
&= B(t,T) \E((S_T-K)^+\I(\max_{t\le u \le T} S_u < H) \mid S_t = s)\\
  &+  B(t,T) \E((S_T-K)^+\I(\max_{t\le u \le T} S_u \ge H) \mid S_t = s)\\
&= B(t,T) \E((S_T-K)^+ \mid S_t = s) = V^\text{call}(t,s).
\end{align*}


\summary
\begin{itemize}
\item Уравнение \bs\ с частными производными для цены европейского платежного обязательства можно решать численно. Метод конечных разностей состоит в замене производных на разности функции в близких точках, что сводит уравнение с частными производными к системе разностных уравнений. 
\item В этой лекции рассмотрены три разностные схемы: явная, неявная и схема \cn. Их основные характеристики приведены в таблице.

\medskip
\begin{tabular}{|l|p{2cm}|p{2cm}|c|}
\hline
\multirow{2}{*}{Схема} & \multicolumn{2}{|c|}{Порядок аппроксимации} & \multirow{2}{*}{Устойчивость}\\ \cline{2-3}
                       & по $t$ & по $x$       & \\
\hline
Явная & 1 & 2 & Условная \\
Неявная & 1 & 2 & Да \\
\cn & 2 & 2 & Да \\
\hline
\end{tabular}
\end{itemize}
