%!TEX root=finmath1.tex
\chapter{Американские опционы}
\label{ch:american-discrete}
\chaptertoc

Американский опцион дает право покупателю исполнить его в любой момент времени до экспирации (в отличие от европейского опциона, который может быть исполнен только в момент экспирации).
Из-за этой особенности задача оценки американского опциона является гораздо более сложной, чем для европейского опциона. 

Сначала мы рассмотрим эту задачу с точки зрения продавца опциона и найдем минимальную цену, по которой он будет готов продать опцион.
Затем рассмотрим задачу с точки зрения покупателя и покажем, что цены, справедливые для продавца и для покупателя, на полном рынке совпадают.

В этой лекции изучается только случай полного рынка и предполагается, что базовый акив не выплачивает дивиденды.
Об оценке американских опционов на неполном рынке, см., например, гл.~6 в книге \cite{FollmerSchied11}.


\section{Справедливая цена для продавца}

Будем рассматривать полную модель рынка, заданную на вероятностном пространстве $(\Omega,\F,\P)$ с фильтрацией $\FF=(\F_t)_{t=0}^T$.
Пусть $\Q$ обозначает единственную эквивалентную мартингальную меру.

\begin{definition}
Платежное обязательство \emph{американского типа} отождествляется с последовательностью случайных величин $X = (X_t)_{t=0}^T$, где $X_t$ представляет выплату покупателю от продавца обязательства в случае, когда покупатель решает исполнить обязательство в момент времени $t$.
\end{definition}

\begin{example}
Для американского опциона колл $X_t = (S_t - K)^+$, где $K$ "--- цена страйк.
Для американского опциона пут $X_t = (K - S_t)^+$.
\end{example}

\begin{remark}
Покупатель может исполнить американское платежное обязательство в случайный момент времени (например, принимая во внимание движение цены базового актива).
О том, как правильно определить класс допустимых случайных моментов, написано в следующем разделе.
Продавец не может выбирать момент исполнения обязательства.
\end{remark}

\begin{definition}
\emph{Хеджирующей стратегией} американского платежного обязательства $X$ называется самофинансируемая стратегия $\pi$ такая, что $V_t^\pi \ge X_t$ для всех $t=0,\dots,T$.

Минимальной справедливой ценой для продавца американского платежного обязательства $X$ в начальный момент времени (далее "--- \emph{ценой продавца}) называется число
\[
v_0^{\text{sell}} = \inf\{V_0^\pi \mid\ \text{$\pi$ хеджирует $X$}\}.
\]
\end{definition}

Смысл определения цены продавца состоит в том, что продав обязательство по цене $v_0^\text{sell}$ (далее мы покажем, что инфимум в определении достигается), он сможет расплатиться с покупателем независимо от того, в какой момент времени покупатель решит предъявить платежное обязательство к исполнению.

\begin{theorem}
\label{5:seller-price}
Определим последовательность случайных величин $V_t$ по обратной индукции следующим образом.
Положим $V_T = X$ и, если $V_t$ определено для некоторого $t\in\{1,\dots,T\}$, положим 
\begin{equation}
\label{am:seller-price}
V_{t-1} = \max\left(X_{t-1},\ \frac{B_{t-1}}{B_t} \E^{\Q}(V_t\mid\F_{t-1})\right).
\end{equation}
Тогда $V_0$ является ценой продавца платежного обязательства $X$.
\end{theorem}

\begin{remark}
Формальное доказательство, приводимое ниже, несколько громоздко, но суть формулы \eqref{am:seller-price} проста и состоит в следующем.

Во-первых, стоимость портфеля стратегии продавца в момент времени $t-1$ должна быть не меньше чем $X_{t-1}$, на случай, если покупатель решит исполнить опцион в этот момент.
Это первый аргумент максимума в формуле \eqref{am:seller-price}.

Во-вторых, если покупатель решит не исполнять опцион, то продавец будет действовать так, чтобы получить на следующем шаге портфель $\pi_{t+1}$, который позволит ему хеджировать опцион далее (такой портфель уже найден по предположению обратной индукции).
Этот портфель стоит $V_t$ в момент времени $t$. 
Следовательно, стоимость портфеля стратегии продавца в момент $t-1$ должна быть не меньше, чем безарбитражная цена портфеля $\pi_{t+1}$ в момент $t$, а это есть второй аргумент максимума в формуле \eqref{am:seller-price}.
\end{remark}

\begin{proof}[Доказательство \difficult]
Сначала покажем, что $v_0^\text{sell}\ge V_0$, для чего достаточно доказать, что $V_0^\pi\ge V_0$ для любой хеджирующей стратегии $\pi$.
Пусть $\pi$ "--- хеджирующая стратегия для $X$. Тогда $V_t^\pi \ge X_t$ по определению хеджирующей стратегии, а также 
\[
V_{t-1}^\pi = \frac{B_{t-1}}{B_t} \E^{\Q}(V_t^\pi \mid \F_t)
\]
в силу того, что дисконтированная стоимость любой самофинансируемой стратегии является мартингалом%
\footnote{Вообще говоря, $V^\pi$ является мартингальным преобразованием, но, так как рынок полон, то вероятностное пространство в сущности дискретно согласно расширенному варианту второй фундаментальной теоремы, и, следовательно, любое мартингальное преобразование является мартингалом.}
относительно ЭММ.
Отсюда по обратной индукции легко показать, что $V_t^\pi \ge V_t$.

Покажем теперь, что существует хеджирующая стратегия с $V_0^\pi = v_0^\text{sell}$, откуда будет следовать, что $v_0^\text{sell} = V_0$.
Мы приведем не вполне строгое рассуждение, но из этой идеи должно быть понятно, как сделать доказательство строгим.

В силу полноты модели найдется $\F_{T-1}$-измеримый портфель $\pi_T=(G_T,H_T)$, реплицирующий выплату $V_T = X_T$ в момент $T$, \te\  $V_T^\pi = G_TB_T + \scal{H_T}{S_T} = V_T$.
Его стоимость в момент времени $T-1$ равна $W_{T-1} = B_{T-1}\E^{\Q}(V_T/B_T\mid \F_{T-1})$. 

Опять используя полноту, можно найти $\F_{T-2}$-измеримый портфель $\pi_{T-1}$, который реплицирует выплату $V_{T-1} = \max(X_{T-1}, W_{T-1})$ в момент $T-1$, \te\ $V_{T-1}^\pi = V_{T-1}$.
Так как $V_{T-1} \ge W_{T-1}$, то в момент $T-1$ портфель $\pi_{T-1}$ можно преобразовать в портфель $\pi_T$ и, возможно, останется некоторая сумма денег, которую вложим в безрисковый актив. 
Полученный портфель в следующий момент времени будет покрывать (реплицировать с избытком) величину $X_T$. 

Продолжим рассуждения по индукции.
Пусть построен портфель $\pi_t$, который имеет стоимость $W_{t-1} = B_{t-1}\E^{\Q}(V_t/B_t\mid \F_{t-1})$ в момент времени $t-1$ и покрывает выплаты $X_t,\dots,X_T$. Тогда в момент $t-2$ можно найти портфель $\pi_{t-1}$, который реплицирует выплату $V_{t-1}$.
В момент $t-1$ его можно преобразовать в $\pi_t$, возможно, с избытком.
Далее с помощью портфеля $\pi_t$ покрываются выплаты $X_s$ для $s\ge t$.

Таким образом будет построена последовательность портфелей $\pi_1,\dots,\pi_T$, которые покрывают выплаты $X_1,\dots,X_T$, причем каждый портфель в этой последовательности можно преобразовать в следующий. 
Стоимость начального портфеля равна $W_0 = \E^{\Q}(V_1/B_1)$.
Таким образом, если иметь в начальный момент времени сумму денег $V_0 = \max(X_0, W_0)$, то можно покрыть и $X_0$.
\end{proof}

\begin{example}
Рассмотрим двухшаговую модель \crr\ с параметрами $S_0=100$, $u=0.1$, $d=-0.1$, $r=0.05$ и найдем в ней стоимость американского опциона пут со страйком $K=102$.

Процесс решения изображен на рис.~\ref{5:fig-put-price}.
Сначала составим таблицы значений цены $S_t$ и выплаты $X_t = (K-S_t)^+$ для всех возможных случайных исходов $\omega$ и моментов времени $t=0,1,2$ (рис.~\ref{5:fig-put-price-0}, \ref{5:fig-put-price-1}).
Для $t=2$, положим $V_2 = X_2$ (рис.~\ref{5:fig-put-price-2}). 

Найдем $V_1$ (рис. \ref{5:fig-put-price-3}).
В рассматриваемой модели риск-нейтральная вероятность $q=3/4$, поэтому $V_1 = \max(X_1, (1+r)^{-1}\E^Q(V_2\mid \F_1))$:
\[
V_1(\omega) = 
\begin{cases}
\max(0,\ \tfrac{1}{1.05}(0\cdot\tfrac34 + 3\cdot\tfrac14)) = 0.71,
  &\omega=\omega_1,\,\omega_2,\\
\max(12,\ \tfrac{1}{1.05}(3\cdot\tfrac34 + 21\cdot\tfrac14)) = 12,
  &\omega=\omega_3,\,\omega_4.
\end{cases}
\]
Наконец, найдем $V_0$ (рис. \ref{5:fig-put-price-4}): из формулы $V_0 = \max((K-S_0)^+, (1+r)^{-1}\E^Q V_1)$ получаем
\[
V_0 = \max(2,\ \tfrac{1}{1.05} (0.71\cdot\tfrac34 + 12\cdot\tfrac14)) = 3.36.
\]

\begin{figure}[h]
\centering
\begin{subfigure}{0.45\textwidth}
  \centering
  \begin{tabular}{c|ccc}
    $S_t$ & $t=0$ & $t=1$ & $t=2$\\\hline
    $\omega_1$ & 100 &110 &121\\
    $\omega_2$ & 100 &110 &99\\
    $\omega_3$ & 100 &90 &99\\
    $\omega_4$ & 100 &90 &81\\
  \end{tabular}
  \caption{Цена акции.}
  \label{5:fig-put-price-0}
\end{subfigure}
%
\begin{subfigure}{0.45\textwidth}
  \centering
  \begin{tabular}{c|ccc}
    $X_t$ & $t=0$ & $t=1$ & $t=2$\\\hline
    $\omega_1$ & 2 &0 &0\\
    $\omega_2$ & 2 &0 &3\\
    $\omega_3$ & 2 &12 &3\\
    $\omega_4$ & 2 &12 &21\\
  \end{tabular}
  \caption{Выплата по опциону.}
  \label{5:fig-put-price-1}
\end{subfigure}
\par\vspace{5mm}\par
\begin{subfigure}{0.45\textwidth}
  \centering
  \begin{tabular}{c|ccc}
    $V_t$ & $t=0$ & $t=1$ & $t=2$\\\hline
    $\omega_1$ & ? &? &0\\
    $\omega_2$ & ? &? &3\\
    $\omega_3$ & ? &? &3\\
    $\omega_4$ & ? &? &21\\
  \end{tabular}
  \caption{Цена опциона в момент $t=2$.}
  \label{5:fig-put-price-2}
\end{subfigure}
%
\begin{subfigure}{0.45\textwidth}
  \centering
  \begin{tabular}{c|ccc}
    $V_t$ & $t=0$ & $t=1$ & $t=2$\\\hline
    $\omega_1$ & ? &0.71 &0\\
    $\omega_2$ & ? &0.71 &3\\
    $\omega_3$ & ? &12 &3\\
    $\omega_4$ & ? &12 &21\\
  \end{tabular}
  \caption{Цена опциона в моменты $t=1,2$.}
  \label{5:fig-put-price-3}
\end{subfigure}
\par\vspace{5mm}\par
\begin{subfigure}{0.45\textwidth}
  \centering
  \begin{tabular}{c|ccc}
    $V_t$ & $t=0$ & $t=1$ & $t=2$\\\hline
    $\omega_1$ & 3.36 &0.71 &0\\
    $\omega_2$ & 3.36 &0.71 &3\\
    $\omega_3$ & 3.36 &12 &3\\
    $\omega_4$ & 3.36 &12 &21\\
  \end{tabular}
  \caption{Цена опциона в моменты $t=0,1,2$.}
  \label{5:fig-put-price-4}
\end{subfigure}
\caption{Нахождение цены американского опциона пут.}
\label{5:fig-put-price}
\end{figure}
\end{example}


\section{Справедливая цена для покупателя}

В этом разделе мы рассмотрим задачу оценки американского платежного обязательства с точки зрения покупателя.
Если покупатель решает исполнить его в (случайный) момент времени $\tau$, то можно считать, что он покупает корзину европейских платежных обязательств $X_t'$, где $X_t'$ исполняется в момент времени $t$ и имеет выплату $X_t' = X_t \I(\tau=t)$.
Стоимость каждого такого обязательства в момент $t=0$ есть $\E^{\Q}(X_t'/B_t)$, а поэтому стоимость всей корзины равна
\[
\sum_{t=0}^T \E^{\Q} \frac{X_t'}{B_t} = \sum_{t=0}^T \E^{\Q} \frac{X_t \I(\tau=t)}{B_t}
= \E^{\Q} \frac{X_\tau}{B_\tau}.
\]
Тогда целью покупателя является нахождение момента исполнения $\tau$, который бы максимизировал стоимость корзины.

Чтобы решить такую задачу нам понадобится понятие момента остановки и некоторые общие сведения из теории задач об оптимальной остановке.


\subsection{Моменты остановки}

Пусть задано фильтрованное вероятностное пространство $(\Omega,\F,\FF,\P)$ с фильтрацией $\FF=(\F_t)_{t=0}^T$.
Далее будет рассматриваться только случай конечного горизонта времени $T$.

\begin{definition}
Случайная величина $\tau$, принимающая значения в множестве $\{0,1,\dots,T\}$, называется \emph{моментом остановки} по отношению к фильтрации $\FF$, если%
\footnote{Запись $\{\tau=t\}$ означает событие $\{\omega: \tau(\omega) = t\}$.
Как обычно, $\omega$ для краткости опускается.}
$\{\tau= t\} \in \F_t$ для любого $0\le t \le T$.
\end{definition}

Смысл этого определения состоит в том, что решение <<остановиться в момент $t$>> (например, исполнить опцион) принимается на основе информации до момента $t$ включительно, но не используя будущую информацию.
Подчеркнем, что для разных случайных исходов $\omega$ остановка может происходить в разные моменты времени.

Следующее предложение содержит некоторые простые свойства моментов остановки.
Доказательство можно найти в книге \cite{Shiryaev04}, гл.~7, \S\,1.

\begin{proposition}
Справедливы следующие утверждения.
\begin{alphenum}
\item Случайная величина $\tau$ со значениями $\{0,\dots,T\}$ является моментом остановки тогда и только тогда, когда $\{\tau\le t\} \in \F_t$ для любого $t=0,\dots,T$.

\item Любой неслучайный момент $\tau\equiv t$ является моментом остановки.

\item Если $\tau, \sigma$ являются моментами остановки, то моментами остановки также являются величины $\max(\tau,\sigma)$, $\min(\tau,\sigma)$, $\min(\tau+\sigma,T)$.
\end{alphenum}
\end{proposition}

\begin{remark}
Величина $\tau-\sigma$ моментом остановки, вообще говоря, не является.
\end{remark}

Следующее предложение дает пример большого класса моментов остановки.

\begin{proposition}
\label{5:hitting-time}
Пусть $X_t$ является согласованной последовательностью, а $A_t\in \B(\R)$ является последовательностью борелевских множеств.
Тогда моментом остановки является момент первого попадания $X$ в $A$:
\[
\tau = \min\{t : X_t \in A_t\},
\]
где по определению $\min\emptyset = T$.
\end{proposition}

Как отсюда следует, моментами остановки будут, например, момент первого достижения некоторого уровня $\tau=\min\{t : X_t = x\}$ или момент первого выхода из интервала $\tau=\min\{t : X_t\notin (a,b)\}$. 

\begin{proof}
Заметим, что
\[
\{\tau = t\} = \{X_t \in B_t\} \cap \bigcup_{s=0}^{t-1} \{X_s \notin B_s\}.
\]
События в правой части принадлежат $\F_t$, а поэтому $\{\tau=t\}\in\F_t$.
\end{proof}


\subsection{Задачи об оптимальной остановке}

Пусть $X=(X_0)_{t=0}^T$ "--- согласованная последовательность, причем $\E|X_t| < \infty$ для всех $t=0,\dots,T$.

\begin{definition}
\emph{Задача об оптимальной остановке} для последовательности $X$ состоит в нахождении момента остановки $\tau^*$ на котором достигается
\begin{equation}
\label{5:os-problem}
V = \sup_\tau \E X_\tau
\end{equation}
и вычислении величины $V$.
Момент $\tau^*$ называется \emph{оптимальным моментом остановки} (он может быть не единственным), а величина $V$ "--- \emph{ценой} в задаче об оптимальной остановке.
\end{definition}

\begin{remark}
Запись $X_\tau$ означает случайную величину $X_{\tau(\omega)}(\omega)$.
\end{remark}

Следующее определение и теорема дают решение задачи об оптимальной остановке.
Доказательство можно найти в книге \cite{Shiryaev04}, гл.~VII, \S\,13 .

\begin{definition}
\emph{Огибающей Снелла} последовательности $X$ называется согласованная последовательность $U=(U_t)_{t=0}^T$, определяемая по обратной индукции следующим образом: $U_T=X_T$ и 
\[
U_{t-1} = \max (X_{t-1}, \E(U_t\mid \F_{t-1})).
\]
\end{definition}

\begin{theorem}
\label{5:os-solution}
Решением задачи об оптимальной остановке \eqref{5:os-problem} является момент остановки%
\footnote{То, что так определенный момент $\tau^*$ является моментом остановки, следует из предложения~\ref{5:hitting-time}: это момент первого достижения нуля согласованной последовательностью $U_t-V_t$.}
\[
\tau^* = \min\{t : U_t = X_t\},
\]
где $U_t$ "--- огибающая Снелла последовательности $X_t$.
При этом цена в задаче об оптимальной остановке $V = U_0$.
\end{theorem}

Интуитивное объяснение этого результата состоит в следующем. 
Огибающая Снелла равна максимальной выгоде, которую можно получить, если не останавливаться до момента $t$: можно либо остановиться в момент $t$ и получить $X_t$, либо продолжить наблюдение до следующего момента.
В случае продолжения ожидаемое значение получаемой выплаты равно условному ожиданию огибающей Снелла в следующий момент времени.
Формула для $\tau^*$ показывает, что нужно остановиться сразу, как только выигрыш от продолжения наблюдения становится равным выигрышу от остановки.


\subsection{Цена покупателя американского платежного обязательства}

Будем рассматривать американские платежные обязательства $X$, у которые все выплаты $X_t$ неотрицательны (как, например, у опционов колл и пут).
\begin{definition}
\emph{Моментом исполнения} американского платежного обязательства $X$ может быть произвольный момент остановки $\tau \le T$. 

\emph{Оптимальным моментом исполнения} называется момент остановки $\tau^*$ (возможно, не единственный) на котором достигается 
\[
v_0^\text{buy} = \max_\tau \E^{\Q} \frac{X_\tau}{B_\tau}.
\]
Величина $v_0^\text{buy}$ называется максимально справедливой ценой для покупателя американского платежного обязательства $X$ (или просто \emph{ценой покупателя}).
\end{definition}

Таким образом, покупатель решает задачу об оптимальной остановке для последовательности дисконтированной выплаты платежного обязательства $X$ (с ожиданием по ЭММ $\Q$).

\begin{remark}
Так как мы рассматриваем полный рынок, то вероятностное пространство можно считать конечным, и, следовательно, существует лишь конечное число различных моментов остановки $\tau$.
Поэтому супремум в общей постановке задачи об оптимальной постановке можно заменить на максимум.
\end{remark}

\begin{remark}
Приведенное выше определение предполагает, что покупатель обязательно исполнит платежное обязательство до момента $T$ (нет возможности не предъявлять его к исполнению).
Для неотрицательных $X_t$ это не снижает общности рассуждений.
\end{remark}

Рассмотрим американское платежное обязательство $X$ и определим для него последовательность $V_t$ как в теореме \ref{5:seller-price}.
Нетрудно видеть, что последовательность $U_t = V_t/B_t$ является огибающей Снелла последовательности $X_t/B_t$.
Тогда из теоремы~\ref{5:os-solution} вытекает следующий результат.

\begin{theorem}
Оптимальным моментом исполнения американского платежного обязательства $X$ является момент
$\tau^* = \min\{t : V_t = X_t\}$, при этом цена покупателя $v_0^\text{buy}=V_0$ и, следовательно, совпадает с ценой продавца. 
\end{theorem}

\begin{example}
Рассмотрим американский опцион из примера в предыдущем разделе и найдем оптимальный момент исполнения.
Из сопоставления таблиц для величин $V_t$ и $X_t$ видно (см.~рис.~\ref{5:fig-tau}), что первый момент времени $\tau^*$, когда они совпадают, имеет вид
\[
\tau^*(\omega_1) = \tau^*(\omega_2) = 2, \qquad
\tau^*(\omega_3) = \tau^*(\omega_4) = 1.
\]

\begin{figure}[h]
\centering
\begin{tabular}{c|ccc}
    $X_t$ & $t=0$ & $t=1$ & $t=2$\\\hline
    $\omega_1$ & 2 &0 &0\\
    $\omega_2$ & 2 &0 &3\\
    $\omega_3$ & 2 &12 &3\\
    $\omega_4$ & 2 &12 &21\\
\end{tabular}
\qquad
\begin{tabular}{c|ccc}
    $V_t$ & $t=0$ & $t=1$ & $t=2$\\\hline
    $\omega_1$ & 3.36 &0.71 &\fbox{0}\\
    $\omega_2$ & 3.36 &0.71 &\fbox{3}\\
    $\omega_3$ & 3.36 &\fbox{12} &3\\
    $\omega_4$ & 3.36 &\fbox{12} &21\\
\end{tabular}
\caption{Оптимальный момент исполнения американского опциона пут.}
\label{5:fig-tau}
\end{figure}
\end{example}


\section{О совпадении цен европейских и американских опционов}

\begin{theorem}
Пусть цена безрискового актива не убывает, т.е.\ $B_t\ge B_{t-1}$ для всех $t=1,\dots,T$.
Тогда для любого страйка $K$ имеет место равенство
\[
\max_{\tau} \E^{\Q} \frac{(S_\tau - K)^+}{B_\tau} = \E^{\Q} \frac{(S_T - K)^+}{B_T},
\]
и, в частности, цена американского опциона колл совпадает с ценой европейского опциона колл с таким же страйком.

Если цена безрискового актива не возрастает, то аналогичное утверждение верно для опционов пут.
Если цена безрискового актива постоянна, то цены американских и европейских опционов совпадают как для колл, так и для пут.
\end{theorem}

\begin{remark}
Неубывающая последовательность $B_t$ означает, что безрисковая ставка неотрицательна, что обычно и предполагается.
Случай невозрастающей последовательности $B_t$ соответствует отрицательной ставке, что в реальности бывает крайне редко.
Постоянная последовательность $B_t$ соответствует нулевой ставке%
\end{remark}

\begin{remark}
Отметим, что теорема верна в предположении, что рисковый актив не выплачивает дивиденды.
При наличии дивидендов цены американских и европейских опционов (как колл, так и пут) будут, вообще говоря, различными.
\end{remark}

Мы приведем два доказательства этой теоремы.
Первое короче, но опирается на ряд результатов из теории мартингалов;
эти результаты можно найти в гл.~VII книги \cite{Shiryaev04}.
Второе более длинное, но не использует никаких дополнительных фактов.

\begin{proof}[Первое доказательство]
Так как последовательность $S_t/B_t$ является мартингалом относительно $\Q$, а $B_t$ не убывает, то $(S_t-K)/B_t$ является субмартингалом%
\footnote{Последовательность $Y=(Y_t)_{t=0}^T$ называется \emph{субмартингалом} относительно фильтрации $\FF$, если (a) она согласована с фильтрацией; (b) $\E|Y_t| <\infty$ для всех $t$; (c) $\E^{\Q}(Y_t\mid\F_s) \ge Y_s$ для всех $s\le t$.
Таким образом, различие в определении мартингала и субмартингала заключается в том, что вместо равенства в условии (c) стоит неравенство.}.
Функция $x^+$ выпукла и не убывает, следовательно $Y_t = (S_t-K)^+/B_t$ тоже является субмартингалом по неравенству Йенсена.
Тогда по теореме Дуба об остановке%
\footnote{Если $X$ "--- мартингал, то для любых моментов остановки $\tau\le\sigma\le T$ выполнено $\E X_\tau = \E X_\sigma$. Если $Y$ -- субмартингал, то $\E Y_\tau \le \E Y_\sigma$.}
$\E Y_\tau \le \E Y_T$ для любого момента остановки $\tau\le T$.
\end{proof}

\begin{proof}[Второе доказательство]
Заметим, что для любого $t$ имеем
\begin{multline*}
\E^Q \biggl(\frac{(S_T-K)^+}{B_T} \;\bigg|\; \F_t\biggr) \stackrel{(a)}{\ge}
\biggl(\E^Q \biggl(\frac{S_T-K}{B_T} \;\bigg|\; \F_t\biggr)\biggr)^+ \\ \stackrel{(b)}{=}
\biggl(\frac{S_t}{B_t} - \E^Q\biggl(\frac{K}{B_T}\;\bigg|\; \F_t
\biggr)\biggr)^+ \stackrel{(c)}{\ge}
\biggl(\frac{S_t}{B_t} - \frac{K}{B_t}\biggr)^+ = \frac{(S_t-K)^+}{B_t},
\end{multline*}
где неравенство $(a)$ следует из неравенства Йенсена, равенство $(b)$ следует из мартингального свойства, а $(3)$ выполнено так как $B_t$ не убывает.
Используя телескопическое свойство, отсюда получаем, что
\[
\E^Q \biggl(\frac{(S_t-K)^+}{B_t} \I(\tau=t)\biggr) \le
\E^Q \biggl(\frac{(S_T-K)^+}{B_T} \I(\tau=t)\biggr).
\]
Тогда для любого момента остановки $\tau$ имеем
\begin{multline*}
\E^Q \frac{(S_\tau-K)^+}{B_\tau}
= \sum_{t=0}^T \E^Q \biggl(\frac{(S_t-K)^+}{B_t}\I(\tau=t)\biggr) \\
\le \sum_{t=0}^T \E^Q \biggl(\frac{(S_T-K)^+}{B_T}\I(\tau=t)\biggr)
= \E^Q \frac{(S_T-K)^+}{B_T}.
\end{multline*}
Отсюда следует, что
\[
\max_{\tau} \E^{\Q} \frac{(S_\tau - K)^+}{B_\tau} \le \E^{\Q} \frac{(S_T - K)^+}{B_T},
\]
но при этом ясно, что неравенство на самом деле является равенством, так как $\tau=T$ тоже является моментом остановки.
\end{proof}


\summary

\begin{itemize}
\item Справедливая цена продавца американского платежного обязательства определяется из условия, что соответствующей суммы денег хватит, чтобы построить портфель, который покрывает выплаты по обязательству независимо от того, когда покупатель решит его исполнить.
Справедливая цена для покупателя определяется из выбора оптимального момента исполнения платежного обязательства (из решения задачи об оптимальной остановке).

\item На полном рынке справедливые цены для покупателя и для продавца американских платежных обязательств совпадают.
Эти цены можно найти методом обратной индукции. 

\item Оптимальным моментом исполнения является первый момент времени, когда справедливая цена становится равна выплате по обязательству.

\item Если процентная ставка неотрицательна и рисковый актив не выплачивает дивиденды, то цены американских и европейских опционов колл совпадают.
\end{itemize}
