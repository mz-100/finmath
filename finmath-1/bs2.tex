%!TEX root=finmath1.tex
\chapter[Обобщенная модель \bs. Формула Блэка]{Обобщенная модель \bs.\\Формула Блэка}
\label{ch:bs2}
\chaptertoc

В этой лекции мы обобщим модель Блэка--Шоулза: добавим дивиденды и учтем, что безрисковая процентная ставка может быть переменной.
Такая модель, в частности, позволяет рассматривать не только рынки акций, но и валютные рынки.

Также мы получим \emph{формулу Блэка} "--- альтернативное представление формулы \bs, которое более удобно для практического использования.


\section{Обобщение модели \bs}
\subsection{Активы и торговые стратегии}

Будем считать, что рынок задан на фильтрованном вероятностном пространстве $(\Omega, \F, \FF, \P)$ и состоит из безрискового и рискового активов.
Фильтрация порождена броуновским движением $W=(W_t)_{t\in[0,T]}$ и пополнена, $\F = \F_T$.

Цена безрискового актива $B=(B_t)_{t\in[0,T]}$ является неслучайной и задается формулой
\[
B_t = e^{\int_0^t r(u) du},
\]
где $r(t)$ "--- неслучайная, но зависящая от времени, безрисковая процентная ставка.
Будем далее предполагать, что функция $r(t)$ ограничена на $[0,T]$ (и, естественно, измерима).
В дифференциальной форме можно записать
\[
dB_t = r(t) B_t dt, \qquad B_0=1.
\]
Процесс цены рискового актива $S=(S_t)_{t\in[0,T]}$, как и в обычной модели \bs, является геометрическим броуновским движением:
\[
d S_t = \mu S_t dt + \sigma S_t d W_t, \qquad S_0=s_0 >0.
\]
Однако теперь с рисковым активом будет связана еще функция $q(t)$, задающая ставку дивидендной доходности.
Считается, что за промежуток $[t,t+dt]$ этот актив выплачивает дивиденды в размере $q(t)S_t dt$ (строгое понятие будет дано далее в формуле \eqref{10:sf}).
Будем предполагать, что функция $q(t)$ измерима, неотрицательна и ограничена.

\begin{remark}
В реальности дивиденды выплачиваются дискретно "--- например, раз в год, раз в полгода или раз в квартал.
Модель с дискретными дивидендами более трудна; мы ее не рассматриваем.
\end{remark}

\begin{definition}
\emph{Торговой стратегией} в обобщенной модели \bs\ называется пара измеримых согласованных процессов $\pi=(G,H)$, где $G=(G_t)_{t\in[0,T]} \in \PP^1_T$ выражает количество единиц безрискового актива в портфеле, а $H=(H_t)_{t\in[0,T]} \in \PP^2_T$ количество единиц рискового актива. 
\emph{Стоимостью портфеля} стратегии $\pi$ называется процесс 
\[
V_t^\pi = G_t B_t + H_t S_t.
\]
Стратегия называется \emph{самофинансируемой}, если 
\begin{equation}
\label{10:sf}
d V_t^\pi = G_t d B_t + H_t d S_t + q(t) H_t S_t dt.
\end{equation}
\end{definition}

По сравнению с условием самофинансируемости в обычной модели \bs, новым здесь является последнее слагаемое.
Оно выражает размер дивидендов, полученных за <<бесконечно малый>> промежуток времени $dt$.
Можно заметить, что формула \eqref{10:sf} аналогична условию самофинансируемости в модели с дивидендами в дискретном времени (см.\ раздел \ref{gen:s:dividends} в лекции \ref{ch:general}).

\begin{definition}
\emph{Дисконтированной ценой рискового актива с учетом дивидендов} называется процесс 
\[
\tilde S_t = e^{\int_0^t q(u) du} \frac{S_t}{B_t} = e^{\int_0^t (q(u)-r(u)) du} S_t.
\]
\emph{Дисконтированной стоимостью портфеля} стратегии $\pi$ называется процесс 
\[
\tilde V_t^\pi = \frac{V_t^\pi}{B_t} = e^{-\int_0^t r(u) du} V_t^\pi.
\]
\end{definition}

Интерпретация процесс $\tilde S$ здесь такая же, как была в дискретном времени: он представляет собой дисконтированную стоимость портфеля самофинансируемой стратегии $\pi_t=(G_t,H_t)$, которая в начальный момент времени покупает 1 единицу рискового актива и реинвестирует в него дивиденды, \te\ имеет компоненты $G_t\equiv 0$, $H_t = e^{\int_0^t q(u) du}$.

\begin{proposition}
\label{10:p:sf}
Стратегия $\pi$ является самофинансируемой тогда и только тогда, когда
\begin{equation}
\label{10:sf-discounted}
d\tilde V_t^\pi = e^{-\int_0^t q(s)ds} H_t d \tilde S_t.
\end{equation}
\end{proposition}

\begin{proof}
Пусть $\pi$ является самофинансируемой стратегией.
Введем для удобства функцию $D_t = 1/B_t = e^{-\int_0^t r(u)du}$.
Нетрудно видеть, что $dD_t = -r(t) D_t dt$.
Тогда из формулы Ито и условия самофинансируемости находим
\begin{align*}
d \tilde V_t^\pi &= d(D_t V_t^\pi) = D_t(-r(t) V_t^\pi dt + dV_t^\pi) \\
&= D_t (-r(t)V_t^\pi dt + G_tdB_t + H_t d S_t + q(t)H_tS_tdt) \\
&= D_t (-r(t)V_t^\pi dt +r(t)  G_t B_tdt + H_t d S_t + q(t)H_tS_tdt)\\
&= (q(t)-r(t))D_t H_tS_t dt + D_t H_t dS_t.
\end{align*}
Далее введем функцию $U_t = e^{\int_0^t q(s) ds}$, имеющую $dU_t = q(t) U_t dt$.
Заметим, что $d(U_tD_t) = (q(t)-r(t))U_t D_t dt$ (это видно из формулы $U_tD_t = e^{\int_0^t (q(u)-r(u))du}$).
Тогда
\begin{align*}
e^{-\int_0^t q(u)du} H_t d \tilde S_t &= \frac{H_t}{U_t} d(U_tD_tS_t) 
  = \frac{H_t}{U_t} (S_t d(U_tD_t) + U_tD_t dS_t) \\
&= (q(t)-r(t))D_t H_tS_t dt + D_t H_t dS_t. 
\end{align*}
У двух получившихся выражений совпадают правые части.
Значит совпадают и левые, что доказывает формулу \eqref{10:sf-discounted} для самофинансируемых стратегий.

Доказательство в обратную сторону проводится с использованием аналогичных рассуждений (упражнение).
\end{proof}


\subsection{Эквивалентная мартингальная мера}

\begin{theorem}
\label{10:t:emm}
В обобщенной модели \bs\ существует единственная вероятностная мера $\Q\sim\P$, относительно которой процесс $\tilde S$ является мартингалом.
\end{theorem}

\begin{proof}
Пользуясь теоремой Гирсанова, заменим исходную меру $\P$ на такую меру $\Q$, что относительно нее процесс 
\[
W_t^{\Q} = W_t + \int_0^t \frac{\mu+q(u)-r(u)}{\sigma} du
\]
станет броуновским движением (применимость теоремы Гирсанова легко следует из условия Новикова, учитывая ограниченность функций $q(u)$ и $r(t)$).
Тогда $d S_t$ примет вид
\begin{equation}
\label{10:dS}
d S_t = (r(t)-q(t))S_t dt + \sigma S_t d W_t^{\Q},
\end{equation}
что следует из формул $dS_t = \mu S_t dt + \sigma S_t dW_t$ и $d W_t = dW_t^{\Q} - (\mu+q(t)-r(t))/\sigma dt$.

Применяя формулу Ито к процессу $\tilde S_t = U_tD_t S_t$, где $U_t,D_t$ определены в доказательстве предложения~\ref{10:p:sf}, получаем 
\[
d \tilde S_t = \sigma \tilde S_t d W_t^{\Q}.
\]
По новой мере процесс $\tilde S_t$ является геометрическим броуновским движением с нулевым сносом, и, следовательно, является мартингалом.

Доказательство единственности, как и в предыдущей лекции, выходит за рамки курса.
\end{proof}

\begin{definition}
\label{10:d:admissible}
Будем называть стратегию $\pi=(G,H)$ \emph{допустимой}, если $HS \in \L_T^2(Q)$ (эквивалентно, $H\tilde S \in \L_T^2(Q)$), т.е.\ $\E^{\Q} \int_0^T (H_tS_t)^2 dt < \infty$.
\end{definition} 

Далее все стратегии считаются самофинансируемыми и допустимыми, если не оговорено иного.

\begin{proposition}
Для любой самофинансируемой стратегии (допустимость не требуется) имеем
\begin{equation}
\label{10:V-dynamics}
d \tilde V_t^\pi = \sigma e^{-\int_0^t q(s)ds} H_t \tilde S_t d W_t^{\Q},
\end{equation}
или, эквивалентно,
\[
d \tilde V_t^\pi = \sigma H_t \frac{S_t}{B_t} d W_t^{\Q}.
\]
В частности, $\tilde V_t^\pi$ "--- локальный мартингал относительно $\Q$. Если стратегия к тому же допустима, то $\tilde V_t^\pi$ "--- квадратично интегрируемый мартингал.
\end{proposition}

\begin{proof}
Первое утверждение легко следует из подстановки стохастического дифференциала $d \tilde S_t = \sigma \tilde S_t d W_t^{\Q}$ в формулу \eqref{10:sf-discounted}.
Второе утверждение вытекает из свойств интеграла Ито.
\end{proof}


\subsection{Цены платежных обязательств}

Как и в обычной модели \bs, будем отождествлять европейские платежные обязательства с $\F_T$-измеримыми случайными величинами $X$ такими, что $\E^{\Q} X^2 < \infty$.

Приводимая далее теорема и следствие из нее доказывается практически дословно так же, как аналогичные результаты в предыдущей лекции, поэтому доказательства мы опустим.

\begin{theorem}
В обобщенной модели \bs\ любое платежное обязательство имеет единственную реплицирующую стратегию, а цена репликации находится по формуле
\begin{equation}
\label{10:price}
V_t^X = e^{-\int_t^T r(s) ds} \E^{\Q}(X\mid \F_t).
\end{equation}
\end{theorem}

\begin{corollary}
Пусть платежное обязательство $X$ имеет вид $X = f(S_T)$, а функции $f(s)$, $r(t)$, $q(t)$ достаточно ``хорошие''. Тогда $V_t^X = V(t,S_t)$, где
\begin{equation}
\label{10:markov-price}
V(t,s) = e^{-\int_t^T r(u) du} \E^{\Q} (f(S_T) \mid S_t = s)
\end{equation}
является решением уравнения
\[
\left\{
\begin{aligned}
&V'_t(t,s) + (r(t)-q(t))sV'_s(t,s) + \frac{\sigma^2}{2} s^2 V''_{ss}(t,s) = r(t)V(t,s),
  \qquad t\in[0,T), s>0,\\
&V(T,s) = f(s), \qquad s>0.
\end{aligned}
\right.
\]
При этом реплицирующая стратегия задается компонентой $H_t = V'_s(t,S_t)$
\end{corollary}

Поясним, как использовать формулу \eqref{10:markov-price}.
Относительно меры $\Q$ процесс $\tilde S$ является геометрическим броуновским движением, \te\ $\tilde S_t = s_0 e^{\sigma W_t^\Q - \frac{\sigma^2}{2} t}$. Отсюда следует, что процесс $S_t$ имеет вид
\[
S_t = s_0 e^{\int_0^t (r(s)-q(s))ds} e^{\sigma W_t^\Q - \frac{\sigma^2}{2} t},
\]
и, следовательно, можно представить
\[
S_T = S_t e^{\int_t^T (r(s)-q(s) - \frac{\sigma^2}{2})ds} e^{\sigma (W_T^\Q - W_t^\Q)}. 
\]
Тогда вычисление математического ожидания в формуле \eqref{10:markov-price} сводится к вычислению ожидания функции от нормальной случайной величины $W_T^\Q - W_t^\Q \sim N(0,T-t)$, что можно сделать с помощью интегрирования по ее плотности.

Отсюда получаем новый вариант формулы \bs\ для цен европейских опционов колл и пут, а также формулу паритета цен колл-пут.

\begin{corollary}
Для цен европейских опционов колл и пут в обобщенной модели \bs\ Блэка--Шоулза справедливы формулы
\begin{equation}
\label{bs2:bs-formula}
\begin{aligned}
&\VC(t,s) = e^{-\int_t^T q(u) du} s \Phi(d_1) - e^{-\int_t^T r(u) du} K \Phi(d_2), \\
&\VP(t,s) = e^{-\int_t^T r(u) du} K \Phi(-d_2) - e^{-\int_t^T q(u) du} s \Phi(-d_1),\\
&d_1 = \frac{\ln(s/K) + \int_t^T (r(u)-q(u)+\sigma^2/2)du}{\sigma\sqrt{T-t}},\quad
d_2 = d_1 - \sigma\sqrt{T-t}.
\end{aligned}
\end{equation}
Кроме того, для опционов колл и пут с одинаковым временем исполнения и страйком выполняется \emph{паритет цен колл-пут}:
\begin{equation}
\label{10:bs-pc}
\VC(t,s) - \VP(t,s) = e^{-\int_t^T q(u) du} s - e^{-\int_t^T r(u) du} K.
\end{equation}
\end{corollary}


\section{Формула Блэка}

В этом разделе мы перепишем формулу \bs\ в другом виде и получим так называемую \emph{формулу Блэка}.
Они эквивалентны, но формула Блэка более удобна для практического использования.
В качестве вспомогательных результатов мы вычислим цену бескупонной облигации и форвардную цену рискового актива (эти результаты представляют и самостоятельный интерес).

\subsection{Цена бескупонной облигации и форвардная цена акции}

\begin{definition}
\emph{Бескупонная облигация} с временем погашения $T$  отождествляется с платежным обязательством, которое выплачивает детерминированную величину $X$ (\emph{номинал облигации}).
Для цены бескупонной облигации с номиналом $X=1$ в момент времени $t$ будем использовать обозначение $B(t,T)$. 
\end{definition}

\begin{proposition}
В обобщенной модели \bs
\[
B(t,T) = e^{-\int_t^T r(u) du}.
\]
\end{proposition}

\begin{proof}
Очевидно из формулы \eqref{10:price}.
\end{proof}

\begin{definition}
\emph{$T$-форвардной ценой} рискового актива в момент времени $t$ называется $\F_t$-измеримая случайная величина $F_t^T$ такая, что платежное обязательство $X=S_T - F_t^T$ имеет нулевую цену в момент времени $t$.

Если время исполнения $T$ ясно из контекста, то будем говорить просто \emph{форвардная цена} и использовать обозначение $F_t$.
\end{definition}

\begin{proposition}
В обобщенной модели \bs
\[
F_t^T = \E^\Q(S_T\mid \F_t) = S_t e^{\int_t^T (r(u)-q(u)) du}, \qquad
d F_t^T = \sigma F_t^T d W_t^{\Q}.
\]
В частности, форвардная цена является мартингалом относительно $\Q$.
\end{proposition}

\begin{proof}
Имеем
\begin{multline*}
0 = V_t^X = e^{-\int_t^T r(u) du} \E^{\Q}(S_T - F_t^T \mid \F_t) \\= e^{-\int_t^T r(u) du} \E^{\Q} (S_T\mid \F_t) - e^{-\int_t^T r(u) du} F_t^T
= e^{-\int_0^T q(u) du} S_t - e^{-\int_t^T r(u) du} F_t^T,
\end{multline*}
где в последнем равенстве воспользовались тем, что процесс $\tilde S_t = e^{\int_0^t (q(u)-r(u))du} S_t$ является мартингалом относительно меры $\Q$.
Отсюда получаем первую доказываемую формулу.
Применяя формулу Ито и используя выражение для стохастического дифференциала $d S_t$ из \eqref{10:dS}, получаем вторую формулу.
\end{proof}

\subsection{Формулировка и доказательство формулы Блэка}

Пусть $V_t^\text{call}$ и $V_t^\text{put}$ обозначают цены европейских опционов колл и пут.
Будем считать время исполнения $T$ фиксированным и рассмотрим процесс $T$-форвардной цены $F_t$ и цену бескупонной облигации $B(t,T)$. 

\begin{theorem}[формула Блэка]
\label{10:t:black}
Цены опционов колл и пут можно найти в виде $V_t^\text{call} = \VC(t,F_t)$ и $V_t^\text{put} = \VP(t,F_t)$, где функции $\VC(t,f)$ и $\VP(t,f)$ имеют вид
\begin{equation}
\label{bs2:b-formula}
\begin{aligned}
&\VC(t,f) = B(t,T) (f\Phi(d_1) - K\Phi(d_2)), \\
&\VP(t,s) = B(t,T) (K\Phi(-d_2) - f\Phi(-d_1)),\\
&d_1 = \frac{1}{\sigma\sqrt{T-t}} \left(\ln\frac{f}{K} + \frac{\sigma^2}{2}(T-t)\right), \quad
d_2 = d_1 - \sigma\sqrt{T-t}.
\end{aligned}
\end{equation}
\end{theorem}

\begin{proof}
Эти формулы можно получить, если подставить в них выражения для $B(t,T)$ и $F_t$ и показать, что получается формула \bs.

Приведем другой метод доказательства.
Рассмотрим опцион колл (для опциона пут все аналогично). Так как $S_T = F_T$, то
\[
V_t^\text{call} = e^{-\int_t^T r(u) du} \E^{\Q} ((S_T-K)^+ \mid \F_t) 
= B(t,T) \E^{\Q}((F_T - K)^+ \mid \F_t). 
\]
Процесс $F_t$ является геометрическим броуновским движением с нулевым сносом относительно $\Q$, а поэтому для вычисления ожидания в правой части можно воспользоваться стандартной формулой Блэка--Шоулса с нулевой безрисковой ставкой и нулевой дивидендной доходностью (считая, что $F_t$ играет роль цены рискового актива).
Это дает
\[
\E^{\Q}((F_T - K)^+ \mid \F_t) = V(t,F_t),
\]
где $V(t,f)$ -- такая же функция, как в формуле Блэка--Шоулса c $r=q=0$.
Это как раз то, что требовалось показать.
\end{proof}

\noindent
Аналогичным образом можно получить и формулу для паритета колл-пут.

\begin{theorem}[паритет цен колл-пут]
Для опционов колл и пут с одинаковым временем исполнения $T$ и одинаковым страйком $K$ выполнено равенство
\[
V_t^\text{call} - V_t^\text{put} = B(t,T) (F_t - K).
\]
\end{theorem}

\begin{proof}
Можно непосредственно переписать формулу \eqref{10:bs-pc}, либо использовать следующую цепочку равенств:
\begin{multline*}
V_t^\text{call} - V_t^\text{put} = e^{-\int_t^T r(u) du} \E^{\Q} (S_T - K\mid \F_t) \\= 
B(t,T) \E^{\Q} (F_T - K\mid \F_t) = B(t,T) (F_t - K),
\end{multline*}
где в последнем равенстве воспользовались мартингальностью процесса $F_t$. 
\end{proof}

\begin{remark}[о практических применениях]
Формулы \bs и Блэка математически эквивалентны, но для практических применений формула Блэка гораздо удобнее.
Дело в том, что формула \bs\ требует, чтобы были заданы функции $r(t)$ и $q(t)$.
Это означает, что сначала нужно выбрать модель для безрисковой ставки и дивидендов, потом определить ее параметры, и только тогда можно использовать формулу Блэка--Шоулза.

Формула Блэка в этом смысле проще: входящие в нее величины $B(t,T)$ и $F_t$ можно найти из рыночных данных, не используя никакую конкретную модель.
А именно, $B(t,T)$ находится из данных по рынку облигаций%
\footnote{Либо берется облигация с временем погашения в точности $T$, либо, если именно с таким временем погашения облигации не торгуются, то интерполируются цены соседних облигаций.},
а $F_t$ можно найти из паритета цен колл-пут%
\footnote{Форвардные контракты не торгуются на бирже, поэтому для них нет публичных цен.}:
возьмем из рыночных данных цены опционов $V_t^\text{call}$ и $V_t^\text{put}$ для какого-либо страйка $K$, цену $B(t,T)$ посчитаем из данных по облигациям, и тогда получим $F_t = K + (V_t^\text{call} - V_t^\text{put})/B(t,T)$.
\end{remark}


\section{Цены фьючерсов}
\label{bs2:ss:futures}

Покажем, как найти цену фьючерса в обобщенной модели \bs.
Мы будем следовать здесь тем же рассуждениям, что для полной модели рынка в дискретном времени (см.\ раздел \ref{fut:s:complete} в лекции \ref{ch:futures-discrete}) "--- определим цену репликации фьючерса и покажем, что она задается однозначно.

Ради большей общности рассуждений, как и в дискретном времени, будем рассматривать произвольные маржируемые контракты, частным случаем которых является фьючерс.
Маржируемый контракт задается расчетной ценой $X$ в момент экспирации $T$ (фьючерс получается при $X=S_T$).
Далее всегда будем предполагать, что $\E^{\Q} X^2 < \infty$.
Кроме того, будет считаться, что клиринг происходит <<непрерывно>> во времени.

\begin{definition}
\label{bs2:d:fut-price}
\emph{Ценой репликации} маржируемого контракта $X$ называется процесс Ито $F=(F_t)_{t\in[0,T]}$ такой, что $F_T=X$ и существует процесс $\pi_t=(G_t,H_t)$ с компонентами $G\in\PP^1_T$, $H\in\PP^2_T$, причем $HS\in\L_T^2(\Q)$, удовлетворяющий равенствам
\begin{align}
\label{bs2:fut-df}
&dF_t = G_t dB_t + H_tdS_t,\\
\label{bs2:fut-v}
&G_tB_t + H_tS_t = 0,
\end{align}
\end{definition}

\begin{remark}
Равенство \eqref{bs2:fut-df} нужно понимать в интегральном смысле, \te\ $F_t = F_0 + \int_0^t G_u dB_u + \int_0^t H_u d S_u$, где значение $F_0$ не известно.
\end{remark}

Интерпретация равенств \eqref{bs2:fut-df}--\eqref{bs2:fut-v} такая же, как в дискретном времени (определение \ref{fut:d:replication} в лекции \ref{ch:futures-discrete}; более точно, в том определении нужно оставить формулу \eqref{fut:complete-next} "--- аналог формулы \eqref{bs2:fut-v} выше, а формулу \eqref{fut:complete-cur} заменить на разность \eqref{fut:complete-cur} и \eqref{fut:complete-next} "--- аналог \eqref{bs2:fut-df}).
Равенство \eqref{bs2:fut-df} означает, что изменение стоимости портфеля стратегии $\pi$ совпадает с вариационной маржой по контракту.
Равенство \eqref{bs2:fut-v} означает, что стоимость портфеля равна 0, что соответствует тому, что позицию по маржируемому контракту можно открывать и закрывать без издержек.
Отметим, что стратегия $\pi$ не является самофинансируемой "--- приток капитала у нее равен вариационной марже.

\begin{proposition}
В обобщенной модели \bs\ для любого маржируемого контракта $X$ (квадратично интегрируемого по $\Q$) цена репликации существует и задается формулой
\[
F_t = \E^\Q (X\mid \F_t).
\]
В частности, она является квадратично интегрируемым мартингалом относительно $\Q$. 
\end{proposition}

\begin{proof}
Так определенный процесс $F_t$ является квадратично интегрируемым мартингалом по $\Q$ в силу квадратичной интегрируемости $X$.
Согласно теореме о мартингальном представлении, найдется процесс $H'\in \L_T^2(Q)$, для которого 
\[
F_t = F_0 + \int_0^t H_u' d W_u^{\Q}.
\]
Положим $H_t = H_t'/S_t$ и $G_t = -H_t'/B_t$.
Так как $H'\in \L_T^2(\Q)$, то, очевидно, $HS\in\L^2_T(\Q)$, а, следовательно, и $H\in\PP_2^T(\Q)$ и $\G\in\PP^1(\Q)$.
Кроме того, легко видеть, что для $\pi_t=(G_t,H_t)$ выполнены свойства \eqref{bs2:fut-df}--\eqref{bs2:fut-v}.
Значит, $F$ является ценой репликации контракта $X$.

Единственность реплицирующей стратегии следует из того, что для любой реплицирующей стратегии $\pi$ и соответствующей цены $F$ равенства \eqref{bs2:fut-df}--\eqref{bs2:fut-v} влекут
\[
d F_t = -\frac{H_tS_t}{B_t}dB_t + H_tdS_t = \sigma H_t S_t dW_t^\Q,
\]
и, следовательно, $F$ является квадратично интегрируемым мартингалом по $\Q$. По теореме о мартингальном представлении, процесс $\sigma H_t S_t$ однозначно определен.
Значит, однозначно определен и процесс $H_t$. Тогда $G_t$ из него однозначно выражается по формуле \eqref{bs2:fut-v}.
\end{proof}

\begin{corollary}
В обобщенной модели \bs\ $T$-форвардная и $T$"=фьючерсная цена рискового актива совпадают и равны $F_t=\E^\Q(S_T\mid \F_t)$.
\end{corollary}

\begin{remark}
Совпадение форвардных и фьючерсных цен имеет место благодаря тому, что безрисковая процентная ставка считается детерминированной.
Для стохастической ставки это будет не так.
\end{remark}

\begin{remark}
\label{bs2:r:black}
Представляет также интерес \emph{модель Блэка} фьючерсного рынка (часто называемая \emph{моделью Блэка--76} по году публикации работы \cite{Black76}), в которой нет торгуемого базового рискового актива $S_t$, а присутствуют только безрисковый актив и фьючерс, цена которого относительно исходной вероятностной меры $\P$ моделируется геометрическим броуновским движением.

Более подробно модель Блэка изложена в дополнении \ref{ch:black}, где, в частности, выводится \emph{формула Блэка} для цен опционов на фьючерсы.

% Показывается, что в модели Блэка существует единственная ЭММ, относительно которой цена фьючерса является мартингалом (геометрическим броуновским движением с нулевым сносом).
% Тогда цена (премиального) платежного обязательства $X$ вычисляется в виде $V_t=B(t,T)\E^{\Q} (X/B_T\mid\F_t)$.

% В частности, для цен европейских \emph{опционов на фьючерс}%
% \footnote{Опцион на фьючерс "--- это контракт, который дает право покупателю в будущем открыть длинную (опцион колл) или короткую (опцион пут) фьючерсную позицию по фиксированной цене.
% Считается, что экспирация фьючерса происходит не раньше экспирации опционов.}
% получаем формулы $\VC_t = B(t,T)\E^\Q((F_T-K)^+\mid \F_t)$ и $\VP = B(t,T)\E^\Q((K-F_T)^+ \mid \F_t)$,  которые после вычисления математических ожиданий превращаются в точности в фрмулу Блэка.
% Уточним, что здесь $F_T$ обозначает цену фьючерса в момент времени $T$, который экспирируется в момент времени $T'\ge T$. В формулу Блэка вместо параметра $f$ нужно подставлять цену этого фьючерса в момент $t$. Параметр $T'$ в формулу не входит.


\end{remark}


\section{Модель валютного рынка}

Рассмотрим рынок из двух активов "--- денежного счета в \emph{домашней валюте} с безрисковой ставкой $r_d(t)$ и денежного счета в \emph{иностранной валюте} с безрисковой ставкой $r_f(t)$.
Обе ставки детерминированные, а соответствующие процессы цен безрисковых активов имеют вид
\begin{align*}
&d B_t^d = r_d(t) B_t^d dt, \qquad B_0^d = 1,\\
&d B_t^f = r_f(t) B_t^f dt, \qquad B_0^f = 1,
\end{align*}
но при этом $B_t^d$ показывает стоимость в единицах домашней валюты, а $B_t^f$ "--- в единицах иностранной валюты.

Вся случайность в модели связана с обменным курсом.
Будем предполагать, что он моделируется геометрическим броуновским движением
\[
d R_t = \mu R_t dt + \sigma R_t d W_t, \qquad R_0>0,
\]
где $R_t$ выражает стоимость одной единицы иностранной валюты в единицах домашней валюты. 

Как и в модели \bs, считается что фильтрация порождена броуновским движением $W$. 

\begin{definition}
\emph{Торговой стратегией} в модели валютного рынка будем называть пару измеримых согласованных процессов $\pi=(G,H)$, где процесс $G=(G_t)_{t\in[0,T]} \in \PP^1_T$ равен количеству единиц домашнего безрискового актива в портфеле, а $H=(H_t)_{t\in[0,T]} \in \PP^2_T$ -- количеству единиц иностранного безрискового актива. 
\emph{Стоимостью портфеля} стратегии $\pi$ (в домашней валюте) называется процесс 
\[
V_t^\pi = G_t B_t^d + H_t B_t^f R_t.
\]
Стратегия называется \emph{самофинансируемой}, если 
\begin{equation}
\label{10:forex-sf}
d V_t^\pi = G_t d B_t^d + R_t H_t d B_t^f + H_t B_t^f dR_t.
\end{equation}
\end{definition}

Поясним смысл условия самофинансируемости.
Оно является непрерывным аналогом следующего соотношения в дискретном времени (ср.~с формулой на стр.~\pageref{9:self-financing-discrete}):
\[
\Delta V_t^\pi = G_t \Delta B_t^d + R_t H_t\Delta B_t^f + H_tB_t^f\Delta R_t^f.
\]
Первое слагаемое в правой части "--- это приращение стоимости портфеля за счет изменения цены домашнего безрискового актива (эквивалентно, за счет начисления процентов на счет в домашней валюте).
Второе слагаемое "--- приращение стоимости портфеля за счет изменения цены иностранного безрискового актива, выраженное в домашней валюте (умножается на обменный курс).
Третье слагаемое "--- изменение стоимости иностранной части портфеля из-за изменения обменного курса.

\medskip
Далее можно было бы повторить те же рассуждения, что мы проделали для модели Блэка--Шоулза выше (ввести понятие допустимых стратегий, эквивалентной мартингальной меры и \td), но будет проще пойти другим путем и показать, что модель рынка валют можно свести к модели с дивидендами.

Заметим, что стоимость портфеля стратегии $\pi_t=(G_t,H_t)$ будет такая же, как стоимость портфеля стратегии $\pi'=(G_t,H_t')$, где $H_t' = H_tB_t^f$, в обобщенной модели \bs, в которой цена безрискового актива $B_t=B_t^d$, цена рискового актива $S_t=R_t$, безрисковая процентная ставка $r(t) = r_d(t)$, а дивидендная доходность $q(t) = r_f(t)$.
Это вытекает из формулы $V_t^{\pi'} = G_t B_t^d  + H_t' S_t$. 

Следовательно, имеется равенство дифференциалов $d V_t^{\pi'} = d V_t^{\pi}$, а, с другой стороны, формулу \eqref{10:forex-sf} можно переписать в виде
\[
dV_t^{\pi} =  G_t d B_t + q(t) H_t S_t dt + H_t d S_t,
\]
что есть условие самофинансируемости в обобщенной модели \bs.
Таким образом, если стратегия $\pi$ самофинансируемая, то и $\pi'$ самофинансируемая (каждая в своей модели), и наоборот.

Из этого следует, что цена любого платежного обязательства должна совпадать в обеих моделях. 
Отсюда легко получить формулы для цены бескупонной облигации, форвардного и фьючерсного обменного курса и цен опционов в модели валютного рынка.
Сначала дадим соответствующие определения.

\begin{definition}
\emph{Бескупонная облигация} (в домашней валюте) отождествляется с платежным обязательством $X=1$.
\emph{Бескупонная еврооблигация}\footnote{Еврооблигациями называются облигации, номинированные в иностранной валюте. Приставка <<евро>> "--- дань традиции; она не означает, что облигации номинированы в евро.} отождествляется с платежным обязательством $X=R_t$ (\te\ она выплачивает единицу иностранной валюты при погашении).

\emph{$T$-форвардным обменным курсом}%
\footnote{Форвардный контракт -- это соглашение о покупке единицы иностранной валюты в момент времени $T$ по курсу $F_t^T$, который фиксируется в момент $t$.}
в момент времени $t$ называется $\F_t$"=измеримая случайная величина $F_t^T$ такая, что платежное обязательство $X=R_T - F_t^T$ имеет нулевую цену в момент времени $t$. \emph{Валютным фьючерсом} называется фьючерс на поставку единицы иностранной валюты в момент экспирации, отождествляемый с маржируемым контрактом с расчетной ценой $F_T=R_T$.

\emph{Валютные опционы колл и пут} (европейского типа) с временем исполнения $T$ и страйком $K$ дают право купить/продать единицу иностранной валюты в момент $T$ по курсу $K$ и отождествляются с платежными обязательствами $X= (R_T - K)^+$ и $X=(K-R_T)^+$.
\end{definition}

Теперь, сводя модель валютного рынка к обобщенной модели \bs, получаем следующий результат.
\begin{proposition}
В рассматриваемой модели валютного рынка
\begin{itemize}
\item цена бескупонной домашней облигации равна $B_d(t,T) = e^{-\int_t^T r_d(u) du}$;
\item цена бескупонной еврооблигации равна $B_f(t,T) = R_t e^{-\int_t^T r_f(u) du}$ (в единицах домашней валюты);
\item $T$-форвардный обменный курс и цена фьючерса с моментом экспирации $T$ совпадают и равны $F_t^T = R_t e^{\int_t^T (r_d(u)-r_f(u)) du}$;
\item формула Блэка для цен опционов имеет такой же вид, как в теореме~\ref{10:t:black}.
\end{itemize}
\end{proposition}


\summary
\begin{itemize}
\item В обобщенной модели \bs\ процентная ставка $r(t)$ неслучайна, но зависит от времени, а рисковый актив выплачивает дивиденды со ставкой доходности $q(t)$. Цены опционов колл и пут можно найти из обобщенной формулы \bs\ \eqref{bs2:bs-formula}.

\item Формула Блэка \eqref{bs2:b-formula} позволяет переписать формулу \bs\ через форвардную цену рискового актива и цену бескупонной облигации таким образом, что в нее в явном виде не входят безрисковая ставка и дивидендная доходность.

\item Цена бескупонной облигации с номиналом 1 равна $B(t,T) = e^{-\int_t^t r(s) ds}$, а $T$-форвардная цена рискового актива равна $F_t^T = S_t e^{\int_t^T (r(u)-q(u)) du}$, причем последняя является мартингалом относительно ЭММ. 

\item В модели валютного рынка присутствуют два безрисковых актива "--- в домашней и зарубежной валюте "--- с разными процентными ставками $r_d(t)$ и $r_f(t)$.
Обменный курс задается геометрическим броуновским движением. 

\item Для опционов на зарубежную валюту верна формула Блэка, в которую нужно подставить цену бескупонной домашней облигации $B_d(t,T) = e^{-\int_t^t r_d(s) ds}$ и $T$-форвардный обменный курс $F_t^T = R_t e^{\int_t^T (r_d(u)-r_f(u)) du}$.
\end{itemize}
