%!TEX root=finmath1.tex
\chapter{Интеграл Ито}
\label{ch:ito-integral}
\chaptertoc

В этой лекции мы определим стохастический интеграл по броуновскому движению (\emph{интеграл Ито})
\[
\int_0^t H_s d W_s,
\]
где $H$ -- случайный процесс, а $W$ -- стандартное броуновское движение.
Мы также получим \emph{формулу Ито} о дифференцировании функций от броуновского движения.
Эти теоретические понятия необходимы для построения моделей рынков с непрерывным временем.


\section{Конструкция интеграла Ито}

Трудность определения интеграла по броуновскому движению состоит в том, что его нельзя определить потраекторно, \te\ зафиксировать $\omega$ и вычислить $\int_0^t H_s(\omega) d W_s(\omega)$.
Действительно, написать $\int_0^t H_s(\omega) W_s'(\omega) ds$ нельзя, так как траектории броуновского движения не дифференцируемы.
Не получится воспользоваться и интегралами Римана"--~Стилтьеса или Лебега"--~Стилтьеса, так как броуновское движение имеет траектории неограниченной вариации. 

Конструкция интеграла Ито будет дана в три этапа.
Сначала мы определим его для кусочно-постоянных случайных процессов $H$ (называемых \emph{простыми}).
Затем распространим определение интеграла на согласованные измеримые процессы $H$, удовлетворяющие условию $\E \int_0^t H_s^2 ds <\infty$.
Наконец, общая конструкция интеграла будет дана для согласованных измеримых процессов со свойством $\int_0^t H_s^2 ds < \infty$ \as


\subsection{Интеграл Ито для простых процессов}

Пусть задано фильтрованное вероятностное пространство $(\Omega, \F, \FF, \P)$ и стандартное броуновское движение $W$ на нем.
Для начала будем считать временной горизонт ограниченным моментом времени $T$, поэтому все процессы будут рассматриваться на отрезке $[0,T]$.

\begin{definition}
Случайный процесс $H=(H_t)_{t\in[0,T]}$ называется \emph{простым}, если его можно представить в виде
\begin{equation}
\label{7:simple}
H_t = \xi_0 \I(t=0) + \sum_{i=1}^n \xi_i \I(t \in (t_{i-1}, t_i]),
\end{equation}
где $0= t_0 < \ldots < t_n =  T$, а $\xi_i$ являются $\F_{t_{i-1}}$-измеримыми ограниченными%
\footnote{Ограниченность означает, что $|\xi_i| \le c$ \as\ для некоторой константы $c$.}
случайными величинами.
\end{definition}

Таким образом, у простого процесса траектории кусочно постоянны, а на каждом интервале постоянства $(t_{i-1},t_i]$ значение траектории является случайной величиной, измеримой относительно $\F_{t_{i-1}}$.

\begin{definition}
Для простого процесса $H$ определим \emph{интеграл Ито} на отрезке $[0,T]$ как случайную величину
\begin{equation}
\int_0^T H_s d W_s 
= \sum_{i=1}^n \xi_i (W_{t_i} - W_{t_{i-1}}).
\end{equation}
\end{definition}

Идея этого определения довольно естественна: так как процесс $H$ кусочно постоянный, то в качестве значения интеграла нужно взять сумму значений этого процесса на каждом отрезке постоянства, умноженную на приращение броуновского движения.
Заметим, что значение $H_0$ (\te\ $\xi_0$) не учитывается. 

Необходимо показать корректность определения: если представить простой процесс $H$ по формуле \eqref{7:simple} двумя способами с разными разбиениями, то значения интегралов совпадут.
Это можно сделать с помощью перехода к объединенному разбиению.
Доказательство остается в качестве упражнения.

\begin{definition}
Для простого процесса $H$ определим интеграл Ито с переменным верхним пределом $t\in[0,T]$ как случайный процесс $\int_0^t H_s d W_s = I_t(H)$, где
\begin{equation}
I_t(H) = \int_0^T H_s \I(s\le t) d W_s
\end{equation}
(интеграл в правой части определен, так как  $H_s\I(s\le t)$ "--- простой процесс).

В явном виде имеем
\begin{equation}
\label{7:simple-integral}
\begin{aligned}
\int_0^t H_s d W_s &= \sum_{i=1}^n \xi_i (W_{t\wedge t_i} - W_{t\wedge t_{i-1}}) \\
& = \sum_{i=1}^k \xi_i (W_{t_i} - W_{t_{i-1}}) + \xi_{k+1} (W_t - W_{t_k})\ \text{для}\ t\in(t_k,t_{k+1}].
\end{aligned}
\end{equation}
\end{definition}

Далее, говоря <<интеграл Ито>>, мы, как правило, будем иметь ввиду именно интеграл с переменным верхним пределом, и для нас будут важны свойства, которыми он обладает как случайный процесс: (непрерывность траекторий, мартингальное свойство и \td).

\begin{proposition}
Пусть $H$ "--- простой процесс.
Тогда процесс $I_t(H) = \int_0^t H_s dW_s$ обладает следующими свойствами:
\begin{alphenum}
\item $I_t(H)$ является согласованным с фильтрацией $\FF$, имеет непрерывные траектории, и $I_0(H) = 0$;
\item $I_t(H)$ является мартингалом относительно $\FF$;
\item $\E I_t(H) = 0$;
\item если $G$ "--- тоже простой процесс, то $\E(I_t(H)I_t(G)) = \E \int_0^t H_sG_s ds$ и, в частности, $\E(I_t(H)^2) = \E \int_0^t H_s^2 ds$.
\end{alphenum}
\end{proposition}

\begin{remark}
В последнем свойстве можно поменять местами интеграл по $ds$ и математическое ожидание%
\footnote{Для этого можно воспользоваться теоремой Фубини, но конкретно здесь все проще, так как интеграл является конечной суммой (подынтегральное выражение кусочно-постоянно).}:
$\E(I_t(H)I_t(G)) = \int_0^t \E(H_sG_s) ds$.  
\end{remark}

\begin{proof}
Свойство а) следует из формулы \eqref{7:simple-integral} и непрерывности траекторий броуновского движения.
Конечность ожидания $\E I_t(H)$ следует опять же из этой формулы, учитывая, что величины $\xi_i$ ограничены.

Проверим мартингальное свойство $\E(I_t(H) \mid \F_s) = I_s(H)$.
Пусть $t\in (t_k,t_{k+1}]$, а $s\in (t_j, t_{j+1}]$, где $j\le k$.
Тогда из \eqref{7:simple-integral} имеем
\[
I_t(H) - I_s(H) = \xi_{j+1}(W_{t_{j+1}} - W_s) + \sum_{i=j+2}^k \xi_i(W_{t_{i+1}} - W_{t_i}) + \xi_{k+1}(W_t - W_{t_k}). 
\]
Пользуясь независимостью приращений броуновского движения, нетрудно показать, что условное ожидание $\E(\,\cdot\,|\, \F_s)$ каждого слагаемого в правой части равно нулю, и, таким образом, $\E(I_t(H) -I_s(H) \mid \F_s) = 0$.
Следовательно, $I_t(H)$ -- мартингал.
Это доказывает свойство b).
Свойство c) следует из того, что $\E I_t(H) = \E I_0(H) = 0$ в силу мартингальности.

Докажем d).
Достаточно рассмотреть случай $t=T$ (иначе возьмем процессы $\tilde H_s = H_s \I(s\le t)$ и $\tilde G_s = G_s \I(s\le t)$).
Также можно считать, что разбиения $t_i$ у процессов $H$ и $G$ совпадают (всегда можно взять общее разбиение).
Пусть
\[
H_t = \xi_0 \I(t=0) + \sum_{i=1}^n \xi_i \I(t \in (t_{i-1}, t_i]), \qquad
G_t = \eta_0 \I(t=0) + \sum_{i=1}^n \eta_i \I(t \in (t_{i-1}, t_i]).
\]
Имеем
\begin{multline*}
\E(I_T(H)I_T(G)) 
= \E\biggl(\sum_{i=1}^n \xi_i (W_{t_i} - W_{t_{i-1}}) \cdot
  \sum_{i=1}^n \eta_i (W_{t_i} - W_{t_{i-1}})\biggr) \\
= \sum_{i=1}^n \E(\xi_i\eta_i (W_{t_i} - W_{t_{i-1}})^2)
+ \sum_{i\neq j} \E(\xi_i\eta_j (W_{t_i} - W_{t_{i-1}})(W_{t_j} - W_{t_{j-1}})).
\end{multline*}
Для слагаемых в первой сумме, пользуясь независимостью приращений броуновского движения, находим
\begin{multline*}
\E(\xi_i\eta_i \E(W_{t_i} - W_{t_{i-1}})^2)
= \E(\E(\xi_i\eta_i (W_{t_i} - W_{t_{i-1}})^2 \mid \F_{t_{i-1}})) \\
= \E(\xi_i\eta_i \E(W_{t_i} - W_{t_{i-1}})^2) = \E(\xi_i\eta_i) (t_i - t_{i-1}).
\end{multline*}
Покажем, что все слагаемые во второй сумме равны нулю. Пусть $i<j$ (случай $i>j$ рассматривается аналогично). Тогда
\begin{multline*}
\E (\xi_i\eta_j (W_{t_i} - W_{t_{i-1}})(W_{t_j} - W_{t_{j-1}})) \\
= \E (\E(\xi_i\eta_j (W_{t_i} - W_{t_{i-1}}) (W_{t_j} - W_{t_{j-1}}) 
  \mid \F_{t_{j-1}}))\\
= \E (\xi_i\eta_j (W_{t_i} - W_{t_{i-1}}) \E(W_{t_j} - W_{t_{j-1}} 
  \mid \F_{t_{j-1}})) 
= 0.
\end{multline*}
В итоге получаем
\[
\E(I_T(H)I_T(G))^2 = \sum_{i=1}^n\E( \xi_i\eta_i) (t_i -t_{i-1}) =
\E\left( \sum_{i=1}^n \xi_i\eta_j (t_i -t_{i-1})\right) 
= \E \int_0^T H_sG_s ds,
\]
что и требовалось доказать.
\end{proof}


\subsection{Интеграл Ито для процессов из пространства $\L^2$}

Определенный выше интеграл Ито можно рассматривать как отображение из класса простых процессов в класс непрерывных квадратично интегрируемых мартингалов.
Каждому простому процессу оно ставит в соответствие его процесс-интеграл $I_t(H)$.
Следующий этап конструкции состоит в том, чтобы продолжить по непрерывности это отображение на более широкий класс подынтегральных процессов.
Под продолжением по непрерывности понимается то, что если простые процессы $H^n$ сходятся в подходящем смысле к процессу $H$, то нужно положить по определению $I_t(H) = \lim_{n\to\infty} I_t(H^n)$.

Чтобы говорить сходимости и непрерывности, зададим структуру метрического пространства для подынтегральных процессов и процессов-интегралов. 
Напомним, что если $(S,\mathcal{S})$ "--- измеримое пространство, а $\mu$ "--- мера на нем (конечная, но не обязательно вероятностная), то пространством $L^2(S,\mathcal{S},\mu)$ называется линейное пространство классов эквивалентности измеримых функций $f(s)$ на $(S,\mathcal{S})$ таких, что $\int_S f(s)^2 \mu(ds) < \infty$.
Функции считаются эквивалентными, если они совпадают почти всюду по мере $\mu$.
Далее, как это обычно делается, мы будем опускать слова <<класс эквивалентности>>, а также писать просто $L^2$, когда ясно о каком измеримом пространстве и мере идет речь.

На пространстве $L^2$ определено скалярное произведение
\[
\langle f,\, g\rangle_{L^2} = \int_S f(s)g(s) \mu(ds).
\]
Скалярное произведение порождает норму и метрику
\[
\|f\|_{L^2} = \sqrt{\langle f,\, f\rangle_{L^2}}, \qquad \rho_{L^2}(f, g) = \|f-g\|_{L^2}.
\]
Известно, что пространство $L^2$ является \emph{гильбертовым пространством}, т.е.\ линейным пространством со скалярным произведением, полным относительно порождаемой им метрики%
\footnote{Полнота означает, что любая последовательность Коши $f_n$ в $L^2$ сходится по норме к некоторой функции $f$ в $L^2$.
Последовательность $f_n$ называется \emph{последовательностью Коши}, если для любого $\epsilon>0$ найдется $N$ такое, что $\rho(f_n, f_m) < \epsilon$ для всех $n,m>N$.}.

Далее в качестве пространства $(S,\mathcal{S},\mu)$ будем рассматривать $S=\Omega\times [0,T]$, $\mathcal{S} = \F\otimes\B([0,T])$ и $\mu=\P\otimes\mathrm{Leb}$, где $\mathrm{Leb}$ "--- мера Лебега.
Измеримыми функциями на нем являются измеримые случайные процессы $H=(H_t)_{t\in[0,T]}$.

\begin{definition}
Подпространство $\L^2_T\subset L^2(\Omega\times[0,T],\ \F\otimes\B([0,T])б\ \P\otimes\mathrm{Leb})$ состоит из измеримых согласованных случайных процессов $H$.
\end{definition}

Нетрудно заметить, что все простые процессы лежат в $\L^2_T$.
Следующее предложение устанавливает ключевое свойство, нужное для продолжения интеграла Ито с простых процессов на всё $\L^2_T$.

\begin{proposition}
Множество простых процессов всюду плотно%
\footnote{Для любого $H\in\L^2_T$ и $\epsilon>0$ найдется простой процесс $H'$ такой, что $\|H-H'\|_{L^2} \le \epsilon$.}
в $\L^2_T$.
\end{proposition}

Доказательство этого результата довольно трудное.
В полном виде его можно найти в книге \cite{LiptserShiryaev74}, глава 4, лемма 4.4. 


\medskip
Зададим теперь подходящее гильбертово пространство, в котором будут лежать процессы-интегралы.

\begin{definition}
Линейное пространство $\M^{2,c,0}_T$ \emph{непрерывных квадратично интегрируемых мартингалов с нулевым начальным значением} состоит из непрерывных мартингалов $M=(M_t)_{t\in[0,T]}$ c $M_0=0$ и $\E M_T^2 < \infty$.

Будем считать процессы $M,N\in \M_T^{2,c,0}$ эквивалентными, если они неразличимы.
На пространстве классов эквивалентности таких процессов определим скалярное произведение
\[
\langle M,N\rangle_{\M^2} = \E M_TN_T
\]
и будем обозначать за $\|M\|_{\M^2} = \sqrt{\langle M,M\rangle_{\M^2}}$ и $\rho_{\M^2}(M,N) = \|M-N\|_{\M^2}$ соответствующую норму и метрику.
\end{definition}

\begin{remark}
Если $M,N\in \M_T^{2,c,0}$, то они неразличимы тогда и только тогда, когда $M_T=N_T$ \as\ 
Действительно, если $M_T = N_T$ \as, то из мартингального свойства получаем, что $M_t = N_t$ \as\ для любого $t\le T$, \te\ $M$ является модификацией $N$.
В силу непрерывности процессов отсюда следует неразличимость.
Обратное утверждение очевидно.
\end{remark}

\begin{proposition}[см.~\cite{BulinskiShiryaev04}, глава VIII, лемма 2]
Пространство $M_T^{2,c,0}$ является гильбертовым.
\end{proposition}

Наконец, докажем лемму общего характера, из которой будет следовать возможность продолжения интеграла Ито.

\begin{lemma}[об изометрии]
Пусть $L,M$ "--- два метрических пространства, причем пространство $M$ полно.
Предположим, что на всюду плотном множестве $L_0\subset L$ задана функция $f_0\colon L_0\to M$, являющаяся изометрией, \te\ сохраняющей расстояния ($\rho_M(f(x),f(x')) = \rho_L(x,x')$ для всех $x,x'\in L_0$). 

Тогда существует единственная непрерывная функция $f\colon L\to M$ такая, что $f(x) = f_0(x)$ для всех $x\in L_0$, причем $f$ является изометрией на всем $L$.
\end{lemma}

\begin{proof}
Пусть $x\in L$.
В силу плотности $L_0$ найдется последовательность $x_n\in L_0$ такая, что $x_n\to x$.
В частности, $x_n$ является последовательностью Коши в $L$.
Тогда $f(x_n)$ является последовательностью Коши в $M$ по свойству изометрии.
Так как $M$ полно, то эта последовательность сходится к некоторому элементу $y\in M$.
Определим $f(x) = y$.

Нетрудно показать, что значение $f(x)$ не зависит от приближающей последовательности $x_n$ (если $x_n\to x$ и $x_n'\to x$, то, чтобы показать, что $y=y'$, нужно рассмотреть последовательность $x_1,x_1',x_2,x_2',\ldots$).

Функция $f$ является изометрией, так как для любых $x,x'\in L$ можно взять последовательности $x_n,x_n'\in L_0$ такие, что $x_n\to x$ и $x_n'\to x'$, и тогда
\[
\rho_M(f(x), f(x')) = \lim\limits_{n\to\infty} \rho_M(f(x_n),f(x_n')) = \lim\limits_{n\to\infty} \rho_L(x_n, x_n')_L= \rho_L(x,x'),
\]
где воспользовались непрерывностью метрики.
Из изометричности следует непрерывность, так как если $x_n\to x$ в $L$, то $\rho_M(f(x_n), f(x)) = \rho_L(x_n,x) \to 0$.

Единственность функции $f$ вытекает из ее непрерывности и плотности множества $L_0$.
\end{proof}

Теперь у нас все готово для продолжения интеграла Ито. 
Воспользуемся тем, что интеграл, заданный на множестве простых процессов, является изометрией, отображающей простые процессы в непрерывные квадратично интегрируемые мартингалы.
Следовательно, его можно единственным образом продолжить до непрерывного отображения $H\mapsto I(H)$, где $H\in \L^2_T$ и $I(H)\in \M_T^{2,c,0}$.

\begin{definition}
\emph{Интегралом Ито} процесса $H\in \L^2_T$ называется непрерывный квадратично интегрируемый мартингал $I_t(H) = \int_0^t H_s d W_s$, являющийся значением продолжения отображения $H\mapsto I(H)$ с множества простых процессов на всё $\L^2_T$.
\end{definition}

\begin{proposition}
\label{3:p:integral-properties}
Пусть $H,G\in \L^2_T$. Тогда выполнены следующие свойства.
\begin{alphenum}
\item $\int_0^t H_s d W_s = \int_0^T H_s I(s\le t) d W_s$.

\item Если случайная величина $\xi$ является $\F_u$-измеримой, $0\le u \le t$, и $\E\xi^2 < \infty$, то $\int_u^t \xi H_s dW_s = \xi \int_u^t H_s d W_s$ (где для процесса $V\in \L^2_T$ мы определяем $\int_u^t V_s dW_s = \int_0^t V_s\I(s\ge u) dW_s$).

\item $\int_0^t(aH_s+bG_s) d W_s = a\int_0^t H_s d W_s + b\int_0^t G_s d W_s$ для любых $a,b\in\R$.

\item $\E \int_0^t H_s d W_s = 0$.

\item $\E(\int_0^t H_s d W_s \cdot \int_0^t G_s d W_s) = \E \int_0^t H_sG_s ds$ (изометрия Ито).
\end{alphenum}
\end{proposition}

\begin{proof}
a) Возьмем простые процессы $H^n\to H$.
Для них непосредственно проверяется, что $\int_0^t H_s^n d W_s = \int_0^T H_s^n \I(s\le t)d W_s$.
Заметим, что процессы $H_s^n \I(s\le t)$ сходятся к $H_s\I(s\le t)$ в $\L^2_T$.
Следовательно, 
\[
\lim\limits_{n\to\infty} \int_0^T H_s^n \I(s\le t)d W_s = \int_0^T H_s\I(s\le t) dW_s.
\]
Остается показать, что $\lim\limits_{n\to\infty} \int_0^t H_s^n d W_s = \int_0^t H_s d W_s$. 
В силу мартингального свойства имеем
\[
\int_0^t H_s dW_s = \E\left(I_T(H) \mid \F_t \right), \qquad
\int_0^t H_s^n dW_s = \E\left(I_T(H^n) \mid \F_t \right).
\]
Так как случайные величины $I_T(H^n)$ сходятся к $I_T(H)$ в $L^2(\Omega,\F_T,\P)$, то сходятся и их условные ожидания.
Это доказывает свойство a).

Свойство b) доказывается аналогично.
Свойство c) для простых процессов проверяется непосредственно, а в общем случае нужно найти $H^n\to H$ и $G^n\to G$ в $\L^2$ (и тогда $aH^n+bG^n \to aH+bG)$, и перейти к пределу в соотношении $I(aH^n+bG^n) = aI(H^n) + bI(G^n)$.

Свойство d) следует из того, что $I_t(H)$ является мартингалом с нулевым начальным значением и, следовательно, нулевым средним. 

Свойство e) для $t=T$ следует из изометричности интеграла: нам нужно показать, что $\langle I_T(H), I_T(G) \rangle_{\M^2} = \langle H, G \rangle_{L^2}$, а это верно, так как изометрия сохраняет скалярные произведения%
\footnote{Если $f$ -- изометрия одного пространства со скалярным произведением в другое, то сохраняются нормы, так как $\|x\| = \rho(x,0)$.
Отсюда следует сохранение скалярных произведений, так как $\langle x,y\rangle = \frac{1}{4}(\|x+y\|^2 - \|x-y\|^2)$.}.
Чтобы получить утверждение для произвольного $t$, воспользуемся свойством a).
\end{proof}

Для полноты изложения, приведем утверждение, показывающее, как построить интеграл Ито на всей прямой.
Будем считать, что фильтрация и броуновское движение $W$ определены для всех $t\ge 0$.
Символом $\L^2_\infty$ будем обозначать линейное пространство\footnote{Пространство $\L^2_\infty$ не является гильбертовым "--- для этого нужно было бы потребовать, чтобы $\E \int_0^\infty H_s^2 ds < \infty$.} согласованных измеримых процессов $H=(H_t)_{t\ge 0}$ таких, что $\E \int_0^t H_s^2 ds < \infty$ для всех $t\ge0$.
Заметим, что $\L^2_\infty = \bigcap\limits_{T\ge0} \L^2_T$.

\begin{proposition}
Пусть $H\in\L^2_\infty$.
Тогда существует единственный непрерывный процесс $I(H) = (I_t(H))_{t\ge 0}$ такой, что $I_T(H) = \int_0^T H_t d W_t$ \as\ для всех $T\ge 0$, где интеграл понимается в смысле определения для процесса из $\L^2_T$.

Процесс $I(H)$ является мартингалом, причем $\E I_T^2(H) < \infty$ и для любого конечного момента времени $T$ он удовлетворяет всем свойствам из предыдущего предложения.
\end{proposition}

Доказательство нетрудное и основано на идее, что нужно положить по определению $I_T(H) = \sum\limits_{u=0}^\infty \int_0^T H_s d W_s \I(T \in [u, u+1))$.
Мы его здесь не приводим; детали можно найти в книге \cite{BulinskiShiryaev04}, глава VIII, теорема 2.


\subsection{\difficult\ Интеграл Ито для процессов из пространства \texorpdfstring{$\mathcal{P}^2$}{P2}}

\begin{definition}
Линейное пространство $\mathcal{P}^2_T$ состоит из согласованных измеримых случайных процессов $H=(H_t)_{t\in[0,T]}$ таких, что $\int_0^T H_t^2 dt < \infty$ \as
\end{definition}

Ясно, что $\L^2_T \subset \mathcal{P}^2_T$.
Кроме того, любой измеримый согласованный процесс, у которого траектории являются ограниченными функциями, лежит в $\mathcal{P}^2_T$.
В частности, в $\mathcal{P}^2_T$ лежат все непрерывные согласованные процессы.

Интеграл Ито может быть распространен на процессы из $\mathcal{P}^2_T$ (хотя в большинстве дальнейших применений в этом курсе достаточно будет интеграла по процессам из $\L^2_T$).
Идея состоит в том, чтобы аппроксимировать такие процессы с помощью процессов из $\L^2_T$, как показывает следующее предложение.

Дальнейшие результаты приводятся без доказательств, их можно найти в главе~4 книги \cite{LiptserShiryaev74}.


\begin{lemma}
Для любого процесса $H\in \mathcal{P}^2_T$ существует последовательность процессов $H^n \in \L^2_T$ такая, что $\int_0^T (H_s^n - H_s)^2 ds \to 0$ по вероятности при $n\to\infty$.
\end{lemma}

\begin{theorem}
Для любого процесса $H\in \mathcal{P}^2_T$ существует единственный согласованный непрерывный процесс $I_t(H)$, $t\in[0,T]$, такой, что для любой последовательности $H^n\in \L^2_T$, сходящейся к $H$ в смысле предыдущей леммы, имеем $I_t(H^n) \to I_t(H)$ по вероятности при $n\to\infty$ для всех $t\in[0,T]$.
\end{theorem}

\begin{definition}
Так определенный процесс $I_t(H)$ называется \emph{интегралом Ито} процесса $H$ и обозначается тоже как $\int_0^t H_s d W_s$.
\end{definition}

Далее аналогично рассуждениям в предыдущем разделе можно построить интеграл от процессов $H$, определенных на всех полупрямой $t\in\R_+$ и принадлежащих пространству $\mathcal{P}^2_\infty$, состоящем из измеримых согласованных процессов $H$ таких, что $\int_0^t H_t^2 dt < \infty$ \as\ для всех $t\in\R_+$.

\begin{proposition}
Интеграл Ито для процесса $H\in \mathcal{P}^2_T$ (или $H\in\mathcal{P}^2_\infty$) удовлетворяет свойствам a--c предложения \ref{3:p:integral-properties}, но в общем случае нельзя сказать, что он удовлетворяет свойствам d--e или является мартингалом.
\end{proposition}

Несмотря на то, что интеграл Ито $\int_0^t H_s d W_s$ для процесса $H\in \mathcal{P}^2_T$ не обязательно является мартингалом, он является \emph{локальным мартингалом}.
Дадим соответствующие определения.

\begin{definition}
\emph{Моментом остановки} по отношению к фильтрации $\FF$ называется неотрицательная случайная величина $\tau$ такая, что 
\[
\{\omega: \tau(\omega) \le t \} \in \F_t\ \text{для любого }\ t\ge 0.
\]
\end{definition}
Это определение аналогично определению момента остановки в случае дискретного времени (см.~лекцию \ref{ch:american-discrete}).

\begin{definition}
Если $X$ "--- случайный процесс, а $\tau$ "--- момент остановки, то процесс $X^{\tau} = (X_{t\wedge \tau})_{t\ge 0}$ называется \emph{остановленным процессом} (здесь обозначается $t\wedge \tau = \min(t,\tau)$). 
\end{definition}

\begin{definition}
Согласованный случайный процесс $X = (X_t)_{t\ge0}$ с $\E|X_0|<\infty$ называется \emph{локальным мартингалом} относительно фильтрации $\FF$, если существует последовательность моментов остановки $\tau_n$, называемая \emph{локализующей последовательностью}, такая, что
\begin{alphenum}
\item $\tau_{n+1}\ge \tau_n$ \as\ для всех $n$,
\item $\lim\limits_{n\to\infty} \tau_n = \infty$ \as,
\item $X^{\tau_n}$ является мартингалом относительно $\FF$ для каждого $n$.
\end{alphenum}
Если у процесса $X=(X_t)_{t\in[0,T]}$ горизонт времени конечен, то в свойстве 2 достаточно требовать, чтобы для почти каждого  $\omega$ нашелся номер $N$ такой, что $\tau_n(\omega) = T$ для всех $n\ge N$.
\end{definition}

\begin{remark}
Любой мартингал является локальным мартингалом.
\end{remark}

\begin{proposition}
Если $H \in \mathcal{P}^2_T$ (или $H \in \mathcal{P}^2_\infty$), то процесс $\int_0^t H_s dW_s$ является локальным мартингалом.
\end{proposition}

Следующий результат будет использован в одной из дальнейших лекций.

\begin{proposition}
Если процесс $X$ является локальным мартингалом и $X_t\ge c$ для всех $t\in[0,T]$ (или для всех $t\ge 0$), где $c$ "--- константа, то $X$ является супермартингалом.
\end{proposition}

\begin{proof}
Пусть $\tau_n$ "--- локализующая последовательность.
Тогда по мартингальному свойству остановленного процесса имеем $\E (X_{\tau_n\wedge t} \mid \F_s) = X_{\tau_n\wedge s}$ для $t\ge s$.
Перейдем к пределу $n\to\infty$. Так как $\tau_n\to\infty$, то $X_{\tau_n\wedge s} \to X_s$ и $X_{\tau_n\wedge t} \to X_t$ \as\ 
Пользуясь условной леммой Фату%
\footnote{Если $\xi^n\ge \zeta$, где $\E\zeta^- < \infty$, то $\E (\liminf\limits_{n\to\infty} \xi_n \mid \G) \le \liminf\limits_{n\to\infty} \E(\xi_n \mid \G)$.},
получаем
\[
\E (X_t \mid \F_s) = \E(\lim\limits_{n\to\infty} X_{\tau_n\wedge t} \mid \F_s) \le \liminf_{n\to\infty} \E(X_{\tau_n\wedge t} \mid \F_s) = \liminf_{n\to\infty} X_{\tau_n\wedge s} = X_s.
\]
Таким образом, выполнено супермартингальное свойство.
Взяв $s=0$ и учитывая неотрицательность $X_t$, получаем $0 \le \E X_t \le \E X_0$ и, следовательно, $\E|X_t| < \infty$.
\end{proof}


\section{Формула Ито}
\subsection{Формулировка и примеры}

\begin{theorem}[формула Ито]
Пусть функция $f(t,x)$ непрерывна и имеет непрерывные частные производные $f'_t(t,x)$, $f'_x(t,x)$ и $f''_{xx}(t,x)$. Тогда
\[
f(t, W_t) = f(0,0) + \int_0^t \left(f'_t(s,W_s) + \frac12 f''_{xx}(s,W_s)\right) ds + \int_0^t f'_x(s,W_s) dW_s,
\]
где первый интеграл понимается в смысле Римана потраекторно%
\footnote{Для каждого $\omega$ его значение равно интегралу функции $s\mapsto f'_t(s,W_s(\omega)) + \frac12 f''_{xx}(s,W_s(\omega))$.},
а второй понимается в смысле Ито.
\end{theorem}

\begin{remark}
1. Оба интеграла в формуле Ито корректно определены.
Для интеграла по $dt$ это так, потому что непрерывные функции на отрезке интегрируемы по Риману, а для интеграла по $dW_t$ под интегралом стоит согласованный непрерывный процесс, \te\ процесс из пространства $\mathcal{P}^2_\infty$.

2. Формулу Ито обычно записывают в \emph{дифференциальной форме}
\[
d f(t,W_t) = f'_t(t,W_t) dt + f'_x(t,W_t) dW_t + \frac12 f''_{xx}(t,W_t) dt.
\]
В частности, если функция $f=f(x)$ не зависит от $t$, то формула принимает вид
\[
d f(W_t) = f'(W_t) dW_t + \frac12 f''(W_t) dt.
\]
Подчеркнем, что дифференциальная форма "--- это всего лишь удобный способ записи, так как $dW_t$ не является дифференциалом в обычном смысле.
\end{remark}

\begin{example}
\label{7:e:WdW}
Пусть $f(x) = x^2$.
Подставляя в формулу Ито, получаем
\[
d W_t^2 = 2 W_t d W_t + dt.
\]
Переписывая это соотношение в интегральной форме, находим интеграл от броуновского движения по броуновскому движению:
\[
\int_0^t W_s d W_s = \frac12 (W_t^2 - t).
\]

Можно заметить отличие интеграла Ито от интеграла Римана или Лебега: для любой дифференцируемой функции $w(t)$ с $w(0)=0$ имеем $\int_0^t w(s) d w(s) = \frac12 w(t)^2$, и поправка $-\frac12 t$ не появляется.
\end{example}

\begin{example}
Пусть $X$ "--- геометрическое броуновское движение, \te
\[
X_t = x_0e^{\sigma W_t + (\mu - \frac12 \sigma^2)t}, \qquad x_0>0.
\]
Применяя формулу Ито к функции $f(t,x) = x_0 e^{\sigma x + (\mu - \frac12 \sigma^2)t}$, получаем
\[
d X_t = \mu X_t dt + \sigma X_t d W_t.
\]
Здесь процесс $X_t$ входит и в левую, и в правую часть.
Таким образом, мы получили \emph{стохастическое дифференциальное уравнение}, решением которого является геометрическое броуновское движение.
Более подробно о стохастических дифференциальных уравнениях будет рассказано в следующей лекции.
\end{example}


\subsection{\difficult\ Доказательство}

Приведем доказательство формулы Ито только для дважды непрерывно дифференцируемых функций $f(x)$, у которых производная $f'(x)$ ограничена на всей прямой.
Полное доказательство можно найти в главе~4 книги \cite{LiptserShiryaev74}.

\begin{lemma}
\label{7:l:p-convergence}
Пусть $H\in \L^2_t$ является непрерывным процессом.
Рассмотрим разбиение $t_i^n = i\frac tn$, $i=0,\dots,n$. 
Тогда
\[
\sum_{i=0}^{n-1} H_{t_i^n}(W_{t_{i+1}^n}-W_{t_i^n})^2 \xrightarrow[n\to\infty]{\P} \int_0^t H_s ds.
\]
\end{lemma}

\begin{proof}
Введем последовательности случайных величин 
\[
Z^n = \sum_{i=0}^{n-1} H_{t_i^n}(W_{t_{i+1}^n}-W_{t_i^n})^2,\qquad
Y^n = \sum_{i=0}^{n-1} H_{t_i^n}(t_{i+1}^n- t_i^n).
\]
Заметим, что, как следует из конструкции интеграла Римана,
\[
Y^n \xrightarrow[n\to\infty]{\text{п.н.}} \int_0^t H_s ds.
\]
Поэтому достаточно будет показать, что $Z^n-Y^n \xrightarrow{\P} 0$.
Имеем
\[
Z^n-Y^n = \sum_{i=0}^{n-1} H_{t_i^n}((W_{t_{i+1}^n} - W_{t_i^n})^2 - (t_{i+1}^n - t_i^n)).
\]
Из примера~\ref{7:e:WdW} следует, что $(W_s-W_r)^2 - (s-r) = 2\int_r^s W_u dW_u - W_r(W_s-W_r)$. 
Тогда получаем
\[
Z^n - Y^n 
= 2\sum_{i=0}^{n-1} H_{t_i^n} \biggl(
  \int_{t_i^n}^{t_{i+1}^n} W_s dW_s - W_{t_i^n}(W_{t^n_{i+1}}-W_{t_i^n})
\biggr) 
= 2\int_0^t H^n_s(W_s-W^n_s) dW_s,
\]
где
\[
H^n_t = \sum_{i=0}^{n-1} H_{t_i^n} \I(t\in(t_i^n,t_{i+1}^n]), \qquad
W^n_t = \sum_{i=0}^{n-1} W_{t_i^n} \I(t\in (t_i^n,t_{i+1}^n]).
\]
Пользуясь теоремой Фубини и независимостью приращений броуновского движения, находим
\[
\E \int_0^t (H_s^n (W_s - W_s^n))^2 ds = \int_0^t \E (H_s^n)^2 \E(W_s - W_s^n))^2 ds \\
\le \frac tn \int_0^t \E (H_s^n)^2 ds \to 0.
\]
Таким образом, $H^n(W-W^n) \to 0$ в $\L^2_t$, что влечет сходимость $Z^n-Y^n \to 0$ в $L^2$ и, следовательно, по вероятности.
\end{proof}

Перейдем к доказательству формулы Ито.
Для разбиения $t_i^n = i\frac tn$ имеем (опуская индекс $n$ для краткости)
\[
f(W_t) = f(0) + \sum_{i=0}^{n-1} (f(W_{t_{i+1}}) - f(W_{t_i})).
\]
Применяя формулу Тейлора с остаточным членом в форме Лагранжа, получаем
\begin{equation}
\label{7:taylor}
f(W_t) = f(0) + \sum_{i=0}^{n-1} \biggl\{f'_x(W_{t_i})(W_{t_{i+1}} - W_{t_i})
+ \frac12 f''_{xx}(W_{\tau_i})(W_{i_{i+i}}-W_{t_i})^2 \biggr\} 
\end{equation}
для некоторых точек $\tau_i \in[t_i,t_{i+1}]$, зависящих от $\omega$.
Обозначим за $X^n$ простой процесс
\[
X^n_s = \sum_{i=0}^{n-1} f'_x(W_{t_i}) \I(s\in(t_i,t_{i+1}]).
\]
Заметим, что 
\[
\sum_{i=0}^{n-1} f'_x(W_{t_i})(W_{t_{i+1}} - W_{t_i}) = \int_0^t X_s^n d W_s.
\]
Так как $f'(x)$ непрерывна, то $X^n_s(\omega) \to f'_x(W_s(\omega))$ для всех $t$ и $\omega$. Тогда в силу предположения об ограниченности $f'(x)$ имеет место сходимость $X^n\to f'_x(W)$ в $\L^2_t$, \te\ $\E \int_0^t (X_s^n - f'_x(W_s))^2 ds \to 0$ (по теореме Лебега о мажорируемой сходимости).
Следовательно,
\[
\sum_{i=0}^{n-1} f'_x(W_{t_i})(W_{t_{i+1}} - W_{t_i}) \xrightarrow[n\to\infty]{L^2} \int_0^t f'_x(W_s) d W_s.
\]
Далее преобразуем второе слагаемое под суммой в формуле \eqref{7:taylor}.
Положим
\[
\epsilon_i = f''_{xx}(W_{\tau_{i+1}}) - f''_{xx}(W_{t_i}).
\]
Так как производные функции $f$ и траектории броуновского движения непрерывны, то для фиксированного $\omega$ функция $f'_x(W_t(\omega))$ непрерывна по $t$ и, следовательно, равномерно непрерывна по $t$ на любом отрезке.
Тогда имеем $\sup_i |\epsilon_i|\to 0$ \as\ при $n\to\infty$.
Отсюда получаем
\[
\Biggl|\sum_{i=0}^{n-1} \epsilon_i (W_{t_{i+1}} - W_{t_i})^2\Biggr|
\le \sup_i |\epsilon_i| \sum_{i=0}^{n-1} (W_{t_{i+1}} - W_{t_i})^2
\xrightarrow{\P} 0,
\]
где воспользовались тем, что $\sum_{i=0}^{n-1} (W_{t_{i+1}^n}-W_{t_i^n})^2 \to t$ по вероятности, как следует из леммы~\ref{7:l:p-convergence}.
Следовательно, опять пользуясь леммой~\ref{7:l:p-convergence}, получаем
\[
\sum_{i=0}^{n-1} f''_{xx}(W_{\tau})(W_{i_{i+i}}-W_{t_i})^2
= \sum_{i=0}^{n-1} (f''_{xx}(W_{t_i}) + \epsilon_i)(W_{i_{i+i}}-W_{t_i})^2
\xrightarrow{\P} \int_0^t f''_{xx}(W_s) ds.
\]


\summary
\begin{itemize}
\item Интеграл Ито $\int_0^t H_s d W_s$ от измеримого согласованного процесса $H$ по броуновскому движению $W$ определяется в три этапа: сначала для простых процессов $H$, затем для $H\in\L^2_T$ (что означает $\E \int_0^T H_t^2 dt < \infty$), и, наконец, для $H\in\PP^2_T$ ($\int_0^T H_t^2 dt < \infty$ \as).

\item Если $H\in \L^2_T$, то интеграл Ито является квадратично интегрируемым мартингалом с нулевым средним, причем  $\E(\int_0^t H_s dW_s \cdot \int_0^t G_s d W_s) = \E\int_0^t H_sG_s ds$ для $H,G\in\L^2_T$, где $t\le T$ (изометрия Ито).
Если $H\in\PP^2_T$, то это, в общем случае, не верно.

\item Для любой функции $f(t,x)$ с непрерывными частными производными $f'_t$ и $f''_{xx}$ справедлива формула Ито:
\[
d f(t,W_t) = f'_t(t,W_t) dt + f'_x(t,W_t) dW_t + \frac12 f''_{xx}(t,W_t) dt.
\]
\end{itemize}
