%!TEX root=finmath1.tex
\part{Практика}
\titleformat{\chapter}[display]{\bfseries\Large}{\underline{Практикум~\thechapter}}{0.5em}{}
\titlecontents{chapter}[0em]{}{\textbf{Практикум\ \thecontentslabel.}\hspace{2mm}}{}{\dotfill\contentspage}
\setcounter{chapter}{0}
\renewcommand{\theHchapter}{P\arabic{chapter}}%

\section*{Общие замечания}
Эта часть курса содержит введение в программирование финансовых моделей на языке Python.
У нее две цели: во-первых, проиллюстрировать теоретические результаты основной части курса, а, во-вторых, разобрать некоторые приемы программирования на Python, которые полезно применять в задачах, где требуются интенсивные вычисления.
Предполагается, что читатель уже знаком с основами Python.

Большинство дальнейших примеров будут продублированы в прилагаемом коде, который разделен на две части: мини-пакет \verb"finmath1" и несколько ноутбуков Jupyter.
В \verb"finmath1" содержатся универсальные методы, которые можно использовать как самостоятельную библиотеку.
Ноутбуки содержат конкретные примеры.
Имя файла, содержащего соответствующий фрагмент кода, будет указываться в комментарии в первой строке.

Для краткости мы не будем каждый раз описывать необходимые импорты. Если не оговорено иного, считается, что сделано следующее%
\footnote{Чтобы \verb"finmath1" корректно импортировался, скопируйте его в рабочую директорию.}.
\begin{python}
from math import *
import numpy as np
import scipy.stats as st
import matplotlib.pyplot as plt
from finmath1 import *
\end{python}

В некоторых примерах, где приводится результаты работы кода, вывод результатов может незначительно отличаться от того, что получится, если запустить код в реальности. Например, мы будем округлять десятичные дроби (для экономии места), элементы массивов будут в тексте разделяться запятой, а не пробелом, и \tp

\begin{remark*}
Поясним, почему в качестве языка выбран Python, хотя <<кванты>> (количественные аналитики) традиционно используют C++.
Преимущества Python заключаются в простоте и скорости написания кода, а также в наличии развитой экосистемы пакетов для численных расчетов и обработки данных (NumPy, SciPy, Pandas и др.).
Благодаря этому Python удобен для прототипирования финансовых моделей, а также в исследовательских применениях, где скорость работы кода не так важна по сравнению с возможность легко испытывать разные алгоритмы. 
\end{remark*}

%Список пакетов, необходимых для работы кода, содержится в прилагаемом файле \verb"requirements.txt" (его удобно использовать с утилитой \verb"pip").
