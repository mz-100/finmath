\addcontentsline{toc}{chapter}{\textbf{О чем этот курс}}
\chapter*{О чем этот курс}

Курс посвящен математической теории \emph{оценивания производных финансовых инструментов} таких как опционы, фьючерсы и \tp\ 
Основная его цель "--- познакомить с фундаментальными идеями, используемыми в этой области, и изучить их на примере базовых моделей.
Этот курс в разное время читался и продолжает читаться в Высшей школе экономики на факультетах МИЭФ и ФКН, а также в МГУ на механико-математическом факультете и в рамках программы Института <<Вега>>. 

В первой части курса будут рассмотрены модели с дискретным временем, в во второй части "--- модели с непрерывным временем.
При изучении моделей с непрерывным временем мы также обсудим основы стохастического исчисления, которые нам потребуются: интеграл Ито, формула Ито, стохастические дифференциальные уравнения и \tp\ 
Эти понятия интересны и важны сами по себе.
Курс завершается на модели \bs\ и ее вариантах.

Стоит отметить, что модели, изучаемые в этом курсе, слишком просты, чтобы применять их на практике, однако они задают фундамент для продвинутой теории, которая излагается в курсах <<Модели стохастической волатильности>> (<<Финансовая математика --- 2>>) и <<Финансовая математика --- 3>>\footnote{Курс <<Финансовая математика --- 3>> пока не читается, но когда-нибудь будет.}.

Курс рассчитан на один семестр. Часть материала оставлена для самостоятельного изучения и не входит в экзамен; она вынесена в раздел <<Дополнения>>.
Также по ходу изложения некоторые разделы отмечены звездочками "--- это дополнительный материал, который более труден, чем средний уровень курса.

%Раздел <<Практикум>> посвящен решению задач по финансовой математике на языке Python и может служить первой частью самостоятельного курса по программированию финансовых моделей.

Для полного освоения курса желательно иметь уверенное знания теории вероятностей и наличие  базовых представлений о финансовых рынках.
%Для освоения практической части требуется умение программировать на языке Python.
