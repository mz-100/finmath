%!TEX root=finmath1.tex
\chapter{Модель Башелье}
\label{ch:bachelier}

Модель Башелье "--- это первая модель финансового рынка (1900 г.)%
\footnote{Модель была предложена в диссертации Л.~Башелье \cite{Bachelier00}.
Долгое время, до 1960-х гг., большинству экономистов она была не известна, хотя в классических работах по теории вероятностей, в т.\,ч.~в работах А.\,Н.~Колмогорова, результаты Башелье упоминаются.}.
Говоря современным языком, процесс цены рискового актива в ней задается броуновским движением со сносом.
В этом дополнении дается описание модели Башелье и доказывается формула для цены европейских опционов колл и пут, аналогичная формуле \bs.

\section{Описание модели}
Структура модели во многом схожа с моделью \bs, поэтому мы опустим подробные пояснения.

Будем считать заданным некоторое фильтрованное вероятностное пространство $(\Omega,\F,\FF,\P)$, в котором фильтрация $\FF=(\F_t)_{t\in[0,T]}$ порождена броуновским движением $W=(W_t)_{t\in[0,T]}$ и пополнена.

В модели имеются два актива.
Безрисковый актив имеет цену $B_t\equiv 1$, \te\ процентная ставка равна 0, а цена рискового актива задается процессом
\[
S_t = s_0 + \mu t + \sigma W_t.
\]
Отметим, что цена рискового актива может стать отрицательной,
это один из недостатков модели%
\footnote{Однако отметим, что возможность модели Башелье учитывать отрицательные цены оказалась полезной в апреле 2020 г., когда цены поставочных фьючерсов на нефть стали отрицательными из-за отсутствия свободных емкостей для принятия поставок.
В частности, Чикагская товарная биржа на некоторое время сменила используемую модель для расчета цен опционов на фьючерсы с модели Блэка на модель Башелье.}.
% https://www.cmegroup.com/notices/clearing/2020/04/Chadv20-171.html

Торговой стратегией будем называть процесс $\pi_t=(G_t,H_t)$ с компонентами $G\in \PP_T^1$ и $H\in\PP_T^2$.
Стоимость портфеля стратегии $\pi$ выражается процессом
\[
V_t^\pi = G_t + H_tS_t.
\]
Стратегия называется самофинансируемой, если
\[
d V_t^\pi = H_t dS_t.
\]

Следующее предложение и следствие из него доказываются так же, как в модели \bs.

\begin{proposition}
В модели Башелье существует единственная вероятностная мера $\Q\sim\P$ такая, что процесс процесс $S$ является мартингалом относительно $\Q$.
Более того, справедливо представление $S_t = s_0 + \sigma W_t^\Q$, где $W_t^\Q = W_t - \mu t/\sigma$ "--- броуновское движение относительно $\Q$.

Стоимость портфеля любой самофинансируемой стратегии $V_t^\pi$ является локальным мартингалом относительно $\Q$, а если стратегия допустима в том смысле, что $H\in \L^2_T(\Q)$, то $V_T^\pi$ является квадратично интегрируемым мартингалом.
\end{proposition}

\begin{corollary}
В модели Башелье нет арбитража.
\end{corollary}

\section{Цены платежных обязательств}
Будем отождествлять платежные обязательства европейского типа с $\F_T$-измеримыми случайными величинами $X$ такими, что $\E^\Q X^2 < \infty$.  

Следующее предложение вытекает из теоремы о мартингальном представлении и того факта, что стоимость любой допустимой самофинансируемой стратегии является мартингалом относительно ЭММ.

\begin{proposition}
В модели Башелье любое платежное обязательство  является реплицируемым, причем реплицирующая стратегия единственна.
Цена репликации (стоимость реплицирующего портфеля) равна
\[
V_t^X = \E^\Q (X \mid \F_t).
\]
\end{proposition}

Если платежное обязательство зависит только от значения цены в последний момент времени, $X=f(S_T)$, то его цену можно найти вычислением условного математического ожидания, пользуясь тем, броуновскогое движение "--- это марковский процесс с известной переходной плотностью. А именно, $V_t^X = V(t,S_t)$ с функцией
\[
V(t,s) = \int_\R f(s + \sigma x\sqrt{T-t}) \phi(x) dx.
\]

\begin{corollary}[формула Башелье]
Для цен европейских опционов колл и пут в модели Башелье справедлива формула
\[
\begin{aligned}
&\VC(t,s) = (s-K)\Phi(d) + \sigma \sqrt{T-t}\phi(d),\\
&\VP(t,s) = (K-s)\Phi(-d) + \sigma\sqrt{T-t}\phi(d),
\end{aligned}
\]
где $d = (s-K) /(\sigma\sqrt{T-t})$.
\end{corollary}

% \begin{corollary}
% Пусть $x=f(S_T)$, где $|f(s)|  \le  ce^{m|s|}$ для некоторых констант $c,m$.
% Тогда цена репликации платежного обязательства $X$ представима в виде $V_t^X = V(t,S_t)$ с функцией
% \[
% V(t,s) = \int_\R f(s + \sigma x\sqrt{T-t}) \phi(x) dx
% \]
% и удовлетворяет уравнению с частными производными
% \[
% \left\{\begin{aligned}
% &V'_t(t,s) + \frac{\sigma^2}{2} V''_{ss}(t,s) = 0, \qquad t\in[0,T),\ s\in \R,\\
% &V(T,s) = f(s), \qquad s\in\R,
% \end{aligned}
% \right.
% \]
% а компонента $H$ реплицирующей стратегии задается формулой $H_t = V'_s(t,S_t)$.
% \end{corollary}
