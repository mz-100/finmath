%!TEX root=finmath1.tex
\chapter{Общая модель рынка в дискретном времени}
\label{ch:general}
\chaptertoc

В этой лекции рассматривается общая модель рынка в дискретном времени.
Две ключевых идеи такие.
Во-первых, безарбитражность рынка эквивалентна существованию \emph{эквивалентной мартингальной меры}.
Во-вторых, \emph{безарбитражные цены} деривативов нужно вычислять как математические ожидания дисконтированных выплат по эквивалентным мартингальным мерам.


\section{Описание модели}

Приводимое далее описание модели идейно повторяет рассуждения для модели \crr, поэтому мы не будем подробно пояснять каждый шаг "--- пояснения можно найти в главе~\ref{ch:crr}.

Рынок функционирует в моменты времени $t=0,\dots,T$.
Случайность описывается фильтрованным вероятностным пространством $(\Omega,\F,\FF,\P)$.
Будем считать, что $\F_0 = \{\emptyset,\Omega\}$.

Ны рынке имеются $N+1$ активов: один безрисковый актив и $N$ рисковых активов.
Цена безрискового актива задается положительной предсказуемой случайной последовательностью $B=\{B_t\}_{t=0}^T$, где $B_0=1$.
Цена $n$-го рискового актива задается согласованной случайной последовательностью $S^n=\{S^n_t\}_{t=0}^T$.
Допускается, что цены $S_t^n$ могут быть отрицательными, но при этом мы будем считать, что они ограничены снизу, \te\ найдется константа $c$ такая, что $S_t^n\ge c$ \as\ для всех $t,n$.

Торговыми стратегиями являются предсказуемые последовательности $\pi = \{\pi_t\}_{t=1}^T$ вида $\pi_t=(G_t, H_t)$, где компонента $G_t$ является скалярной и выражает количество единиц безрискового актива, купленных в момент $t-1$, а компонента $H_t$ векторная (размерности $N$),    где $H_t^n$ выражает количество единиц $n$-го рискового актива, купленных в момент $t-1$. 

Стоимостью портфеля стратегии $\pi$ называется последовательность $V^\pi = \{V_t^\pi\}_{t=0}^T$, определяемая следующим образом:
\begin{align}
\label{gen:V0}
&V_0^\pi = G_1B_0 + \scal{H_1}{S_0},\\
\label{gen:Vt}
&V_t^\pi =  G_tB_t + \scal{H_t}{S_t},
  \qquad t=1,\dots,T,
\end{align}
где точка обозначает скалярное произведение, \te\ $\scal{H_t}{S_t} = \sum_{n=1}^N H_t^nS_t^n$.

Дисконтированными ценами рисковых активов называются последовательности $\tilde S_t^i = S_t^i/B_t$. 
Дисконтированной стоимостью портфеля стратегии $\pi$ называется последовательность $\tilde V_t^\pi = V_t^\pi/B_t$. 

Стратегия $\pi$ называется самофинансируемой, если для всех $t=1,\dots,T-1$ выполнено равенство
\[
(H_{t+1} - H_t)\cdot S_t = -(G_{t+1} - G_t)B_t.
\]

Следующее представление для стоимости портфеля самофинансируемой стратегии будет играть важную роль в дальнейших рассуждениях.
Из него, в частности, будет следовать, что $\tilde V_t^\pi$ является мартингальным преобразованием относительно так называемой эквивалентной мартингальной меры, вводимой в следующем разделе.

\begin{proposition}
Справедливы следующие утверждения.

\label{gen:p:sf}
1. Стратегия $\pi$ является самофинансируемой тогда и только тогда, когда
\[
V_t^\pi = V_0^\pi + \sum_{u=1}^t (G_u\Delta B_u + \scal{H_u}{\Delta S_u}),
\]
где $\Delta B_t = B_{t} - B_{t-1}$ (скалярная величина) и $\Delta S_t = S_t - S_{t-1}$ (вектор).

2. Стратегия $\pi$ является самофинансируемой тогда и только тогда, когда
\begin{equation}
\label{gen:sf-discounted-value}
\tilde V_t^\pi = \tilde V_0^\pi + \sum_{u=1}^t \scal{H_u}{\Delta \tilde S_u},
\end{equation}
\end{proposition}

\begin{remark}
\label{gen:r:sf-construction}
Второе утверждение дает удобный способ задавать самофинансируемые стратегии: нужно задать только начальную стоимость $V_0^\pi$ и последовательность $H_t$, которые можно выбрать произвольным образом. 
Затем $G_t$ однозначно выражается из стоимости портфеля: $G_t = \tilde V_t^\pi - \scal{H_t}{\tilde S_t}$.
\end{remark}

\begin{proof}
1. Необходимость доказывается так же, как в предложении~\ref{crr:p:sf-value} в лекции \ref{ch:crr}.
Для доказательства достаточности предположим, что $V_t^\pi$ задается по формуле выше.
Тогда
\[
(G_{t+1} - G_t)B_t + (H_{t+1} - H_t)\cdot S_t = 
V_{t+1}^\pi - G_{t+1}\Delta B_{t+1} - \scal{H_{t+1}}{\Delta S_{t+1}} - V_t^\pi = 0,
\]
что влечет выполнение условия самофинансируемости.

2) Доказательство проводится таким же образом, как и в первой части, используя то, что условие самофинансируемости эквивалентно равенству
\[
(H_{t+1} - H_t)\cdot \tilde S_t = -(G_{t+1} - G_t).
\]
\end{proof}


\section{Безарбитражность рынка и эквивалентные мартингальные меры}

\begin{definition}
\emph{Арбитражной возможностью} называется самофинансируемая стратегия $\pi$ такая, что 
\begin{alphenum}
\item $V_0^\pi = 0$,
\item $\P(V_T^\pi > 0) > 0$,
\item $\P(V_T^\pi \ge 0)=1$.
\end{alphenum}
\emph{Гипотеза отсутствия арбитража} NA состоит в том, что в модели рынка нет арбитражных возможностей.
\end{definition}

Для формулировки следующего определения напомним, что рынок задан на фильтрованном вероятностном пространстве $(\Omega,\F,\FF,\P)$.
Вводимое в нем понятие обобщает понятие риск-нейтральной вероятности в модели Кокса--Росса--Рубинштейна.

\begin{definition}
\emph{Эквивалентной мартингальной мерой} (ЭММ) называется вероятностная мера $\Q$ на $(\Omega,\F)$ такая, что 
\begin{alphenum}
\item $\Q$ эквивалентна $\P$ (обозначение: $\Q\sim\P$), что означает, что для любого события $A\in\F$ имеем $\Q(A)=0$ тогда и только тогда, когда $\P(A)=0$,

\item дисконтированная цена $\tilde S_t^n$ является мартингалом на фильтрованном вероятностном пространстве $(\Omega,\F,\FF,\Q)$ для каждого $n=1,\dots,N$.
\end{alphenum}
\end{definition}

\begin{proposition}
\label{gen:p:discount-value}
Вероятностная мера $\Q\sim\P$ является ЭММ тогда и только тогда, когда дисконтированная стоимость портфеля любой самофинансируемой стратегии является мартингальным преобразованием относительно $\Q$.
\end{proposition}

\begin{proof}
Если $\Q$ является ЭММ, то из  формулы~\eqref{gen:sf-discounted-value} следует, что дисконтированная стоимость портфеля является мартингальным преобразованием (мы пользуемся тем, что каждое слагаемое в формуле \eqref{gen:sf-discounted-value} является мартингальным преобразованием, а сумма мартингальных преобразований "--- тоже мартингальное преобразование).

В обратную сторону: если дисконтированная стоимость любой самофинансируемой стратегии является мартингальным преобразованием, то рассмотрим стратегию $\pi_t \equiv (G,H)$ с $G=0$, $H=(0,\dots,1,\dots,0)$, портфель которой состоит только из одного рискового актива $n$. Ясно, что она самофинансируемая, а ее стоимость $V_t^\pi = S_t^n$. Следовательно, $\tilde S_t^n$ является мартингальным преобразованием, а, в силу предположения об ограниченности снизу, и настоящим мартингалом по предложению~\ref{mart:martingale-transform} в лекции \ref{ch:mart}.
\end{proof}

\begin{corollary}
Пусть $\Q$ является ЭММ, а $\pi$ является самофинансируемой стратегией с $V_T^\pi \ge c$ для некоторой константы $c$. Тогда последовательность $V_t^\pi$ является мартингалом.
\end{corollary}

\begin{proof}
Этот результат сразу вытекает из предыдущего предложения и предложения~\ref{mart:martingale-transform} в лекции \ref{ch:mart}.
\end{proof}

Следующая теорема играет очень важную роль в теоретической финансовой математике и поэтому называется \emph{фундаментальной}\footnote{Также она называется теоремой Даланга"--~Мортона"--~Виллинджера.}.

\begin{theorem}[первая фундаментальная теорема финансовой математики]
Отсутствие арбитража равносильно существованию эквивалентной мартингальной меры.
\end{theorem}
\begin{proof}
Пусть существует ЭММ $\Q$. Предположим, от противного, что имеется арбитражная возможность $\pi$. Тогда
\[
0 = \tilde V_0^\pi = \E^Q \tilde V_T^\pi > 0,
\]
где неравенство выполнено в силу того, что $V_T^\pi \ge 0$ и $\Q(V_T^\pi>0)>0$.
Получаем противоречие: $0>0$.
Следовательно, существование $\Q$ влечет отсутствие арбитражных возможностей. 
Доказательство в обратную сторону гораздо труднее и приводится в дополнении \ref{ch:ftap-proof}.
\end{proof}

\begin{example}
В модели \crr\ с параметрами $d<r<u$ эквивалентная мартингальная мера (которую мы раньше называли риск-нейтральной вероятностью) задается формулой 
\[
Q(\omega) = q^{\sum_{t=1}^T \omega_t}(1-q)^{T-\sum_{t=1}^T \omega_t}, \qquad 
\omega=(a_1,\dots,a_T),\ a_t=0\text{ или }1,
\]
где $q$ "--- риск-нейтральная вероятность того, что цена идет вверх за один шаг:
\[
q = \frac{r-d}{u-d}.
\]

Чтобы соблюсти формальное определение ЭММ, вообще говоря, нужно сначала задать вероятностное пространство $(\Omega,\F,\FF,\P)$.
В качестве $\Omega$ возьмем, как и раньше, множество последовательностей $(a_1,\dots,a_T)$ из нулей и единиц.
На нем определим последовательность случайных величин $\xi_1,\dots,\xi_T$, где $\xi_t(\omega) = 1+u$, если в исходе $\omega=(a_1,\dots,a_T)$ имеем $a_t=1$, и $\xi_t(\omega) = 1+d$, если $a_t=0$.
В качестве фильтрации $\FF$ возьмем фильтрацию, порожденную этой последовательностью. 
Вероятностную меру $\P$ можно взять произвольным образом, лишь бы вероятность каждого исхода $\omega$ была строго положительной.

Покажем, что $Q$, определенная выше, действительно является ЭММ.
Во-первых, $\Q\sim\P$, так как относительно обеих мер все исходы имеют строго положительные вероятности, и, следовательно, $\Q(A) = 0$ и $\P(A)=0$ только если $A=\emptyset$.

Далее заметим, что $S_{t+1} = S_t\xi_{t+1}$, причем величины $\xi_t$ независимы и одинаково распределены относительно $\Q$ (проверьте это самостоятельно). 
Тогда $\tilde S_{t+1} = \tilde S_t \xi_{t+1}/(1+r)$, и получаем
\begin{multline*}
\E^{\Q} (\tilde S_{t+1} \mid \F_t)
= \frac{\tilde S_t}{1+r} \E^{\Q} (\xi_{t+1} \mid \F_t)
= \frac{\tilde S_t}{1+r} \E^{\Q} (\xi_{t+1}) \\
= \frac{\tilde S_t}{1+r} ((1+u)q + (1+d)(1-q))
= \tilde S_t.  
\end{multline*}
Таким образом, $\tilde S_t$ является мартингалом относительно $\Q$.
\end{example}

\begin{example}
Рассмотрим многошаговую \emph{триномиальную модель} с параметрами $d<c<u$ и $r\in(d,u)$.
В этой модели два актива, безрисковый и рисковый.
Цена безрискового равна $B_t = (1+r)^t$.
Цена рискового на каждом шаге может измениться в $1+d$, $1+c$ или $1+u$ раз.
Таким образом, $S_{t+1} = S_t\xi_{t+1}$, где случайная величина $\xi_{t+1}$ принимает значения $1+d$, $1+c$ и $1+u$ с какими-то (положительными) вероятностями.
Вероятностное пространство и фильтрацию, аналогично биномиальной модели, можно задать в виде $\Omega = \{(a_1,\dots,a_T) \mid a_i\in\{-1,0,1\}\}$, $\F = 2^\Omega$, $\FF$ "--- фильтрация, порожденная последовательностью $\xi_t$. 

В такой модели ЭММ существует, но не единственна.
Например, в качестве ЭММ можно взять
\[
Q(\omega) = q_+^{\sum_{t=1}^T \I(\omega_t=1)} 
\cdot q_0^{\sum_{t=1}^T \I(\omega_t=0)} 
\cdot q_-^{\sum_{t=1}^T \I(\omega_t=-1)},
\]
где $q_+$, $q_0$, $q_-$ являются решением (не единственным) системы уравнений
\[
\left\{\begin{aligned}
  &q_+>0,\quad q_0>0,\quad q_- > 0,\\
  &q_++q_0+q_- = 1,\\
  &(1+u)q_+ + (1+c)q_0 + (1+d) q_- = (1+r).
\end{aligned}
\right.
\]
По своему смыслу, $q_+$ "--- это риск-нейтральная вероятность того, что цена пойдет <<вверх>> за один период, $q_-$ "--- пойдет <<вниз>>, а $q_0$ "--- пойдет по <<средней>> траектории.
Нетрудно проверить, что для таких вероятностей $\E \xi_t = 1+r$, и, аналогично предыдущему примеру,  $\tilde S_t$ является мартингалом относительно $\Q$.
\end{example}


\section{Цены европейских платежных обязательств}
\subsection{Общие формулы}

\begin{definition}
\emph{Платежное обязательство европейского типа} отождествляется с $\F_T$-измеримой случайной величиной $X$, которая задает выплату, производимую продавцом платежного обязательства покупателю в момент $T$.
\end{definition}

\begin{definition}
\emph{Безарбитражной ценой} платежного обязательства $X$ называется согласованная последовательность $V^X = (V_t^X)_{t=0}^T$ такая, что $V_T^X = X$ и расширенная модель рынка, заданная на том же фильтрованном вероятностном пространстве и содержащая безрисковый актив $B_t$, рисковые активы $S_t^n$, $n=1,\dots,N$ и новый рисковый актив с ценой $S_t^{N+1} = V_t^X$, является безарбитражной.
\end{definition}

Смысл этого определения в том, что мы предполагаем, что рынке начинает торговаться новое платежное обязательство $X$.
Тогда его цена не должна привносить арбитраж на рынок.

\begin{theorem}
\label{gen:t:price}
Пусть платежное обязательства $X$ ограничено снизу (т.е.\ $X\ge c$ для некоторой константы $c$). 
Тогда согласованная последовательность $V_t$ является безарбитражной ценой $X$ тогда и только тогда, когда для некоторой ЭММ $Q$ выполнено равенство
\begin{equation}
\label{gen:na-price}
V_t =  B_t \E^{\Q} \left( \frac{X}{B_T} \;\bigg|\; \F_t \right).
\end{equation}
\end{theorem}
Как следствие из этой теоремы, получаем, что безарбитражная цена платежного обязательства в момент времени $t=0$ задается формулой
\[
V_0^X = \E^{\Q} \left( \frac{X}{B_T} \right).
\]

\begin{proof}
Если $V_t$ определена по формуле \eqref{gen:na-price}, то дисконтированная цена $\tilde V_t = V_t/B_t$ является $\Q$-мартингалом (мартингалом Леви), и, следовательно, $\Q$ является ЭММ в расширенной модели.
Тогда по фундаментальной теореме в расширенной модели нет арбитража.

Если $V_t$ является безарбитражной ценой, то в расширенной модели существует ЭММ $\Q$, причем она также будет и ЭММ для исходной модели.
Следовательно, $V_t$ должна удовлетворять \eqref{gen:na-price}, так как $\tilde V_t$ является $\Q$-мартингалом.
\end{proof}

Итак, формула \eqref{gen:na-price} дает способ вычисления безарбитражной цены платежного обязательства.
В случае, когда ЭММ не единственна, безарбитражная цена, вообще говоря, тоже не единственна. 
Какую из безарбитражных цен в этом случае выбрать?
На практике поступают так: в модели надо найти такую ЭММ, которая бы давала наилучшее соответствие между ценами деривативов, вычисленными в модели, и рыночными ценами этих деривативов (часто в качестве деривативов рассматривают ванильные опционы со всевозможными страйками и моментами экспирации).
Затем цены остальных деривативов вычисляются как математические ожидания дисконтированных выплат по этой ЭММ.

Поясним теперь соответствие между понятием цены репликации, которое у нас было в предыдущих лекциях, и безарбитражной цены.

\begin{definition}
Платежное обязательство $X$ называется \emph{реплицируемым}, если существует самофинансируемая стратегия $\pi$ такая, что $V_T^\pi = X$.
\end{definition}

\begin{remark}
Как было доказано, в модели КРР любое платежное обязательство реплицируемо.
В общем случае это не так.
Например, уже в триномиальной модели могут существовать нереплицируемые обязательства.
В качестве примера рассмотрим одношаговую триномиальную модель с параметрами $u=0.1$, $c=0$, $d=-0.1$, $r=0$, $S_0=100$ и опцион колл с моментом исполнения $T=1$ и страйком $K=100$.
Тогда система уравнений репликации 
\[
\left\{
\begin{aligned}
&G_1 + H_1\cdot 110 = 10,\\
&G_1 + H_1\cdot 100 = 0,\\
&G_1 + H_1\cdot 90 = 0
\end{aligned}
\right.
\]
не имеет решения.
\end{remark}

\begin{proposition}
\label{gen:p:price}
Пусть платежное обязательство $X$ реплицируемо и ограничено снизу.
Тогда для любой реплицирующей стратегии $\pi$ и любой ЭММ $Q$ выполнено
\[
V_t^\pi = B_t\E^{\Q} \left( \frac{X}{B_T} \;\bigg|\; \F_t \right).
\]
Как следствие, цена репликации реплицируемого платежного обязательства однозначно определена и является его единственной безарбитражной ценой.
\end{proposition}

\begin{proof}
Если $\pi$ и $\pi'$ "--- две реплицирующие стратегии для одного и того же платежного обязательства, то стоимости их портфелей должны совпадать.
Действительно, иначе в первый момент времени, когда они различаются, можно купить тот портфель, что дешевле, а продать тот, что дороже; в момент $T$ портфели компенсируют друг друга, и разность цен будет безрисковым доходом (упражнение: покажите это строго).
Следовательно, имелась бы арбитражная возможность, что противоречит существованию ЭММ.

Так как $\tilde V_t^\pi$ "--- мартингал относительно ЭММ $Q$, то $\tilde V_t^\pi = \E^Q(\tilde V_T^\pi \mid \F_t)$ и, следовательно, 
\[
\frac{V_t^\pi}{B_t} = \E^Q\left(\frac{X}{B_T} \;\bigg|\; \F_t\right),
\]
где воспользовались тем, что $V_T^\pi = X$.
\end{proof}

\subsection{Примеры}
\begin{example}[бескупонная облигация]
Бескупонную облигацию можно отождествить с константным платежным обязательством $X$ (номиналом облигации), выплачиваемым в момент погашения $T$.
Тогда, согласно теореме \ref{gen:t:price}, ее цена  имеет вид
\[
V_t = XB_t \E^{\Q}(B_T^{-1} \mid \F_t)  = X B(t,T),
\]
где для цены бескупонной облигации с номиналом 1 ввели обозначение 
\[
B(t,T) = B_t \E^{\Q}(B_T^{-1} \mid \F_t).
\]

Если безрисковая процентная ставка в модели предполагается детерминированной (\te\ $B_t$ "--- детерминированная последовательность), то $B(t,T)=B_t/B_T$ не зависит от выбора ЭММ.

В случае же стохастической процентной ставки величина $B(t,T)$, вообще говоря, может принимать разные значения для разных ЭММ. На практике тогда нужно рассматривать такие ЭММ, которые дают цены бескупонных облигаций в модели, равные рыночным ценам, или, что в целом эквивалентно, нужно включать облигации в число базовых рисковых активов с заданными зачальными ценами $S_0 = B(0,T)$.
\end{example}

\begin{example}[форвардная цена]
\label{gen:e:forward}
Напомним (см.~лекцию \ref{ch:crr}), что $T$-форвардная цена рискового актива $n$ в момент времени $t$ определяется как такая $\F_t$"=измеримая величина $F_t^n$, что платежное обязательство $X = S_T^n - F_t^n$ имеет нулевую стоимость в момент $t$, \te\
\[
0 = V_t^X = B_t\E^{\Q}\left(\frac{X}{B_T} \;\bigg|\; \F_t\right).
\]
Отсюда, пользуясь тем, что $\E^\Q(S_T^n/B_T\mid \F_t) = S_t^n/B_t$, находим форвардную цену
\[
F_t = \frac{S_t^n}{B_t\E^{\Q}(B_T^{-1} \mid \F_t)} = \frac{S_t^n}{B(t,T)}
\]
(величина $B(t,T)$ здесь опять зависит от выбора ЭММ).

Если безрисковая ставка является детерминированной, то от условного математического ожидания можно избавиться и получить, что $F_t^n = (B_T/B_t) S_t^n$.
Нетрудно также заметить, что в этом случае форвардная цена является мартингалом относительно ЭММ, а именно $F_t^n = \E^{\Q}(S_T^n\mid \F_t)$ для любой ЭММ.
\end{example}

\begin{example}[паритет цен колл-пут]
Справедлив следующий результат: если $C$ и $P$ обозначают цены опционов колл и пут на один и тот же рисковый актив $n$ с одинаковым страйком $K$ и одинаковым временем исполнения $T$, то
\[
C - P = S_t^n - KB(t,T) = B(t,T)(F_t^n - K).
\]
Действительно, выплата длинной позиции по опциону колл и короткой позиции по опциону пут равна $S_T^n - K$ и реплицируется единицей рискового актива и бескупонной облигацией с номиналом $K$, что доказывает первое равенство. Второе равенство, очевидно, следует из полученного выражения для форвардной цены. 
Можно также пойти путем вычисления безарбитражной цены как ожидания по ЭММ и получить ту же формулу (упражнение).
\end{example}


\subsection{\difficult\ Множество безарбитражных цен}

В качестве дополнения приведем более детальное описание множества безарбитражных цен произвольного платежного обязательства.
Для простоты рассмотрим только платежные обязательства $X\ge 0$ и их цены в момент $t=0$.
Будем считать, что $\sigma$-алгебра $\F_0$ тривиальна, и, следовательно, цены в момент~$0$ являются детерминированными величинами.
Обозначим далее за $\Pi$ множество самофинансируемых стратегий.

\begin{definition}
\emph{Верхней ценой} платежного обязательства $X$ называется величина
\[
V^*(X) = \inf\{v\in\R : \exists\, \pi\in\Pi\ V_0^\pi=v,\ V_T^\pi \ge X\ \as\},
\]
\emph{Нижней ценой} платежного обязательства $X$ называется величина
\[
V_*(X) = \sup\{v\in \R : \exists\, \pi\in\Pi\ V_0^\pi=v,\ V_T^\pi \le X\ \as\}.
\]
\end{definition}

Смысл верхней цены состоит в том, что любая цена $v$ выше нее будет заведомо представлять арбитражную возможность для продавца платежного обязательства, \tk\ он может найти самофинансируемую стратегию $\pi$ со  стоимостью начального портфеля немного меньше $v$, которая в последний момент времени покрывает выплату $X$.
Разница $v-V_0^\pi$ дает безрисковый доход.
Аналогично, любая цена ниже нижней цены будет представлять арбитражную возможность для покупателя платежного обязательства.

Нетрудно показать, что всегда $V_*(X) \le V^*(X)$.

\begin{proposition}
\label{gen:p:price-interval}
\begin{alphenum}
\item Если платежное обязательство $X$ реплицируемо, то $V^*(X) = V_*(X)$ и эти величины равны безарбитражной цене $V_0^X$.
\item Если платежное обязательство $X$ не реплицируемо, то множество его безарбитражных цен $($\te\ множество $\{\E^{\Q}(X/B_T) \mid \text{$\Q$ "--- ЭММ}\}$$)$ является интервалом $(V_*(X),V^*(X))$.
\end{alphenum}
\noindent
В обоих случаях
\[
V_*(X) = \inf_{\Q\in\mathcal{Q}} \E^{\Q}\left(\frac{X}{B_T}\right),\qquad
V^*(X) = \sup_{\Q\in\mathcal{Q}} \E^{\Q}\left(\frac{X}{B_T}\right),
\]
где $\mathcal{Q}$ "--- множество всех ЭММ.
\end{proposition}

\begin{remark}
Для реплицируемого платежного обязательства $V_*(X)$ и $V^*(X)$ являются безарбитражными ценами, а дле нереплицируемого "--- нет.
\end{remark}

Справедливость предложения \ref{gen:p:price-interval} следует из результатов раздела 3 главы 5 книги \cite{FollmerSchied11} и \S\,1c главы VI книги \cite{Shiryaev98}; здесь мы не приводим доказательства.
Следующий рисунок иллюстрирует утверждение предложения \ref{gen:p:price-interval}.

\begin{figure}[h]
\centering
\begin{tikzpicture}
\draw (1,0)--(12,0);

\draw[very thick] (3.875,0.3)--(4,0.3)--(4,-0.1)--(3.875,-0.1);
\node[anchor=north] at (4,-0.1) {\small $V_*(X)$};
\node[anchor=south east] at (4,0.2) {\parbox{4cm}{\small \raggedleft Арбитраж для покупателя}};
\fill[pattern=north west lines] (1,0.2) rectangle  (4,0);

\draw[very thick] (9.125,0.3)--(9,0.3)--(9,-0.1)--(9.125,-0.1);
\node[anchor=north] at (9,-0.1) {\small $V^*(X)$};
\node[anchor=south west] at (9,0.2) {\parbox{4cm}{\small \raggedright Арбитраж для продавца}};
\fill[pattern=north west lines] (9,0.2) rectangle  (12,0);

\node[anchor=south] at (6.5,0.2) {\small Безарбитражные цены};
\end{tikzpicture}
\caption{Множество (интервал) безарбитражных цен.}
\label{gen:fig:na-interval}
\end{figure}


\section{Полнота рынка}

\begin{definition}
Безарбитражная модель рынка называется \emph{полной}, если любое ограниченное платежное обязательство реплицируемо.
\end{definition}

\begin{example}
Модель КРР является полной, а триномиальная модель неполной.
\end{example}

\begin{theorem}[вторая фундаментальная теорема финансовой математики]
Модель рынка является полной тогда и только тогда, когда в ней существует единственная ЭММ.
\end{theorem}

\begin{proof}
\textit{Необходимость.} Пусть рынок полон.
Существование ЭММ следует из безарбитражности.
Покажем, что ЭММ единственна.
Для начала будем предполагать, что случайная величина $B_T$ ограничена. 

Для произвольного события $A\in\F_T$ рассмотрим платежное обязательство $X=B_T \I_A$, где $\I_A$ "--- индикатор события.
Так как оно реплицируемо, то у него есть однозначно определенная цена $V_0^X$.
Тогда для любой ЭММ имеем 
\[
V_0^X = \E^{\Q}\frac{X}{B_T} = \E^Q I_A = \Q(A).
\]
Следовательно, $\Q(A)$ однозначно определена для любого $A\in \F$.
Значит, ЭММ единственна.

В случае неограниченной величины $B_T$ нужно рассмотреть события вида $A_n = A\cap \{\omega: B_t \le n\}$ и применить к ним рассуждение выше.
Тогда вероятность $\Q(A_n)$ однозначно определена, и следовательно, $\Q(A) = \lim_{n\to\infty} \Q(A_n)$ тоже однозначно определена.

\textit{Достаточность} в нашем курсе не доказывается, ее доказательство довольно трудное.
Его можно найти, например, в главе V.4 книги \cite{Shiryaev98}.
\end{proof}

В заключение приведем расширенный вариант второй фундаментальной теоремы.
Он показывает, что в дискретном времени любая полная модель является, в сущности, <<$(N+1)$-номиальной>>.
В частности, ее можно задать на конечном вероятностном пространстве. 
Следовательно, если изменения цен на один период времени описываются непрерывной случайной величиной (например, с нормальным или лог-нормальным распределением), то модель заведомо не полна.

\begin{theorem}[расширенный вариант второй фундаментальной теоремы]
Следующие утверждения эквивалентны.
\begin{alphenum}
\item Модель рынка полна.
\item Любое платежное обязательство реплицируемо (ограниченность уже не требуется).
\item Существует единственная ЭММ.
\item $\F_t = \sigma(S_1,\dots,S_t)$ (с точностью до множеств нулевой вероятности) и существует предсказуемая последовательность $a_t$, где $a_t=(a_t^1,\dots,a^{N+1})$, такая, что для всех $t=0,\dots,T-1$
\[
\P(\Delta S_{t+1} \in\{a_{t+1}^1,\dots,a_{t+1}^{N+1}\} \mid \F_t) = 1\ \as
\]
\item Любой мартингал $M=(M_t)_{t=0}^T$, заданный на $(\Omega,\F,\FF,\Q)$, представим в виде мартингального преобразования
\[
M_t = M_0 + \sum_{u=1}^t H_u\cdot \Delta \tilde S_u
\]
с некоторой $N$-мерной предсказуемой последовательностью $H_t$.
\end{alphenum}
Более того, если рынок полон, то $\sigma$-алгебра $\F_T$ является чисто атомической с не более чем $(N+1)^T$ атомами%
\footnote{Это означает, что найдутся непересекающиеся множества $A_i\in\F$ в количестве не более чем $(N+1)^T$ штук такие, что любое множество из $\F_T$ представимо в виде их объединения.
Эквивалентно, $\F_T$ порождается некоторой случайной величиной, принимающей не более чем $(N+1)^T$ значений.}, и, в частности, любая $\F_T$-измеримая случайная величина принимает лишь конечное число значений.
\end{theorem}

Идея доказательства приведена в главе V.4 книги \cite{Shiryaev98}; см.~также раздел 4 главы 5 в книге \cite{FollmerSchied11}.


\section{\difficult\ Модель с дивидендами}
\label{gen:s:dividends}
Покажем, как можно обобщить модель рынка на случай, когда рисковые активы платят дивиденды.
%Мы сведем эту модель к уже рассмотренной, изменив последовательность цен рисковых активов.

Будем считать, что все цены $S_t^n$ строго положительны\footnote{Дивиденды выплачиваются акциями, а цены акций положительны.}, а также предположим, что рисковый актив $n$ платит в момент времени $t$ дивиденд в размере $D_t^n\ge 0$ единиц валюты на единицу актива.
Такое предположение выражается в том, что условие самофинансируемости торговой стратегии нужно  заменить на следующее:
\begin{equation}
\label{gen:sf-div}
(H_{t+1} - H_t)\cdot S_t  = -(G_{t+1} - G_t)B_t + H_t\cdot D_t,
\end{equation}
где $D_t = (D_t^1,\dots,D_t^N)$ "--- вектор дивидендов.
Торговой стратегией здесь по-прежнему называется предсказуемая последовательность $\pi_t = (G_t,H_t)$, где $G_t$ "--- количество безрискового актива в портфеле, а $H_t^n$ "--- количество $n$-го рискового актива. Таким образом, условие самофинансируемости означает, что затраты на покупку рисковых активов должны компенсироваться продажей безрискового актива (и наоборот), а также дивидендными выплатами.

Стоимость портфеля $V_t^\pi$ определим соотношениями
\begin{align}
\label{gen:V0-div}
&V_0^\pi = G_1B_0 + \scal{H_1}{S_0},\\
\label{gen:Vt-div}
&V_t^\pi = G_tB_t + \scal{H_t}{S_t} + \scal{H_t}{D_t},\ t=1,\dots,T.
\end{align}
Различие с моделью без дивидендов здесь в том, что стоимость портфеля в момент $t$ теперь состоит из стоимости активов, купленных в предыдущий момент времени, плюс выплаченных дивидендов. Нетрудно видеть, что при выполнении условия самофинансируемости равенство \eqref{gen:Vt-div} эквивалентно равенству
\begin{equation}
\label{gen:Vt-div-sf}
V_t^\pi = G_{t+1}B_t + \scal{H_{t+1}}{S_t}.
\end{equation}


Аналогично первому утверждению предложения~\ref{gen:p:sf} можно показать, что стратегия $\pi$ является самофинансируемой тогда и только тогда, когда
\[
V_t^\pi = V_0^\pi + \sum_{u=1}^t (G_u\Delta B_u + \scal{H_u}{(\Delta S_u + D_u)}).
\]

Чтобы сформулировать аналог второго утверждения предложения \ref{gen:p:sf}, обозначим $q_t^n = D_t^n/S_t^n$ и введем последовательности
\begin{equation}
\label{gen:div-yield}  
Y_t^0 = 1, \qquad Y_t^n = \prod_{s=1}^t \left(1+q_t^n\right),\ t=1,\dots,T.
\end{equation}
Величину $q_t^n$ нужно интерпретировать как \emph{дивидендную доходность} актива $n$ в момент $t$. 
Тогда $Y_t^n$ "--- это стоимость портфеля стратегии, которая вкладывает капитал в рисковый актив $n$, реинвестируя дивиденды, и имеет единичную начальную стоимость портфеля%
\footnote{Можно заметить некоторое сходство между последовательностью $Y_t^n$ и ценой безрискового актива $B_t = \prod_{s=1}^t (1+r_s)$. Действительно, стоимость безрискового актива получается в сущности реинвестированием капитала в денежный рынок с доходностью $r_t$ на каждом шаге.}.
А именно, эта стратегия имеет вид $\pi_t=(G_t,H_t)$, где $G_t=0$, $H_t^i = 0$ для $i\neq n$ и $H_t^n = Y_{t-1}^n/S_{0}^n$.
Непосредственно проверяется, что $\pi$ удовлетворяет условию самофинансируемости \eqref{gen:sf-div}.

Определим также последовательности
\[
\tilde S_t^n = \frac{Y_t^nS_t^n}{B_t},
\]
которые будем называть \emph{дисконтированными ценами рисковых активов с учетом дивидендов}.


\begin{proposition}
Стратегия $\pi$ является самофинансируемой тогда и только тогда, когда
\begin{equation}
\label{gen:sf-div-tilde}
\tilde V_t^\pi = V_0^\pi + \sum_{u=1}^t \sum_{n=1}^N \frac{H_u^n}{Y_{u-1}^n} \Delta \tilde S_u^n.
\end{equation}
\end{proposition}
\begin{proof}
Условие \eqref{gen:sf-div-tilde} равносильно тому, что
\[
\tilde V_t^\pi -\tilde V_{t-1}^\pi = \sum_{n=1}^N \frac{H_t^n}{Y_{t-1}^n} \Delta \tilde S_t^n. 
\]
Если стратегия $\pi$ самофинансируемая, то, выражая $V_t^\pi$ по формуле \eqref{gen:Vt-div}, а $V_{t-1}^\pi$ по формуле \eqref{gen:Vt-div-sf}, находим
\begin{multline*}
\tilde V_t^\pi -\tilde V_{t-1}^\pi 
= \sum_{n=1}^N H_t^n \left(\frac{(1+q_t^n)S_t^n}{B_t} - \frac{S_{t-1}^n}{B_{t-1}}\right)
\\= \sum_{n=1}^N \frac{H_t^n}{Y_{t-1}^n} \left(\frac{Y_t^nS_t^n}{B_t} - \frac{Y_{t-1}^nS_{t-1}^n}{B_{t-1}}\right)
=\sum_{n=1}^N \frac{H_t^n}{Y_{t-1}^n} \Delta \tilde S_t^n,
\end{multline*}
что доказывает представление \eqref{gen:sf-div-tilde}. Доказательство в обратную сторону проводится аналогично (упражнение). 
\end{proof}

\begin{definition}
Вероятностная меры $\Q\sim\P$ называется \emph{эквивалентной мартингальной мерой} в модели с дивидендами, если для каждого $n$ последовательность $\tilde S_t^n$ является $\Q$-мартингалом.
\end{definition}

Представление \eqref{gen:sf-div-tilde} позволяет легко доказать аналог предложения \ref{gen:p:discount-value} (доказательство такое же, как в модели без дивидендов): дисконтированная стоимость портфеля любой самофинансируемой стратегии является мартингальным преобразованием относительно $\Q$; если к тому же $\tilde V_T^\pi \ge c$ для некоторой константы $c$, то $\tilde V_t^\pi$  является мартингалом.

Аналогично, используя представление \eqref{gen:sf-div-tilde}, доказывается первая фундаментальная теорема для модели с дивидендами: отсутствие арбитража равносильно существованию ЭММ.

Безарбитражную цену ограниченного снизу платежного обязательства $X$ в модели с дивидендами можно найти по формуле, аналогичной \eqref{gen:na-price}:
\[
V_t^X = \E^{\Q}\left(\frac{X}{B_T} \;\bigg|\;\F_t\right).
\]

Теперь, однако не удастся получить простое представление для форвардной цены как в модели без дивидендов. А именно, для $T$-форвардной цены актива $n$ имеем
\begin{equation}
\label{gen:fut-div}
F_t^n = \frac{\E^\Q(S_T/B_T \mid \F_t)}{\E^\Q(B_T^{-1}\mid \F_t)} = B_t\frac{\E^\Q(S_T/B_T \mid \F_t)}{B(t,T)},
\end{equation}
где $B(t,T) = B_t\E^{\Q}(B_T^{-1} \mid \F_t)$ "--- цена бескупонной облигации с единичным номиналом.
Правую часть формулы \eqref{gen:fut-div} упростить далее не получается, так как последовательность $S_t/B_t$, вообще говоря, не является мартингалом.

При этом, паритет цен колл-пут можно по-прежнему записать в виде
\[
C - P = B(t,T)(F_t - K).
\]
Доказательство этого равенства остается в качестве упражнения.


\summary

\begin{itemize}
\item Общая модель рынка в дискретном времени состоит из одного безрискового актива, цена которого является предсказуемой случайной последовательностью, и $N$ рисковых активов, цены которых являются согласованными случайными последовательностями.
\item Эквивалентная мартингальная мера (ЭММ) "--- это такая вероятностная мера, что она (a) эквивалентна исходной мере, (b) дисконтированные цены рисковых активов являются мартингалами относительно нее.

\item Первая фундаментальная теорема финансовой математики утверждает, что отсутствие арбитража равносильно существованию ЭММ.

\item В безарбитражной модели безарбитражная цена европейского платежного обязательства $X$ может быть найдена по формуле 
\[
V_t =  B_t \E^{\Q} \left( \frac{X}{B_T} \;\bigg|\; \F_t \right),
\]
где $\Q$ "--- ЭММ (для разных ЭММ могут получаться разные цены).

\item Если платежное обязательство реплицируемо, то его цены по всем ЭММ одинаковы.

\item Вторая фундаментальная теорема утверждает, что рынок полон (\te\ отсутствует арбитраж и любое платежное обязательство реплицируемо) тогда и только тогда, когда ЭММ существует и единственна. Любую модель полного рынка всегда можно задать на конечном вероятностном пространстве.

\item \difficult\ В модели с дивидендами ЭММ характеризуется тем условием, что мартингалами должны являться дисконтированные цены рисковых активов с учетом дивидендов.
\end{itemize}
