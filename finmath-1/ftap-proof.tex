%!TEX root=finmath1.tex
\chapter{Доказательство первой фундаментальной теоремы}
\label{ch:ftap-proof}

Это дополнение содержит доказательство трудной импликации в первой фундаментальной теореме финансовой математики в дискретном времени (о том, что из безарбитражности следует существование эквивалентной мартингальной меры).


\section{План доказательства}

Для простоты мы будем рассматривать случай одного рискового актива и считать, что процентная ставка равна нулю ($B_t=1$ для всех $t$).
Доказательство в общем случае можно найти в оригинальной статье Даланга, Мортона и Виллинджера \cite{DalangMorton+90}; более короткое доказательство приведено в статье Кабанова и Стрикера \cite{KabanovStricker01}.

При наших предположениях докрываемую импликацию можно переформулировать в следующем абстрактном виде.

\begin{theorem}
\label{ftap-p:theorem}
Пусть на вероятностном пространстве $(\Omega,\F,\P)$ c фильтрацией $\FF=(\F_t)_{t=0}^T$ задана согласованная последовательность $S=(S_t)_{t=0}^T$ такая, что для любой предсказуемой последовательности $H=(H_t)_{t=1}^T$, удовлетворяющей неравенству $V_T^H := \sum_{t=1}^H H_t \Delta S_t \ge 0$ \as, выполнено $V_T^H = 0$ \as\ (гипотеза отсутствия арбитража NA). 

Тогда на $(\Omega,\F)$ существует вероятностная мера $\Q\sim\P$, относительно которой $S$ является мартингалом.
\end{theorem}

Доказательство будет состоять из трех шагов.
На первом шаге мы докажем замкнутость относительно сходимости \as\ множества $A$, состоящего из платежных обязательств, которые могут быть \emph{суперреплицированы}%
\footnote{Суперрепликация означает, что найдется торговая стратегия, стоимость портфеля которой в последний момент времени не меньше, чем выплата по заданному обязательству.} самофинансируемыми стратегиями, имеющими нулевую стоимость начального портфеля.  
Затем с помощью теоремы Крепа"--~Яна мы найдем вероятностную меру $\Q\sim\P$, разделяющую множества $A\cap L^1$ и $L_+^1$.
На третьем шаге будет показано, что $\Q$ является ЭММ.


\section{Вспомогательные результаты}

В этом разделе собраны вспомогательные результаты, которые нам потребуются.
Непосредственно для доказательства теоремы \ref{ftap-p:theorem} из них нужны только леммы \ref{ftap-p:l:kreps-yan}, \ref{ftap-p:l:subsequence}, \ref{ftap-p:l:strong-na}, а остальные используются для доказательства этих трех лемм.

\begin{lemma}[теорема Рисса; см.~\cite{Bogachev03}, теорема 2.2.5]
Пусть $X_n\to X$ в $L^1$ при $n\to\infty$.
Тогда найдется подпоследовательность $X_{n_k} \to X$ \as
\end{lemma}
\begin{remark}
Напомним, что $L^1$ "--- это пространство классов эквивалентности случайных величин с конечным математическим ожиданием и нормой $\|X\|_{L^1} = \E|X|$, относительно которой оно является банаховым пространством.
Под сходимостью $X_n\to X$ в $L^1$ понимают сходимость по норме, \te\ $\|X_n-X\|_{L^1} = \E|X_n-X| \to 0$.
В теории вероятностей ее также называют \emph{сходимостью в среднем}.
\end{remark}

\begin{corollary}
Пусть некоторое множество случайных величин $A$ замкнуто относительно сходимости \as\ (\te\ из сходимости $A\ni X_n \to X$ \as\ следует, что $X\in A$).
Тогда множество $A \cap L^1$ замкнуто в $L^1$.
\end{corollary}

\begin{proof}
Если $A \cap L^1 \ni X_n \to X$ в $L^1$, то по теореме Рисса найдется подпоследовательность $X_{n_k} \to X$ \as\
Тогда $X\in A$, и, значит, $X\in A\cap L^1$.
\end{proof}
  
\begin{definition}
Пусть $\{X_\lambda\}_{\lambda\in\Lambda}$ является семейством случайных величин, индексированных произвольным множеством $\Lambda$. \emph{Существенным супремумом} этого семейства называется случайная величина $S$ со значениями в $\R\cup\{+\infty\}$ такая, что $S \ge X_\lambda$ \as\ для любого $\lambda\in\Lambda$, и при этом для любой случайной величины $S'$ с таким же свойством выполнено неравенство $S'\ge S$ \as

Обозначение: $\esssup_{\lambda\in\Lambda} X_\lambda$.
\end{definition}

\begin{lemma}[см.~\cite{Shiryaev04}, гл.~VII, \S\,13.3]
У любого семейства случайных величин существует существенный супремум, единственный с точностью до равенства \as\ 
Более того, всегда можно выбрать такое счетное подсемейство $(X_{\lambda_n})_{n=1}^\infty$, что величина $S(\omega) = \sup_{n\ge1} X_{\lambda_n}(\omega)$ является существенным супремумом.
\end{lemma}

\begin{remark}
Нельзя просто определить $S(\omega) = \sup_{\lambda\in\Lambda} X_\lambda(\omega)$ для каждого $\omega$, так как функция $S(\omega)$ может оказаться неизмеримой, когда множество $\Lambda$ несчетно.  
\end{remark}

\begin{lemma}[теорема Хана"--~Банаха об отделимости; см.~\cite{KolmogorovFomin76}, гл.~IV, \S\,1.3, следствие 2]
Пусть $A$ "--- выпуклое множество в нормированном пространстве $L$, а некоторый ненулевой элемент $x\in L$ не лежит в $A$.
Тогда найдется непрерывный линейный функционал $p$ на $L$ такой, что $p(x) >0$ и $p(a)\le 0$ для любого $a\in A$.
\end{lemma}

\begin{lemma}[теорема Крепса"--~Яна]
\label{ftap-p:l:kreps-yan}
Пусть $A\subseteq L^1$ "--- выпуклое множество, причем $A\supseteq -L^1_+$ и $A\cap L^1_+ = \{0\}$, где $L_+^1$ "--- множество неотрицательных случайных величин с конечным математическим ожиданием на вероятностном пространстве $(\Omega,\F,\P)$.
Тогда найдется вероятностная мера $\Q\sim\P$ такая, что $\E^{\Q} X \le 0 $ для всех $X\in A$.
\end{lemma}

\begin{proof}
Пусть $L^\infty$ обозначает нормированное пространство классов эквивалентности ограниченных случайных величин с нормой $\|X\|_{L^\infty} = \inf\{c\ge 0 : |X| \le c\ \as\}$. Известно, что $L^\infty$ является банаховым пространством, причем оно двойственно к пространству $L^1$, \te\ любой непрерывный линейный функционал $p$ на $L^1$ можно отождествить с элементом $Z\in L^\infty$ по правилу $p(X) = \E(ZX)$.

Тогда по теореме Хана"--~Банаха для любого $X\in L^1_+\setminus\{0\}$ найдется случайная величина $Z_X\in L^\infty$ такая, что $\E Z_X X > 0$ и $\E Z_X X' \le 0$ для любого $X'\in A$.
Так как $A\supseteq - L^1_+$, то $Z_X\ge 0$.
Более того, можно считать, что $\|Z_X\|_{L^\infty} = 1$ (иначе поделим $Z_X$ на $\|Z_X\|_{L^\infty}$).

Положим $S = \esssup_{X\in L^1_+\setminus\{0\}} Z_X$.
Заметим, что $S \le 1$, так как $Z_X \le 1$.
Кроме того, $S>0$, так как иначе нельзя было бы отделить $X=\I(S=0)$ от $A$. 

Пусть $X_k\in L_+^1\setminus\{0\}$ является таким счетным семейством, что $S = \sup_k Z_{X_k}$.
Определим $Z = c \sum_{k=1}^\infty 2^{-k} Z_{X_k}$, где константа $c$ выбирается так, чтобы сделать $\E Z = 1$.
Заметим, что ряд для $Z$ сходится в $L^\infty$, причем $Z>0$ \as\ и $\E ZX' \le 0$ для всех $X'\in A$.
Тогда искомую меру можно задать в виде\footnote{Запись $d\Q = Z d\P$ означает, что $Z$ "--- это плотность Радона"--~Никодима $\Q$ относительно $\P$. Подробнее см.~раздел~\ref{itopr:girsanov} в лекции~\ref{ch:itopr}.} $d\Q = Z d\P$.
\end{proof}

\begin{lemma}[о сходящейся подпоследовательности]
\label{ftap-p:l:subsequence}
Предположим, что последовательность случайных величин $X_n$, $n\ge 0$, такова, что $\liminf_{n\to\infty} |X_n| < \infty$ \as\ 
Тогда найдется такая строго возрастающая случайная последовательность $\sigma_n$ со значениями в $\mathbb{Z}_+$, что последовательность $X_{\sigma_n}$ имеет предел \as\ при $n\to\infty$.
\end{lemma}

\begin{proof}
Положим $X = \liminf_{n\to\infty} |X_n|$.
Определим $\tau_0=0$, а для $n\ge 1$ положим
\[
\tau_n(\omega) 
  = \min\{k>\tau_{n-1}(\omega) : |X(\omega) - |X_k(\omega)|| < 1/n\}.
\]
Обозначим $\tilde X_n := X_{\tau_n}$.
Заметим, что $\sup_{n\ge 0} |\tilde X_n| < \infty$ \as, и, следовательно, $\tilde X := \liminf_{n\to\infty} \tilde X_n < \infty$.
Определим $\eta_0=0$ и
\[
\eta_n(\omega) = \inf\{k > \eta_{n-1}(\omega) 
  : |\tilde X(\omega) - \tilde X_k(\omega)| < 1/n\}.
\]
Тогда имеет место сходимость $\tilde X_{\eta_n} \to \tilde X$ \as, а, кроме того, $\tilde X_{\eta_n} = X_{\tau_{\eta_n}}$.
Следовательно, в качестве искомой последовательности можно взять $\sigma_n = \tau_{\eta_n}$.
\end{proof}

\begin{lemma}[об усиленном свойстве безарбитражности]
\label{ftap-p:l:strong-na}
Предположим, что на вероятностном пространстве $(\Omega,\F,\P)$ c фильтрацией $\FF=(\F_t)_{t=0}^T$ задана согласованная последовательность $S=(S_t)_{t=0}^T$, удовлетворяющая гипотезе NA в том виде, как она сформулирована в теореме \ref{ftap-p:theorem}.

Пусть $V_T^H := \sum_{t=1}^T H_t\Delta S_t \ge 0$ для некоторой предсказуемой последовательности $H$. 
Тогда для этой последовательности $V_t^H := \sum_{u=1}^t H_u\Delta S_u = 0$ при всех $t=1,\dots,T$.
\end{lemma}

\begin{proof}
Сначала докажем, что $V_t^H\ge 0$ для всех $t$.
Предположим противное и возьмем первое $t$ для которого $\P(V_t^H<0)>0$. 
Положим $\tilde H_u = 0$ при $u\le t$ и $\tilde H_u = H_u\I(V^H_t<0)$ при $u> t$.
Тогда имеем
\[
V_T^{\tilde H} = \sum_{u=t+1}^T \tilde H_u \Delta S_u 
= \I(V^H_t<0) \sum_{u=t+1}^T H_u\Delta S_u = -\I(V^H_t<0) V_t^H.
\]
Следовательно, $V_T^{\tilde H} \ge 0$ и $\P(V_T^{\tilde H}>0)>0$, что противоречит гипотезе NA.

Теперь предположим, что $\P(V_t^H>0)>0$ для некоторого $t$.
Положим $\tilde H_u = H_t$ при $u\le t$ и $\tilde H_u = 0$ при $u> t$.
Тогда $V_T^{\tilde H} = \sum_{u=1}^t H_u\Delta S_u = V_t^H$.
Снова получаем $V_T^{\tilde H} \ge 0$ и $\P(V_T^{\tilde H}>0)>0$, что противоречит гипотезе NA.
\end{proof}


\section{Доказательство теоремы}

\textit{Шаг 1.} Определим множество платежных обязательств, суперреплицируемых самофинансируемыми стратегиями с нулевой стоимостью начального портфеля:
\begin{multline*}
A = \biggl\{X : X \le \sum_{t=1}^T H_t\Delta S_t,
  \ \text{где $H$ предсказуема}\biggr\} \\
= \biggl\{\sum_{t=1}^T H_t\Delta S_t - \gamma : 
  \text{$H$ предсказуема, $\gamma\ge 0$ "--- $\F_T$-измерима}\biggr\}.
\end{multline*}
Покажем, что $A$ замкнуто относительно сходимости \as\ 

Пусть $X^n = \sum_{t=1}^T H_t^n \Delta S_t - \gamma^n \to X$ \as\ 
Нужно показать, что $X\in A$.
Докажем это индукцией по $T$.
Для $T=0$ утверждение очевидно.
Предположим, что оно доказано для $T-1$ и $\sum_{t=1}^T H_t^n\Delta S_t - \gamma^n \to X$ \as\ 
Положим
\[
\Omega_1 = \left\{\omega : \liminf_{n\to\infty} |H_1^n(\omega)| < \infty\right\}.
\]
По лемме о сходящейся подпоследовательности, найдутся $\F_0$-измеримые величины $\sigma_n$ такие, что предел $H_1:=\lim\limits_{n\to\infty }H_1^{\sigma_n} \I_{\Omega_1}$ существует и конечен.

Пусть $\Omega_2 = \Omega\setminus\Omega_1$.
Покажем, что $\Delta S_1 = 0$ \as\ на $\Omega_2$.
Положим $G_t^n = H_t^n/|H_1^n|$ и $f^n = \gamma^n/|H_1^n|$.
Тогда
\[
\lim_{n\to\infty}\biggl(\sum_{t=1}^T G_t^n\Delta S_t - f^n\biggr) 
= \lim_{n\to\infty} \frac{X}{|H_1^n|} =  0\ \text{на $\Omega_2$}.
\]
По лемме о сходящейся подпоследовательности найдутся величины $\tau_n$ такие, что $G_1'^n := G_1^{\tau_n}\I_{\Omega_2} \to G_1'$.
Заметим, что $G_1'$ принимает значения $\pm1$ на $\Omega_2$. 
Пусть $f'^n = f^{\tau_n}$.
Тогда
\[
\sum_{t=2}^T G_t'^n \Delta S_t - f'^n \to - G'_1 \Delta S_1\ \text{на $\Omega_2$}.
\]
По предположению индукции существуют $\tilde G_t'$, $t=2,\dots,T$, и $\tilde f'\ge 0$ такие, что 
\[
\sum_{t=2}^T \tilde G_t' \Delta S_t - \tilde f' = -  G'_1 \Delta S_1.
\]
Пусть $\tilde G_1' = G_1'$.
Тогда $\sum_{t=1}^T \tilde G_t' \Delta S_t = \tilde f'$.
В силу усиленной гипотезы безарбитражности, имеем $\tilde G_1'\Delta S_1 = 0$.
Так как $\tilde G_1'\in\{-1,1\}$ на $\Omega_2$, то $\Delta S_1 = 0$ \as\ на $\Omega_2$. 

Найдем теперь такие $H$ и $\gamma$, что $X=\sum_{t=1}^T H_t\Delta S_t - \gamma$.
Определим $\tilde H_t^n = H^{\sigma_n}_t$, $\tilde \gamma^n = \gamma^{\sigma_n}$.
Так как $\tilde H_1^n\Delta S_1 \to H_1\Delta S_1$ и $\sum_{t=1}^T \tilde H_t^n \Delta S_1 - \tilde \gamma^n \to X$,
то имеем
\[
\sum_{t=2}^T \tilde H_t^n \Delta S_1 - \tilde \gamma^n \to X - H_1 \Delta S_1.
\]
По предположению индукции существуют $H_t$, $t=2,\dots,T$, и $\gamma\ge 0$ такие, что
\[
\sum_{t=2}^T H_t \Delta S_1 - \gamma = X - H_1 \Delta S_1.
\]
Отсюда уже легко следует доказываемое утверждение.

\medskip
\emph{Шаг 2.}
Далее без ограничения общности будем считать, что $\E|S_t| < \infty$ для всех $t=1,\dots,T$.
Если это не так, что можно перейти к мере $\P' \sim \P$ с плотностью
\[
\frac{d\P'}{d\P} = c e^{-\sum\limits_{t=1}^T |S_t|},
\]
где $c$ "--- нормализующая константа.
Тогда $\E^{\P'} |S_t| < \infty$ и достаточно заметить, что замена меры на эквивалентную сохраняет свойство замкнутости относительно сходимости \as, а также справедливость гипотезы NA.

Обозначим $A^1 = A \cap L^1$.
Имеем $A^1\supset -L^1_+$ по определению $A$.
Кроме того, $A^1\cap L^1_+ = \{0\}$ в силу гипотезы NA.
Тогда по теореме Крепса"--~Яна найдется мера $\Q\sim\P$, для которой
\[
\E^{\Q} X \le 0\ \text{для всех}\ X\in A^1.
\]

\medskip
\textit{Шаг 3.}
Проверим, что мера $\Q$ является ЭММ.
Нужно показать, что для всех $t=0,\dots,T-1$ выполнено равенство $\E^{\Q} (S_{t+1}\mid \F_t) = S_t$.
Предположим, от противного, что $\Q(B) > 0$ для некоторого множества $B = \{\omega : \E (S_{u+1}\mid \F_u)(\omega) > S_u(\omega)\}$.

Рассмотрим последовательность $H_t$, в которой $H_{u+1} = \I_B$ и $H_t = 0$ для всех $t\neq u+1$.
Пусть $X = \sum_{t=1}^T H_t\Delta S_t$.
Тогда $X\in A^1$, так как $H$ ограничена и $\E^{\Q} |S_t| < \infty$.
Следовательно, $\E^{\Q} X = \E^Q \I_B (S_{t+1}-S_t) > 0$ "--- противоречие с конструкцией $\Q$.

Таким образом, $\Q(B) = 0$, и поэтому $\E (S_{t+1} \mid \F_t) \ge S_t$ для всех $t$.
Неравенство в другую сторону доказывается аналогично.
Следовательно, $\Q$ является ЭММ.
