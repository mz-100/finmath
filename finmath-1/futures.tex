%!TEX root=finmath1.tex
\chapter{Фьючерсы и другие маржируемые контракты}
\label{ch:futures-discrete}
\chaptertoc

Фьючерс "--- это контракт на покупку/продажу базового актива в будущем, торгуемый на организованной бирже.
Исполнение фьючерса гарантируется специальным механизмом ежедневного перечислением вариационной маржи.
В этой лекции мы расширим модель рынка из лекции \ref{ch:general} и покажем, как включить в нее фьючерсы и другие маржируемые контракты, а также вычислим их безарбитражные цены.


\section{Механика торговли фьючерсами}
Для лучшего понимая математичкой теории, излагаемой в следующих разделах, мы сначала опишем как технически осуществляется торговля фьючерсами.
Приводимые рассуждения можно легко обобщить на другие контракты с механизмом перечисления вариационной маржи, о чем будет сказано в замечании \ref{fut:r:mtm-contract} в конце этого раздела.

Фьючерс можно представлять себе как контракт на поставку определенного актива в будущем. 
Фьючерс характеризуется базовым активом (акция, индекс, товар и \tp), датой экспирации, типом расчетов (расчетный или поставочный) и объемом лота (количество единиц базового актива в одном контракте).
По одному базовому активу одновременно в обращении находятся несколько фьючерсов с разными датами экспирации.
Наиболее далекие экспирации могут отстоять от текущего момента времени на несколько лет.
По расчетным фьючерсам все расчеты между покупателями и продавцами происходят в денежной форме; по поставочным фьючерсам, помимо промежуточных денежных расчетов, продавцы обязаны выполнить передачу базового актива покупателю, а покупатели обязаны заплатить соответствующую стоимость.

Фьючерсы торгуются на организованных биржах, и в каждый момент времени до даты экспирации имеется биржевая цена фьючерса, которая получается в результате взаимодействия покупателей и продавцов%
\footnote{Торги на современных биржах проходят в электронном режиме и по каждому инструменту доступна не какая-то одна цена, а список заявок на покупку и продажу (биржевой <<стакан>>, order book). 
Когда мы говорим о цене инструмента, то имеем ввиду, например, среднюю цену между лучшими ценами заявок на покупку и продажу.
Для дальнейших рассуждений такие детали несущественны; можно предполагать, что в каждый момент времени  цена однозначно определена.}.
Однако в отличие, скажем, от акций, когда покупатель должен заплатить продавцу текущую цену акции при ее покупке, или в отличие от форвардных контрактов, когда расчеты осуществляются только в последний момент времени, расчеты по фьючерсам проходят по-другому.

Сделка по фьючерсу состоит в том, что одна сторона (покупатель) открывает <<длинную>> позицию по текущей цене (покупает фьючерс), а другая сторона (продавец) "--- <<короткую>> позицию (продает фьючерс).
В момент следующего \emph{клиринга} покупатель фьючерса получает на свой счет денежную выплату, называемую \emph{вариационной маржой}, которая равна разнице между \emph{расчетной ценой} фьючерса на момент клиринга и ценой, по которой он открыл позицию.
Если эта величина отрицательна, то вариационная маржа списывается со счета покупателя.
Продавец получает на свой счет вариационную маржу, равную разнице между ценой фьючерса при открытии позиции и расчетной ценой на момент клиринга (вариационная маржа покупателя со знаком минус).
Клиринг происходит ежедневно или несколько раз в день (дневной и вечерний) и продолжается несколько минут; торги в это время приостанавливаются.
Расчетная цена определяется путем усреднения лучших заявок на покупку и продажу за небольшой промежуток времени до начала клиринга.

В каждый следующий клиринг покупатель, если он не закроет свою позицию, получает вариационную маржу, равную разнице расчетных цен текущего и предыдущего клиринга.
Если покупатель закрывает позицию, то в клиринг он получает разницу между ценой фьючерса на момент закрытия позиции и расчетной ценой предыдущего клиринга.
Под закрытием длинной позиции понимается открытие короткой позиции (продажа фьючерса).
Если покупатель открывает и закрывает длинную позицию по фьючерсу в промежуток времени между двумя клирингами, то он получают разницу между ценами в моменты закрытия и открытия позиции. 
Для продавца фьючерса все аналогично.

В день экспирации правила расчета такие же, как и в промежуточные дни, за исключением того, что расчетная цена фьючерса в последний клиринг полагается равной текущей цене базового актива, умноженной на размер лота.
Это как раз и выражает то обстоятельство, что фьючерс является контрактом на поставку базового актива в момент экспирации, и привязывает цену фьючерса к цене базового актива.
Подчеркнем, что до момента экспирации цена фьючерса устанавливается исключительно рыночными механизмами (спросом и предложением), однако мы покажем, что существует конкретное теоретическое соотношение между справедливой ценой фьючерса и ценой базового актива. 

Если фьючерс поставочный, то дополнительно после экспирации у покупателя фьючерса возникает обязательство купить базовый актив в объеме, равном размеру лота, по расчетной цене последнего клиринга; у продавца же возникает обязательство продать базовый актив.

\begin{example}
Рис.~\ref{fut:fig}  иллюстрирует расчеты по фьючерсу.
На этом рисунке фьючерс экспирируется в момент $t=4$ (единица времени здесь соответствует одному окну между клирингами).
Объема лота равен 1.
В момент времени $t_1\in (1,2)$ покупатель открывает длинную позицию.
В моменты 2 и 3 он получает положительную вариационную маржу $M_2$ и $M_3$ в связи с тем, что цена фьючерса выросла.
В момент времени $t_2 \in (3,4)$ покупатель закрывает длинную позицию по цене ниже, чем предыдущая расчетная цена, что приводит к списанию вариационной маржи $M_4$ в момент 4.
Так как в момент 4 фьючерс экспирируется, то его цена в этот момент совпадает с ценой базового актива.
\end{example}

\begin{figure}[h]
\centering
\begin{tikzpicture}[xscale=2.2,yscale=13]
\draw[very thick,dotted] 
  (0.00, 1.00) node[left] {\footnotesize $S_0$} --
  (0.04, 1.01)--(0.08, 1.02)--(0.12, 1.02)--(0.16, 1.02)--
  (0.20, 1.02)--(0.24, 1.02)--(0.28, 1.02)--(0.32, 1.03)--(0.36, 1.02)--
  (0.40, 1.03)--(0.44, 1.02)--(0.48, 1.02)--(0.52, 1.04)--(0.56, 1.03)--
  (0.60, 1.03)--(0.64, 1.04)--(0.68, 1.05)--(0.72, 1.04)--(0.76, 1.03)--
  (0.80, 1.04)--(0.84, 1.05)--(0.88, 1.06)--(0.92, 1.06)--(0.96, 1.05)--
  (1.00, 1.04)--(1.04, 1.05)--(1.08, 1.05)--(1.12, 1.05)--(1.16, 1.05)--
  (1.20, 1.04)--(1.24, 1.03)--(1.28, 1.03)--(1.32, 1.02)--(1.36, 1.03)--
  (1.40, 1.03)--(1.44, 1.04)--(1.48, 1.03)--(1.52, 1.05)--(1.56, 1.05)--
  (1.60, 1.04)--(1.64, 1.05)--(1.68, 1.05)--(1.72, 1.06)--(1.76, 1.06)--
  (1.80, 1.06)--(1.84, 1.07)--(1.88, 1.09)--(1.92, 1.10)--(1.96, 1.10)--
  (2.00, 1.11)--(2.04, 1.11)--(2.08, 1.11)--(2.12, 1.12)--(2.16, 1.12)--
  (2.20, 1.14)--(2.24, 1.14)--(2.28, 1.13)--(2.32, 1.14)--(2.36, 1.14)--
  (2.40, 1.15)--(2.44, 1.15)--(2.48, 1.17)--(2.52, 1.18)--(2.56, 1.21)--
  (2.60, 1.21)--(2.64, 1.20)--(2.68, 1.17)--(2.72, 1.18)--(2.76, 1.18)--
  (2.80, 1.19)--(2.84, 1.19)--(2.88, 1.22)--(2.92, 1.23)--(2.96, 1.22)--
  (3.00, 1.22)--(3.04, 1.22)--(3.08, 1.21)--(3.12, 1.22)--(3.16, 1.23)--
  (3.20, 1.21)--(3.24, 1.22)--(3.28, 1.23)--(3.32, 1.24)--(3.36, 1.22)--
  (3.40, 1.21)--(3.44, 1.21)--(3.48, 1.22)--(3.52, 1.24)--(3.56, 1.24)--
  (3.60, 1.25)--(3.64, 1.24)--(3.68, 1.24)--(3.72, 1.25)--(3.76, 1.24)--
  (3.80, 1.23)--(3.84, 1.25)--(3.88, 1.25)--(3.92, 1.23)--(3.96, 1.22)--
  (4.00, 1.21);

\draw[thick] 
  (0.00, 1.11) node[left] {\footnotesize $F_0$} --
  (0.04, 1.11)--(0.08, 1.12)--(0.12, 1.12)--(0.16, 1.13)--
  (0.20, 1.12)--(0.24, 1.12)--(0.28, 1.12)--(0.32, 1.13)--(0.36, 1.12)--
  (0.40, 1.13)--(0.44, 1.11)--(0.48, 1.12)--(0.52, 1.13)--(0.56, 1.13)--
  (0.60, 1.12)--(0.64, 1.13)--(0.68, 1.14)--(0.72, 1.13)--(0.76, 1.12)--
  (0.80, 1.13)--(0.84, 1.14)--(0.88, 1.15)--(0.92, 1.14)--(0.96, 1.14)--
  (1.00, 1.12)--(1.04, 1.13)--(1.08, 1.13)--(1.12, 1.13)--(1.16, 1.13)--
  (1.20, 1.11)--(1.24, 1.11)--(1.28, 1.10)--(1.32, 1.09)--(1.36, 1.11)--
  (1.40, 1.10)--(1.44, 1.11)--(1.48, 1.10)--(1.52, 1.11)--(1.56, 1.11)--
  (1.60, 1.10)--(1.64, 1.12)--(1.68, 1.12)--(1.72, 1.12)--(1.76, 1.13)--
  (1.80, 1.12)--(1.84, 1.13)--(1.88, 1.15)--(1.92, 1.16)--(1.96, 1.17)--
  (2.00, 1.17)--(2.04, 1.17)--(2.08, 1.18)--(2.12, 1.18)--(2.16, 1.18)--
  (2.20, 1.19)--(2.24, 1.19)--(2.28, 1.18)--(2.32, 1.19)--(2.36, 1.19)--
  (2.40, 1.19)--(2.44, 1.20)--(2.48, 1.21)--(2.52, 1.23)--(2.56, 1.25)--
  (2.60, 1.25)--(2.64, 1.24)--(2.68, 1.21)--(2.72, 1.22)--(2.76, 1.22)--
  (2.80, 1.23)--(2.84, 1.23)--(2.88, 1.26)--(2.92, 1.26)--(2.96, 1.25)--
  (3.00, 1.26)--(3.04, 1.25)--(3.08, 1.24)--(3.12, 1.25)--(3.16, 1.25)--
  (3.20, 1.24)--(3.24, 1.25)--(3.28, 1.25)--(3.32, 1.26)--(3.36, 1.24)--
  (3.40, 1.23)--(3.44, 1.23)--(3.48, 1.24)--(3.52, 1.25)--(3.56, 1.25)--
  (3.60, 1.26)--(3.64, 1.25)--(3.68, 1.25)--(3.72, 1.26)--(3.76, 1.25)--
  (3.80, 1.24)--(3.84, 1.26)--(3.88, 1.26)--(3.92, 1.24)--(3.96, 1.225)--
  (4.00, 1.21);

\draw[->] (0,0.95)--(4.5,0.95) node[below] {\footnotesize $t$};
\draw[->] (0,0.95)--(0,1.3) node[left] {\footnotesize Цена};

\foreach \i in {0,1,2,3,4} {
  \draw (\i,0.955)--(\i,0.945) node[below] {\footnotesize \i};
}

\filldraw (1.32, 1.09) ellipse (0.03 and 0.005);
\filldraw (2, 1.17) ellipse (0.03 and 0.005);
\filldraw (3, 1.26) ellipse (0.03 and 0.005);
\filldraw (3.4, 1.23) ellipse (0.03 and 0.005);

\draw[dotted] (0,1.09) -- (4.2,1.09);
\draw[dotted] (0,1.17) -- (4.2,1.17);
\draw[dotted] (0,1.26) -- (4.6,1.26);
\draw[dotted] (0,1.23) -- (4.6,1.23);

\draw[dashed] (1.32,0.95) node[below,xshift=4mm] 
  {\footnotesize \parbox{1.4cm}{$t_1$\\Покупка}} -- (1.32,1.3);
\draw[dotted] (2,0.95) -- (2,1.3);
\draw[dotted] (3,0.95) -- (3,1.3);
\draw[dashed] (3.4,0.95) node[below,xshift=4mm] 
  {\footnotesize \parbox{1.45cm}{$t_2$\\Продажа}} -- (3.4,1.3);
\draw[dotted] (4,0.95) -- (4,1.3);

\draw [thick,->,xshift=2mm]
  (4,1.091) -- (4,1.169)
  node [black,midway,right]  {\footnotesize $M_2$};
\draw [thick,->,xshift=2mm]
  (4,1.171) -- (4,1.259)
  node [black,midway,right,yshift=-1mm]  {\footnotesize $M_3$};
\draw [thick,<-,xshift=6mm]
  (4,1.231) -- (4,1.259)
  node [black,midway,right]  {\footnotesize $M_4$};

\draw [thick] (1,1.33)--(1.2,1.33) node[right] {\color{black}\footnotesize \parbox{2cm}{Фьючерс}};
\draw [very thick,dotted] (2.2,1.33)--(2.4,1.33) node[right] {\color{black}\footnotesize Базовый актив};
\end{tikzpicture}
\caption{Пример расчетов по фьючерсу. }
\label{fut:fig}
\end{figure}

\begin{remark}
Далее мы будем считать все рассматриваемые фьючерсы расчетными.
Если базовый актив достаточно ликвидный и издержки поставки низкие, то цена поставочного фьючерса будет не сильно отличаться от цены расчетного; случай больших издержек поставки мы не рассматриваем.

Еще одно упрощение, которое мы делаем, состоит в том, что предполагается отсутствие \emph{гарантийного обеспечения}.
В реальности же при торговле фьючерсами на счетах покупателей и продавцов резервируется некоторая сумма денег, пропорциональная объему открытых позиций, чтобы гарантировать возможность уплаты вариационной маржи "--- она называется гарантийным обеспечением.
Если сумма денег на счете упадет ниже гарантийного обеспечения, то произойдет принудительное закрытие позиции.  
\end{remark}

\begin{remark}
Зачем нужны фьючерсы?
Во-первых, они позволяют легко торговать базовыми активами, непосредственная покупка/продажа которых затруднена.
Например, базовым активом фьючерса на индекс S\&P 500 является портфель из (чуть более) 500 акций, входящих в индекс, взятых в определенных пропорциях.
Постоянно поддерживать такие пропорции, торгуя самими акциями "--- крайне трудная задача как технически, так и в плане транзакционных издержек.
С помощью фьючерса можно получить соответствующую позицию гораздо проще.
Другим примером являются фьючерсы на сырье "--- например, торговля нефтью как базовым активом подразумевала бы транспортировку ее от продавцов к покупателям, что, конечно, трудно осуществить физически\footnote{Отметим, что большинство позиций по фьючерсам закрываются до экспирации, поэтому даже по поставочным фьючерсам объем поставки мал по сравнению с объемом торговли.}.

Во-вторых, фьючерсы позволяют легко торговать <<с плечом>>, \te\ фактически торговать на заемные деньги. 
А именно, чтобы купить, скажем, портфель акций, нужно непосредственно иметь соответствующее количество денег.
Если же трейдер такой суммой не располагает, но считает, что цена портфеля будет расти и хочет на этом заработать, то он может совершить сделку с плечом "--- взять денег взаймы у брокера.
Однако за использование заемных денег нужно платить.
Фьючерс же позволяет открыть позицию любого объема, при условии, что у трейдера достаточно денег для гарантийного обеспечения.
Так как гарантийное обеспечение составляет лишь небольшую сумму по сравнению с ценой базового актива, то это дает значительное плечо (например, гарантийное обеспечение в 10\%  соответствует 10-кратному плечу).
\end{remark}


\begin{remark}
\label{fut:r:mtm-contract}
Единственное, что привязывает фьючерс к базовому активу, "--- это расчетная цена в дату экспирации, равная цене базового актива (умноженной на объем лота). 
Если изменить принцип ее вычисления, но сохранив при этом сам механизм перечисления вариационной маржи, то получится другой маржируемый контракт.
Например, если взять последнюю расчетную цену равной $(S_T-K)^+$, где $S_T$ "--- цена базового актива в момент экспирации, а $K$ "--- фиксированная величина, то получится \emph{маржируемый опцион колл} (европейского типа).
Аналогично можно определить маржируемый опцион пут.

По этой причине в дальнейшем изложении будет удобно рассматривать произвольные \emph{маржируемые контракты} европейского типа\footnote{Маржируемые контракты американского типа в этом курсе не рассматриваются.}, у которых расчетная цена последнего клиринга представлена некоторой случайной величиной $F_T$.
Это не усложнит выкладки, но сделает результаты более общими.
Фьючерс получится, если положить $F_T=S_T$.
\end{remark}

\section{Цена маржируемого контракта на полном рынке}
\label{fut:s:complete}
В этом разделе мы введем понятие цен репликации фьючерсов и других маржируемых контрактов и покажем, как вычислить их на полном рынке.
Общий случай будет рассмотрен в следующем разделе. 

Пусть задана полная модель рынка, содержащая безрисковый актив с ценой $B_t$ и $N$ рисковых активов с ценами $S_t^n$.
Будем считать, что рисковые активы не выплачивают дивиденды.
Обозначим за $\Q$ единственную эквивалентную мартингальную меру.

Рассмотрим маржируемый контракт с временем экспирации $T$, расчетная цена которого в момент экспирации задается $\F_T$-измеримой случайной величиной $X$.
Для краткости, будем говорить просто <<маржируемый контракт $X$>>.
Если взять $X=S_T$, то получим фьючерс.

Пусть клиринги проводятся в каждый момент времени $t=1,2,\ldots,T$.

\begin{definition}
\label{fut:d:replication}
\emph{Ценой репликации} маржируемого контракта $X$ называется согласованная последовательность $F=(F_t)_{t=0}^T$ такая, что $F_T=X$ и существует торговая стратегия $\pi_t=(G_t,H_t)$ (вообще говоря, не самофинансируемая), которая для всех $t=1,\dots,T$ удовлетворяет равенствам
\begin{align}
\label{fut:complete-cur}
&G_tB_t + \scal{H_t}{S_t} = \Delta F_t,\\
\label{fut:complete-next}
&G_tB_{t-1} + \scal{H_t}{S_{t-1}} = 0.
\end{align}
\end{definition}

Смысл этого определения состоит в том, что с помощью стратегии $\pi$ можно получить денежный поток, в точности равный вариационной марже, начисляемой по контракту $X$.
А именно, в каждый момент времени $t=1,\dots,T$ портфель $\pi_t$, купленный в предыдущий момент времени за нулевую стоимость, можно продать по цене, равной вариационной марже $\Delta F_t$.
Покупка портфеля за нулевую стоимость соответствует тому, что открыть длинную или короткую позицию по маржируемому контракту можно бесплатно.
Отметим, что у стратегии $\pi$ отток капитала\footnote{Под оттоком капитала понимается величина $(G_t-G_{t+1})B_t + \scal{(H_t-H_{t+1})}{S_t}$, равная разности стоимостей покупаемого и продаваемого портфеля. Если эта величина отрицательная, то имеется приток капитала.} равен $\Delta F_t$, и, значит, она не является самофинансируемой в смысле определения в лекции \ref{ch:general}. 

Следующее предложение показывает, что цена репликации однозначно определена.

\begin{proposition}
Для любого маржируемого контракта $X$ на полном рынке цена репликации существует и задается формулой 
\begin{equation}
\label{fut:complete-price}
F_t = \E^\Q(X\mid \F_t).
\end{equation}
В частности, она является мартингалом относительно ЭММ.
\end{proposition}

\begin{proof}
Сначала покажем, что для так определенной цены $F_t$ существует реплицирующая стратегия.
Для каждого $t=1,\dots,T$ рассмотрим европейское платежное обязательство с выплатой $X^{(t)}=\Delta F_{t}$ в момент времени $t$.
Пусть $\pi^{(t)}=(\pi^{(t)}_s)_{s=1}^{t}$ "--- его (самофинансируемая) реплицирующая стратегия в модели рынка с $t$ периодами\footnote{Нетрудно показать (упражнение), что если модель с $T$ периодами полна, то и модель c $t\le T$ периодами и такими же активами тоже полна.}.
Условие репликации означает, что 
\begin{equation}
\label{fut:complete-proof-1}
V_t^{\pi^{(t)}} := G_t^{(t)}B_t + \scal{H_t^{(t)}}{S_t} = \Delta F_t.
\end{equation}
Заметим также, что $V_{t-1}^{\pi^{(t)}} = B_{t-1}/B_t \E^\Q(\Delta F_t\mid\F_{t-1}) = 0$, где первое равенство выполнено в силу того, что дисконтированная стоимость $V^{\pi^{(t)}}$ является мартингалом (а также воспользовались тем, что $B_t$ является $\F_{t-1}$-измеримой).

Определим теперь стратегию $\pi_t = (G_t, H_t)$ с компонентами $G_t= G_t^{(t)}$, $H_t=H_t^{(t)}$ для $t=1,\dots,T$.
Тогда равенство \eqref{fut:complete-cur} для нее верно в силу формулы \eqref{fut:complete-proof-1}, а равенство \eqref{fut:complete-next} "--- в силу того, что 
\[
G_{t+1}^{(t+1)} B_t + \scal{H_{t+1}^{(t+1)}}{S_t}
= G_{t}^{(t+1)} B_t + \scal{H_{t}^{(t+1)}}{S_t}
= V_t^{\pi^{(t+1)}} = 0,
\]
где в первом равенстве воспользовались тем, что $\pi^{(t+1)}$ самофинансируема.
Итак, $\pi$ является искомой реплицирующей стратегией.

Докажем теперь, что цена $F_t$ определена однозначно.
Воспользуемся обратной индукцией.
Для $t=T$ равенство \eqref{fut:complete-price} верно, так как $F_t=X$ по определению. 
Предположим, что оно доказано для $t$ и докажем его для $t-1$.
Заметим, что если $\pi$ "--- реплицирующая стратегия маржируемого контракта, то стратегия $\pi'$ с портфелями $\pi'_s=0$ для $s< t$ и $\pi'_t=\pi_t$ является самофинансируемой реплицирующей стратегией для платежного обязательства с выплатой $X' = \Delta F_t$ в момент $t$.
Так как $V_{t-1}^{\pi'} = 0$ согласно \eqref{fut:complete-next}, то получаем
\[
0 = V_{t-1}^{\pi'} = \frac{B_{t-1}}{B_t} \E^\Q(\Delta F_t\mid \F_{t-1})
= \E^\Q(F_t\mid \F_{t-1}) - F_{t-1},
\]
откуда следует справедливость шага индукции.

Наконец, утверждение о том, что $F_t$ "--- мартингал следует из того, что это мартингал Леви (заметим, что $\E|X| < \infty$ так как рынок полон и следовательно, любая случайная величина принимает лишь конечное число значений согласно расширенному варианту второй фундаментальной теоремы).
\end{proof}


\section{Общая модель рынка с маржируемыми контрактами}

Покажем, как расширить модель из лекции \ref{ch:general}, чтобы включить в нее торговлю маржируемыми контрактами.
Как обычно, будем считать, что дело происходит на некотором фильтрованном вероятностном пространстве.

\subsection{Активы и торговые стратегии}
\label{fut:ss:assets}

Пусть на рынке имеется $N+M+1$ актив: один безрисковый актив с ценой $B_t$, $N$ рисковых активов со стандартным принципом расчетов с ценами $S_t^n$ и $M$ маржируемых контрактов с ценами $F_t^m$.
Цена безрискового актива является строго положительной предсказуемой последовательностью, цены стандартных рисковых активов и маржируемых контрактов являются согласованными последовательностями, ограниченными снизу. 
Клиринг осуществляется в моменты времени $t=1,\dots,T$.
Для простоты рассуждений будем считать, что экспирация всех маржируемых контрактов происходит в момент $T$.
Обобщение на случай, когда некоторые контракты могут экспирироваться раньше или позже момента $T$ не представляет трудности, но сделает дальнейшие рассуждения громоздкими. 

Торговой стратегией будем называть любую предсказуемую последовательность $\pi=(\pi_t)_{t=1}^T$ вида $\pi_t=(G_t,H_t,K_t)$, где скалярная компонента $G_t$ выражает количество единиц безрискового актива в портфеле, $N$-мерная компонента $H_t$ "--- количество стандартных рисковых активов, а $M$-мерная компонента $K_t$ "--- количество открытых позиций по маржируемым контрактам (длинных, если $K_t> 0$ и коротких, если $K_t<0$).

Изменение стоимости портфеля стратегии $\pi$ описывается последовательностью $V^\pi=(V_t^\pi)_{t=0}^T$, определяемой следующим образом:
\begin{align}
\label{fut:v-0}
&V_0^\pi = G_1B_0 + \scal{H_1}{S_0},\\
\label{fut:v-t}
&V_t^\pi = G_tB_t + \scal{H_t}{S_t} + \scal{K_t}{\Delta F_t}, \quad t=1,\dots,T.
\end{align}
Выражение для $V_0^\pi$ здесь такое же, как в модели без маржируемых контрактов: это стоимость портфеля, покупаемого в момент $t=0$. При этом в портфеле могут быть открыты позиции по маржируемым контрактам, но так как расчет вариационной маржи произойдет только в клиринг $t=1$, это не отражается на стоимости портфеля.
Во второй формуле последнее слагаемое представляет собой вариационную маржу, зачисляемую или списываемую со счета.
Смысл обеих формул состоит в том, что если закрыть все позиции по безрисковому и рисковым активам в портфеле $\pi_t$, то, с учетом полученной (или списанной) вариационной маржи, останется в точности сумма денег $V_t^\pi$. 

Стратегия называется \emph{самофинансируемой}, если для всех $t=1,\dots,T-1$ выполнено равенство 
\begin{equation}
\label{fut:sf}
\scal{(H_{t+1}-H_t)}{S_t}  = -(G_{t+1}-G_t)B_t + \scal{K_t}{\Delta F_t}.
\end{equation}
Интерпретировать это равенство можно так, что в ходе торгов в момент времени $t$ изменения позиций по стандартным рисковым активам должно быть осуществлено за счет изменения позиции по безрисковому активу и полученной вариационной маржи. 
Заметим, что в формулу не входят изменения позиций по маржируемым контрактам, так как эти позиции можно открывать и закрывать без издержек в текущий момент (вариационная маржа будет рассчитана в следующий клиринг).

Дисконтированной стоимостью стандартных рисковых активов будем называть последовательности $\tilde S_t^n = S_t^n/B_t$, а дисконтированной стоимостью портфеля стратегии $\pi$ последовательность $\tilde V_t^\pi = V_t^\pi/B_t$.
Цены маржируемых контрактов дисконтировать не потребуется.

\begin{proposition}
Стратегия $\pi$ является самофинансируемой тогда и только тогда, когда выполнено любое из двух эквивалентных представлений:
\begin{align}
\label{fut:sf-eq-1}
&V_t^\pi = V_0^\pi + \sum_{u=1}^t (G_u\Delta B_u + \scal{H_u}{\Delta S_u} + \scal{K_t}\Delta F_t),\\
\label{fut:sf-eq-2}
&\tilde V_t^\pi = \tilde V_0^\pi + \sum_{u=1}^t (\scal{H_u}{\Delta \tilde S_u} + B_t^{-1}\scal{K_t}\Delta F_t).
\end{align}
\end{proposition}

\begin{proof}
Первое равенство следует из того, что если воспользоваться формулой \eqref{fut:v-t} для $V_t^\pi$ и $V_{t-1}^\pi$ и условием самофинансируемости, то для $t=1,\dots,T$ получим
\[
V_t^\pi - V_{t-1}^\pi = G_t\Delta B_t + \scal{H_t}{\Delta S_t} + \scal{K_t}\Delta F_t.
\]
Для доказательства второго равенства нужно проделать те же шаги, но предварительно поделив левые и правые части формул \eqref{fut:v-t} и \eqref{fut:sf} на $B_t$.
\end{proof}


\subsection{Безарбитражность и эквивалентные мартингальные меры}

\emph{Арбитражной возможностью}, как и в модели без маржируемых контрактов, будем называть самофинансируемые стратегию $\pi$ такую, что $V_0^\pi=0$, $V_T^\pi \ge 0$ \as\ и $\P(V_T^\pi>0)>0$. 
Модель называется безарбитражной (выполнена \emph{гипотеза NA}), если в ней не существует арбитражных возможностей. 

\begin{definition}
\label{fut:d:emm}
Мера $\Q\sim\P$ называется \emph{эквивалентной мартингальной мерой}, если выполнено любое из следующих двух равносильных условий:
\begin{alphenum}
\item дисконтированные цены стандартных рисковых активов $\tilde S_t^n$ и недисконтированные цены маржируемых контрактов $F_t^m$ являются мартингалами относительно $\Q$;
\item дисконтированная стоимость портфеля любой самофинансируемой стратегии $\tilde V_t^\pi$ является мартингальным преобразованием относительно $\Q$.
\end{alphenum}
\end{definition}

\begin{proposition}
Условия a) и  b) в определении \ref{fut:d:emm} равносильны.
\end{proposition}

\begin{proof}
Импликация a\,$\Rightarrow$\,b сразу следует из формулы \eqref{fut:sf-eq-2}.
Докажем импликацию b\,$\Rightarrow$\,a.

Сначала рассмотрим стратегию $\pi$, у которой компонента $H_t^n\equiv1$ для некоторого $n$, а все остальные компоненты равны 0.
Такая стратегия является самофинансируемой согласно формуле \eqref{fut:sf}, а ее стоимость $\tilde V_t^\pi = \tilde S_t^n$.
Следовательно, $\tilde S_t^n$ является мартингальным преобразованием, а, значит, и мартингалом в силу ограниченности снизу (см.~предложение \ref{mart:martingale-transform} в лекции \ref{ch:mart}).

Чтобы доказать мартингальность цен $F_t^m$, рассмотрим стратегию $\pi$, у которой $K_t^m = 1/B_t$, все компоненты $H^n_t$ и $K^j_t$, $j\neq m$, нулевые, а также $G_0 = F_0^m$, $G_t = G_{t-1} + \Delta F_t^m$. 
Согласно формуле \eqref{fut:sf} эта стратегия является самофинансируемой, а согласно $\eqref{fut:sf-eq-2}$ имеем $\tilde V_t^\pi = V_0^\pi + \sum_{u=1}^t \Delta F_u^m = F_t^m$.
Таким образом, $F_t^m$ является мартингальным преобразование и, значит, мартингалом (в силу ограниченности снизу).
\end{proof}

Первая фундаментальная теорема для рынка с маржируемыми контрактами остается без изменений.
Мы здесь не приводим ее доказательство, оно практически дословно повторяет доказательство в главе \ref{ch:general} и дополнении \ref{ch:ftap-proof}.

\begin{theorem}[первая фундаментальная теорема финансовой математики]
В модели рынка с маржируемыми контрактами отсутствие арбитража равносильно существованию эквивалентной мартингальной меры.
\end{theorem}

\begin{remark}[$*$]
Используя рассуждения, аналогичные приведенным в разделе \ref{gen:s:dividends} лекции \ref{ch:general}, можно дать дальнейшее обобщение рассматриваемой модели на случай, когда стандартные рисковые активы выплачивают дивиденды.
Перечислим без доказательств основные идеи и результаты в этом направлении.
\begin{itemize}
\item Условие самофинансируемости приобретает вид
\[
\scal{(H_{t+1}-H_t)}{S_t}  = -(G_{t+1}-G_t)B_t + \scal{H_t}{D_t} + \scal{U_t}{\Delta F_t}.
\]
\item Эквивалентной мартингальной мерой следует называть такую меру $\Q\sim\P$, что дисконтированная стоимость портфеля любой самофинансируемой стратегии является мартингальным преобразованием или, эквивалентно, мартингалами являются дисконтированные цены стандартных рисковых активов с учетом дивидендов и недисконтированные цены маржируемых контрактов.
\item Безарбитражные цены платежных обязательств вычисляются по таким же формулам, как в модели без дивидендов (см.~теорему \ref{fut:t:pricing} ниже).
\end{itemize}
\end{remark}


\subsection{Безарбитражные цены платежных обязательств}

\subsubsection{Общие результаты}
Будем отождествлять европейские платежные обязательства с $\F_T$"=измеримыми случайными величинами $X$.
Теперь у нас будут два типа обязательств: премиальные (со стандартным принципом расчета) и маржируемые. 

Идея определения безарбитражной цены, как и прежде, будет состоять в том, что нужно предположить, что платежное обязательство начинает торговаться на рынке, и найти такую цену, при которой в расширенной модели, состоящей из исходных активов и добавляемого платежного обязательства, не возникает арбитражных возможностей.
В зависимости от типа платежного обязательства, в расширенную модель добавляется либо стандартный рисковый актив (если обязательство премиальное), либо маржируемый контракт (если обязательство маржируемое).

\begin{definition}
\emph{Безарбитражной ценой} европейского платежного обязательства $X$ называется согласованная последовательность $V^X = (V_t^X)_{t=0}^\infty$ такая, что $V_T^X = X$ и расширенная модель рынка является безарбитражной, где расширенная модель определяется следующим образом:
\begin{itemize}
\item если обязательство $X$ премиальное, то в исходную модель добавляется стандартный рисковый актив с ценой $S_t^{N+1} = V_t^X$,
\item если обязательство $X$ маржируемое, то в исходную модель добавляется маржируемый контракт с ценой $F_t^{M+1} = V_t^X$.
\end{itemize}
\end{definition}

\begin{theorem}
\label{fut:t:pricing}
Пусть выплата $X$ ограничена снизу.
Если платежное обязательство премиальное, то последовательность $V_t$ является его безарбитражной ценой тогда и только тогда, когда для некоторой ЭММ $\Q$ выполнено равенство
\[
V_t = B_t \E^{\Q}\left(\frac{X}{B_T}\;\bigg|\; \F_t\right).
\]
Если платежное обязательство маржируемое, то последовательность $V_t$ является его безарбитражной ценой тогда и только тогда, когда для некоторой ЭММ $\Q$ выполнено равенство
\[
V_t = \E^{\Q}(X\mid \F_t).
\]
\end{theorem}

Утверждение этой теоремы для премиальных платежных обязательств уже было доказано в лекции \ref{ch:general}.
Для маржируемых оно доказывается совершенно аналогично, используя то, что, как мы показали выше, цена маржируемого контракта является мартингалом относительно ЭММ.


\subsubsection{Цены фьючерсов}
Фьючерсный контракт на рисковый актив можно отождествить с маржируемым платежным обязательством $X=S_T$, где $T$ "--- момент экспирации фьючерса, $S_t$ "--- цена рискового актива (для краткости обозначений опустим номер $n$). 
Отсюда получаем следующую формулу для его цены (будем называть ее \emph{$T$-фьючерсной ценой}).

\begin{proposition}
Безарбитражная $T$-фьючерсная задается формулой
\begin{equation}
\label{fut:futures-price}
F_t = \E^{\Q}(S_T \mid \F_t).
\end{equation}
\end{proposition}

Интересно сравнить $T$-фьючерсную цену с $T$-форвардной ценой.
Напомним, что последняя определяется как такая величина $F'_t$, что премиальное платежное обязательство $X' = S_T - F_t'$ имеет нулевую стоимость в момент $t$.
Форвардная цена была найдена в примере \ref{gen:e:forward} в лекции \ref{ch:general}: 
\[
F_t' = \frac{S_t}{B(t,T)} = \frac{\E^{\Q}(S_T/B_T\mid \F_t)}{\E^{\Q}(B_T^{-1} \mid \F_t)}. 
\]
Если безрисковая процентная ставка детерминирована, то можно сократить числитель и знаменатель на $B_T^{-1}$ и получить следующий результат о совпадении фьючерсных и форвардных цен (подчеркнем, что если безрисковая ставка случайная, то они, вообще говоря, будут различны).
Напомним, что в этом случае форвардная цена равна $F_t' = (B_T/B_t)S_t$ и не зависит от выбора ЭММ.

\begin{proposition}
\label{fut:p:equal-prices}
В модели с детерминированной безрисковой ставкой форвардные и фьючерсные цены совпадают.
\end{proposition}

Наконец, отметим, что если базовый рисковый актив не выплачивает дивиденды, то безарбитражная фьючерсная цена будет не ниже, чем текущая цена $S_t$ (при условии, что безрисковая ставка неотрицательна). 
Действительно, это следует из формулы
\[
F_t = \E^{\Q}(S_T^n\mid \F_t) \ge B_t\E^{\Q}\left(\frac{S_T^n}{B_T}\;\bigg|\; \F_t\right) = S_t^n,
\]
где воспользовались тем, что $B_t/B_T \le 1$ и $S_t^n/B_t$ является мартингалом.
Если же рисковый актив платит дивиденды, то $S_t^n = B_t/Y_t^n \E^\Q(S_T^nY_T^n/B_T \mid \F_t)$, где $Y_t^n$ выражает накопленную дивидендную доходность (см.~формулу \eqref{gen:div-yield} в лекции \ref{ch:general}).
Так как $Y_t^n/Y_T^n \le 1$, то может оказаться выполненным как неравенство $F_t\le S_t^n$, так и $F_t\ge S_t^n$.

\begin{remark}
Ситуация, когда фьючерсная цена выше текущей цены рискового актива ($F_t> S_t$) называется \emph{контанго}, а когда ниже "--- \emph{бэквордацией}.
\end{remark}

%! В следующем разделе хотелось бы, в частности, показать, что цены американских и маржируемых опционов
%! на фьючерсы совпадают, но не получается это сделать просто, не вводя расширение модели, где можно
%! торговать американскими маржируемыми контрактами.

% \subsubsection{\difficult\ Цены опционов на фьючерсы}
% Опцион на фьючерс "--- это контракт, который дает право покупателю в будущем открыть длинную (опцион колл) или короткую (опцион пут) фьючерсную позицию по фиксированной цене. Считается, что экспирация фьючерса происходит не раньше экспирации опционов. Опционы могут быть поставояными или рассчетными, европейскими или американскими, премиальными или маржируемыми. В этом разделе мы рассмотрим прмиальные опционы, а в следещбем "--- маржируемые.

% Как обычно, мы не будем делать различия между поставочными и рассчетными опционами (см.~замечание \ref{fut:r:premium-vs-delivery} ниже). Тогда премиальный европейский опцион колл с моментом экспирации $T$ и страйком $K$ на фьючерс, который эксирируется в момент $T'\ge T$, представляет собой премильанй контракт с выплатой величиын $X=(F_T-K)^+$ покупателю, где $F_T$ "--- цена фьючерса.
% Аналогично, премиальный опцион пут выплачивает $X=(K-F_T)^+$.
% Маржируемый европейский опцион колл можно отождествить с маржируемым контрактом с выплатой $X=(F_T-K)^+$, маржируемый опцион пут "--- с выплатой $X=(K-F_T)^+$. Тогда из результатов выше получаем следующее предложение.

% \begin{proposition}
% Пусть $F=(F_t)_{t=0}^{T'}$ задает цену фьючерса с моментом экспирации $T'\ge T$. Тогда безарбитражные цены премиальных европейских опционов колл и пут на этот фьючерс находятся по формулам
% \[
% \VC_t = B_t \E^\Q\left(\frac{(F_T-K)^+}{B_T} \;\bigg|\;\F_t\right), \qquad
% \VP_t = B_t \E^\Q\left(\frac{(K-F_T)^+}{B_T} \;\bigg|\;\F_t\right),
% \]
% а безарбитражные цены европейских маржируемых опционов "--- по формулам
% \[
% \VC_t = \E^\Q((F_T-K)^+ \;|\;\F_t), \qquad
% \VP_t = \E^\Q((K-F_T)^+ \;|\;\F_t).
% \]
% \end{proposition}

% \begin{remark}
% \label{fut:r:premium-vs-delivery}
% Поясним, почему в рамках нашей модели можно не делать различия между поставочнми и расчетными опционами. 

% Рассмотрим премиальный поставочный опцион колл. У покупателя этого опциона в момент $T$ появляется право открыть длинную позицию по фтючерсу, причем ценой открытия позици при клиринге в момент $T$ будет считаться величина $K$. Покупатель может тут же закрыть фьючерсную позицию по цене $F_T$, и, соответственно, получит в клиринг вариационную маржу $F_T-K$. Понятно, что исполнять опцион имеет смысл только если $F_T\ge K$, откуда слеудет, что выгода, поулчаемая покупателем, равна $(F_T-K)^+$ "--- точно такая же, как для рассчетного опциона.

% Аналогичные рассуждения проводятся и для премиального опциона пут, а также для маржируемых опционов обоих типов.
% \end{remark}

% Обратимся теперь к американским опционам. 


\summary
\begin{itemize}
\item Фьючерсом называется контракт на поставку базового актива в будущем с заложенным в него механизмом перечисления вариационной маржи. 

\item Безарбитражной ценой маржируемого платежного обязательства с расчетной ценой $X$ в момент экспирации $T$ является $F_t=\E^{\Q}(X\mid \F_t)$, где $\Q$ "--- эквивалентная мартингальная мера.
В частности, безарбитражная цена фьючерса на рисковый актив с ценой $S_t$ находится по формуле $F_t = \E^{\Q}(S_T\mid \F_t)$.

\item В случае полного рынка (с детерминированной процентной ставкой и без дивидендов) любой маржируемый контракт может быть реплицирован стратегией, торгующей только базовыми активами.

\item В случае детерминированной безрисковой ставки фьючерсная и форвардная цены совпадают и равны $F_t=(B_t/B_t)S_t$.
\end{itemize}
